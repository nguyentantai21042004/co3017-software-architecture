\documentclass[a4paper]{article}
% ===== Cấu hình gói và font cơ bản (tiếng Việt, toán, hình vẽ, bảng biểu) =====
\usepackage{vntex}
\usepackage{mathptmx}[ptm]
\usepackage{a4wide,amssymb,epsfig,latexsym,array,hhline,fancyhdr}
\usepackage[normalem]{ulem}
\usepackage{placeins}
\usepackage{ragged2e}
\usepackage[makeroom]{cancel}
\usepackage{amsmath}
\usepackage{amsthm}
\usepackage{multicol,longtable,amscd}
\usepackage{diagbox}
\usepackage{booktabs}
\usepackage{alltt}
\usepackage[framemethod=tikz]{mdframed}
\usepackage{caption,subcaption}
\usepackage{float}
\usepackage{lastpage}
\usepackage[lined,boxed,commentsnumbered]{algorithm2e}
\usepackage{enumerate}
\usepackage{enumitem}
\setlist{nosep, topsep=0pt, partopsep=0pt, parsep=0pt, itemsep=0.5em, leftmargin=1.5em}
\setlist[enumerate]{nosep, topsep=0pt, partopsep=0pt, parsep=0pt, itemsep=0.5em, leftmargin=1.5em}
\setlist[itemize]{nosep, topsep=0pt, partopsep=0pt, parsep=0pt, itemsep=0.5em, leftmargin=1.5em}
\usepackage{color}
\usepackage{graphicx}
\usepackage{array}
\usepackage{tabularx, caption}
\usepackage{multirow}
\usepackage{multicol}
\usepackage{rotating}
\usepackage{graphics}
\usepackage{geometry}
\usepackage{setspace}
\usepackage{epsfig}
\usepackage{tikz}
\usepackage{titlesec}
\usetikzlibrary{arrows,snakes,backgrounds,calc}
\usepackage[unicode]{hyperref}
\hypersetup{urlcolor=blue,linkcolor=black,citecolor=black,colorlinks=true}
\usepackage{listings}
\usepackage[normalem]{ulem}

% ===== Định nghĩa môi trường định lý/mệnh đề, đánh số, và ẩn số trang ở danh mục =====
\newtheorem{theorem}{{\bf Định lý}}
\newtheorem{property}{{\bf Tính chất}}
\newtheorem{proposition}{{\bf Mệnh đề}}
\newtheorem{corollary}[proposition]{{\bf Hệ quả}}
\newtheorem{lemma}[proposition]{{\bf Bổ đề}}
\theoremstyle{definition}
\newtheorem{exer}{Bài toán}
\addtocontents{toc}{\protect\thispagestyle{empty}}
\addtocontents{lof}{\protect\thispagestyle{empty}}
\addtocontents{lot}{\protect\thispagestyle{empty}}

% ===== Cấu hình khổ giấy và lề trang =====
\def\thesislayout{	% A4: 210 × 297
	\geometry{
		a4paper,
		total={160mm,240mm},  % fix over page
		left=30mm,
		top=22mm,
            right=20mm,
            bottom=20mm,
	}
}
\def\thesisheadlayout{	% A4: 210 × 297
	\geometry{
		a4paper,
		total={160mm,240mm},  % fix over page
		left=30mm,
		top=10mm,
	}
}
\thesislayout

% ===== Cấu hình hiển thị mã nguồn (listings) =====
\lstset{
	language=R,
	basicstyle=\footnotesize\sffamily,
	commentstyle=\ttfamily\color{black},
	numbers=left,
	numberstyle=\ttfamily\color{black}\footnotesize,
	stepnumber=1,
	numbersep=5pt,
	backgroundcolor=\color{white},
	showspaces=false,
	showstringspaces=false,
	showtabs=false,
	frame=single,
	tabsize=2,
	captionpos=b,
	breaklines=true,
	breakatwhitespace=false,
	title=\lstname,
	escapeinside={},
	keywordstyle={},
	morekeywords={}
}

% ===== Cấu hình header/footer (fancyhdr) =====
\setlength{\headheight}{40pt}
\pagestyle{fancy}
\fancyhead{} % clear all header fields
\renewcommand{\footruleskip}{1mm}

\fancyhead[L]{
	\begin{tabular}{rl}
		\begin{picture}(25,15)(0,0)
		\put(0,-8){\includegraphics[width=10mm, height=10mm]{images/hcmut.png}}
	\end{picture}&
		\begin{tabular}{l}
			\textbf{Đại Học Bách Khoa Thành Phố Hồ Chí Minh}\\
			\textbf{Khoa Khoa Học Máy Tính}
		\end{tabular} 	
	\end{tabular}
}

\fancyhead[R]{
	\begin{tabular}{l}
		\tiny \bf \\
		\tiny \bf 
	\end{tabular}  }

\fancyfoot{} % xóa footer mặc định
\fancyfoot[L]{\scriptsize  [CO3017] Kiến Trúc Phần Mềm - HK251}
\fancyfoot[R]{\scriptsize  Trang {\thepage}/\pageref{LastPage}}

\renewcommand{\headrulewidth}{0.3pt}
\renewcommand{\footrulewidth}{0.3pt}

%%%
\setcounter{secnumdepth}{4}
\setcounter{tocdepth}{3}
\makeatletter
\newcounter {subsubsubsection}[subsubsection]
\renewcommand\thesubsubsubsection{\thesubsubsection .\@alph\c@subsubsubsection}
\newcommand\subsubsubsection{\@startsection{subsubsubsection}{4}{\z@}%
									{-3.25ex\@plus -1ex \@minus -.2ex}%
									{1.5ex \@plus .2ex}%
									{\normalfont\normalsize\bfseries}}
\newcommand*\l@subsubsubsection{\@dottedtocline{3}{10.0em}{4.1em}}
\newcommand*{\subsubsubsectionmark}[1]{}
\makeatother

% ===== Thiết lập màu công thức, khoảng cách, caption, khoảng cách hình/bảng =====
\everymath{\color{black}}
\sloppy
\captionsetup[figure]{labelfont={small,bf},textfont={small},position=bottom,belowskip=8pt,aboveskip=8pt}
\captionsetup[table]{labelfont={small,bf},textfont={small},position=bottom,belowskip=8pt,aboveskip=8pt}
\setlength{\floatsep}{5pt plus 2pt minus 2pt}
\setlength{\textfloatsep}{5pt plus 2pt minus 2pt}
\setlength{\intextsep}{10pt plus 2pt minus 2pt}

\thesislayout
\onehalfspacing

% ===== Cấu hình khoảng cách đoạn văn đồng nhất =====
\setlength{\parskip}{0.3em}
\setlength{\parindent}{1.5em}

% ===== Cấu hình khoảng cách section/subsection đồng nhất =====
\titlespacing*{\section}{0pt}{0.2em}{0.2em}
\titlespacing*{\subsection}{0pt}{0.2em}{0.2em}
\titlespacing*{\subsubsection}{0pt}{0.2em}{0.2em}

% ===== Keyword để thụt đầu dòng =====
% Sử dụng \indentpar để thụt đầu dòng đoạn văn
\newcommand{\indentpar}{\indent}

% ===== Bắt đầu tài liệu =====
\onehalfspacing
\begin{document}

% ===== Trang tiêu đề (khung viền, logo, tên môn, tiêu đề) =====
\begin{titlepage}
\begin{tikzpicture}[remember picture, overlay]
	\draw[line width = 4pt] ($(current page.north west) + (0.4in,-0.5in)$) rectangle ($(current page.south east) + (-0.4in,0.5in)$);
	\draw[line width=1.5pt]
		($ (current page.north west) + (0.45in,-0.55in) $)
		rectangle
		($ (current page.south east) + (-0.45in,0.55in) $);
\end{tikzpicture}

\vspace{-0.5cm}

\begin{center}
	\Large \textbf{ĐẠI HỌC BÁCH KHOA THÀNH PHỐ HỒ CHÍ MINH} \\
	\Large \textbf{KHOA KHOA HỌC MÁY TÍNH}
\end{center}

\vspace{0.2cm}

\begin{figure}[h!]
\begin{center}
\includegraphics[width=6cm]{images/hcmut.png}
\end{center}
\end{figure}

\begin{center}
	\textbf{{\LARGE KIẾN TRÚC PHẦN MỀM (CO3017)}} \\
\end{center}

\vspace{-6pt}

\rule{\textwidth - 50pt}{1pt}

\begin{center}
	\textbf{\LARGE Intelligent Tutoring System - ITS}
\end{center}

\vspace{-6pt}

\rule{\textwidth - 50pt}{1pt}

\begin{table}[h]
    % Tăng khoảng cách dòng trong bảng bằng cách sử dụng arraystretch
    \renewcommand{\arraystretch}{2.0}
    \begin{tabularx}{\textwidth}{@{}p{3.7cm} l >{\raggedright\arraybackslash}X l@{}}
		& {\Large Advisor} & {\Large \dotfill} & \\
		& {\Large Group} & {\Large \dotfill} & \\
		& {\Large Students} & {\Large Nguyễn Tấn Tài} & {\Large 2212990} \\
		&  & {\Large \dotfill} & {\Large \dotfill} \\
		&  & {\Large \dotfill} & {\Large \dotfill} \\
    \end{tabularx}
    \renewcommand{\arraystretch}{1.0}
\end{table}

\vspace{6cm}

\begin{center}
	{\large 11/2025, Thành phố Hồ Chí Minh}
\end{center}

\end{titlepage}

\newpage

% ===== Mục lục =====
\newpage
\tableofcontents
\newpage
\listoffigures
\newpage
\listoftables
\newpage

% ===== Thiết lập trang và chèn nội dung chính =====
\setcounter{page}{1}

% Các file con chứa nội dung các chương/mục của báo cáo
\section{Tổng Quan Dự Án}

\noindent\textbf{Tầm Nhìn:} Xây dựng hệ thống học tập thông minh cá nhân hóa trải nghiệm cho từng người dùng, giúp học viên tiếp cận giáo dục chất lượng theo cách phù hợp năng lực và tiến độ riêng; đồng thời mở rộng để phục vụ hàng nghìn người dùng song song.

\noindent\textbf{Những Thách Thức Chính}
\begin{itemize}
    \item Đảm bảo phục vụ trên $5{,}000$ người dùng đồng thời với thời gian phản hồi dưới $500\,$ms
    \item Duy trì cân bằng giữa hiệu năng kỹ thuật và tốc độ triển khai
    \item Mở rộng theo chiều ngang, vẫn giữ hiệu quả học tập và chất lượng AI
    \item Hỗ trợ thử nghiệm, cập nhật mô hình AI không ảnh hưởng hệ thống vận hành
\end{itemize}

\noindent\textbf{Phương Pháp Kiến Trúc Được Chọn:} Áp dụng kiến trúc microservices lai (Hybrid Microservices) kết hợp mô hình hướng sự kiện (Event-Driven Architecture) nhằm tăng tính mô-đun, khả năng mở rộng, hiệu suất cao và kiểm thử toàn diện.

\noindent\textbf{Các Quyết Định Kiến Trúc Chính}

\begin{itemize}
    \item Microservices lai ghép + Event-Driven Architecture:
    Phân tách thành 5 services, gồm:
    \begin{itemize}
        \item Quản lý người dùng \textit{(User Management)}
        \item Nội dung \textit{(Content)}
        \item Mô hình học viên \textit{(Learner Model)}
        \item Công cụ thích ứng \textit{(Adaptive Engine)}
        \item Đánh giá \textit{(Assessment)}
    \end{itemize}
    Các service này kết hợp xử lý sự kiện không đồng bộ cho phân tích thời gian thực
    
    \item Đa ngôn ngữ lập trình (Polyglot Programming):
    \begin{itemize}
        \item Java/Spring Boot xử lý logic nghiệp vụ cốt lõi
        \item Golang phục vụ tác vụ tính toán hiệu năng cao, đặc biệt cho AI/ML
        \item PostgreSQL lưu trữ dữ liệu
        \item RabbitMQ truyền phát và xử lý sự kiện
    \end{itemize}
    
    \item Kiến trúc sạch (Clean Architecture): 
    Tách biệt tầng nghiệp vụ và hạ tầng, giúp dễ bảo trì, kiểm thử, mở rộng. Đảm bảo độ bao phủ kiểm thử trên $85\%$
    
    \item Điều phối bằng Kubernetes: 
    Hỗ trợ tự động mở rộng, triển khai blue-green tránh downtime, cơ chế tự phục hồi pod tối ưu hiệu suất hệ thống
    
    \item Khả năng quan sát:
    Hệ thống tích hợp giám sát:
    \begin{itemize}
        \item Prometheus/Grafana cho số liệu hệ thống
        \item Loki ghi nhật ký tập trung
    \end{itemize}
\end{itemize}

\noindent\textbf{Kết Quả Dự Kiến}

\begin{itemize}
    \item Hỗ trợ từ $5{,}000$ người dùng đồng thời, mở rộng đến $9{,}000$ người dùng\footnote{$9{,}000$ là số lượng trung bình sinh viên một trường đại học tại Việt Nam tính đến 04/2025, theo \url{https://giaoduc.net.vn/nhung-con-so-biet-noi-ve-giao-duc-dai-hoc-viet-nam-post250367.gd}}, đáp ứng quy mô tương đương trung bình một trường đại học ở Việt Nam
    \item Thời gian phản ứng nhỏ hơn $500\,$ms dưới tải cao
    \item Tính mô-đun: phát triển, triển khai dịch vụ độc lập
\end{itemize}

\newpage
\section{Phân Tích Bối Cảnh Và Yêu Cầu}

\subsection{Phạm vi và Mục tiêu Dự án}

\subsubsection{Tầm nhìn dự án}

\indentpar \indentpar Mục tiêu của dự án là xây dựng Hệ thống Gia sư Thông minh (Intelligent Tutoring System -- ITS) --- một nền tảng học tập được cá nhân hóa dựa trên trí tuệ nhân tạo (AI), có khả năng phân tích hành vi học tập, đánh giá năng lực và điều chỉnh lộ trình học cho từng người học một cách tự động.

Tầm nhìn của ITS là tái hiện trải nghiệm học một--kèm--một giữa người học và gia sư, giúp người học được hướng dẫn theo đúng tốc độ, năng lực và sở thích cá nhân; đồng thời đảm bảo hệ thống có thể mở rộng phục vụ hàng nghìn người dùng đồng thời, duy trì hiệu năng và chất lượng học tập ổn định.

\noindent\textit{``Tạo ra một hệ thống học tập thông minh có khả năng thích ứng, giúp mỗi người học đạt hiệu quả tối đa thông qua lộ trình học được cá nhân hóa bởi AI -- mang trải nghiệm gia sư riêng cho mọi người, ở bất kỳ đâu.''}

\subsubsection{Bối cảnh kinh doanh}

\noindent Nhu cầu thị trường

Sự bùng nổ của công nghệ và xu hướng học tập trực tuyến đã mở ra cơ hội lớn cho các hệ thống e-learning cá nhân hóa. Tuy nhiên, phần lớn các nền tảng hiện nay vẫn mang tính đại trà, chưa thể tự động thích ứng với trình độ, tiến độ và phong cách học tập riêng của từng người học.

\noindent Đối tượng người dùng
\begin{itemize}
    \item Học sinh, sinh viên (K--12 và đại học): Cần lộ trình học phù hợp năng lực, tránh học lại kiến thức đã biết
    \item Giảng viên / giáo viên: Cần công cụ giám sát tiến trình, tạo báo cáo, và gợi ý cải thiện cho từng học sinh
    \item Quản trị viên (Admin): Quản lý vận hành hệ thống, phân quyền và triển khai các phiên bản AI mới
\end{itemize}

\noindent Tiêu chí thành công
\begin{itemize}
    \item Hỗ trợ tối thiểu 5.000 người dùng đồng thời, thời gian phản hồi nhỏ hơn 500ms cho truy vấn phổ biến
    \item Hệ thống đạt SLA tối thiểu 99.5\% uptime, triển khai tính năng mới trong vòng 1 ngày
    \item Mức độ cá nhân hóa và khả năng giữ chân người học được cải thiện ít nhất 40\% so với nền tảng học trực tuyến truyền thống
    \item Áp dụng nguyên tắc SOLID và Clean Architecture giúp độ bao phủ kiểm thử tối thiểu 80\%
\end{itemize}

\subsubsection{Bối cảnh kỹ thuật}

\noindent Hệ thống hiện có và tích hợp:
\begin{itemize}
    \item Hệ thống quản lý học tập (LMS) hiện có thông qua API
    \item Dịch vụ xác thực người dùng (Auth Service) dùng JWT/OAuth2
    \item Hệ thống lưu trữ nội dung học tập (MinIO, PostgreSQL)
    \item Môi trường triển khai Kubernetes cho phép auto-scaling và CI/CD
\end{itemize}

\noindent Ràng buộc công nghệ
\begin{itemize}
    \item Ngôn ngữ chính: Java (Spring Boot) cho backend nghiệp vụ, Golang cho các dịch vụ hiệu năng cao (AI/ML)
    \item Cơ sở dữ liệu: PostgreSQL cho dữ liệu người học và nội dung; RabbitMQ cho giao tiếp sự kiện bất đồng bộ
    \item Kiến trúc: Microservices + Event-Driven, áp dụng Clean Architecture để tách biệt tầng nghiệp vụ và hạ tầng
    \item Áp dụng nguyên tắc SOLID cho thiết kế module độc lập và dễ kiểm thử
\end{itemize}

\noindent Kỳ vọng hiệu năng và khả năng mở rộng
\begin{itemize}
    \item Hiệu năng (Performance): API phản hồi nhỏ hơn $300\,$ms; tác vụ chấm điểm hoặc tạo lộ trình nhỏ hơn $1\,$s
    \item Khả năng mở rộng (Scalability): triển khai dạng Kubernetes cluster với Horizontal Pod Autoscaling (HPA) lớn hơn hoặc bằng $5{,}000$ người dùng đồng thời
    \item Khả năng quan sát (Observability): theo dõi hệ thống bằng Prometheus, Grafana, Loki cho phép phát hiện sự cố và phân tích hành vi người học theo thời gian thực
\end{itemize}
\newpage
\subsection{Phân tích các Bên Liên quan}

\indentpar \indentpar Phân tích các bên liên quan (Stakeholder) là bước quan trọng nhằm kết nối tầm nhìn hệ thống (Vision Statement) với các yêu cầu cụ thể (Functional \& Non-functional Requirements).
Mục tiêu của phần này là xác định ai chịu ảnh hưởng, ai ra quyết định, và ai trực tiếp sử dụng hệ thống, từ đó giúp định hướng kiến trúc phần mềm ITS theo nhu cầu thực tế của từng nhóm.

\subsubsection{Ma trận Bên Liên quan}

\begin{table}[!htbp]
    \centering
    \small
    \renewcommand{\tabularxcolumn}[1]{m{#1}}
    \renewcommand{\arraystretch}{1.8}
    \begin{tabularx}{\textwidth}{|>{\centering\arraybackslash}m{2.2cm}|>{\centering\arraybackslash}m{2.5cm}|>{\centering\arraybackslash}m{2cm}|>{\centering\arraybackslash}m{2cm}|>{\centering\arraybackslash}X|}
        \hline
        \textbf{Bên liên quan} & \textbf{Vai trò}                      & \textbf{Quan tâm} & \textbf{Ảnh hưởng} & \textbf{Mối quan tâm chính}                                            \\
        \hline
        Learner                & Người dùng cuối (End User)            & Cao               & Cao                & Trải nghiệm học tập cá nhân hóa, phản hồi nhanh, giao diện thân thiện. \\
        \hline
        Instructor             & Người tạo và giám sát nội dung        & Cao               & Cao                & Quản lý nội dung, báo cáo hiệu suất học tập, dễ sử dụng.               \\
        \hline
        Admin                  & Chủ sở hữu \& vận hành hệ thống       & Trung bình        & Cao                & Bảo mật, chi phí, khả năng mở rộng, triển khai không downtime.         \\
        \hline
        AI Engineer            & Phát triển mô hình thích ứng          & Trung bình        & Trung bình         & Dễ tích hợp thuật toán mới, giám sát hiệu suất mô hình.                \\
        \hline
        System Architect       & Thiết kế \& duy trì cấu trúc hệ thống & Trung bình        & Cao                & Đảm bảo mô-đun hóa, khả năng test, tuân thủ Clean Architecture.        \\
        \hline
    \end{tabularx}
    \caption{Ma trận Bên Liên quan}
    \label{tab:stakeholder_matrix}
\end{table}
\FloatBarrier

\subsubsection{Biểu đồ Ma trận Quyền lực/Sự quan tâm (Power/Interest Grid)}

\begin{figure}[H]
    \centering
    \begin{tikzpicture}[scale=1.2, transform shape]
        % Draw axes
        \draw[->, thick] (0,0) -- (10,0) node[right] {\textbf{Sự quan tâm (Interest)}};
        \draw[->, thick] (0,0) -- (0,10) node[above] {\textbf{Quyền lực (Power)}};

        % Draw quadrant lines
        \draw[dashed] (5,0) -- (5,10);
        \draw[dashed] (0,5) -- (10,5);

        % Labels for quadrants
        \node[align=center, font=\bfseries] at (2.5, 2.5) {MONITOR\\(Giám sát)};
        \node[align=center, font=\bfseries] at (7.5, 2.5) {KEEP INFORMED\\(Cung cấp thông tin)};
        \node[align=center, font=\bfseries] at (2.5, 7.5) {KEEP SATISFIED\\(Làm hài lòng)};
        \node[align=center, font=\bfseries] at (7.5, 7.5) {MANAGE CLOSELY\\(Quản lý chặt chẽ)};

        % Plot stakeholders
        % Learner: High Power, High Interest
        \node[circle, fill=blue!30, draw, minimum size=1.5cm, align=center] at (8, 8.5) {Learner};

        % Instructor: High Power, High Interest
        \node[circle, fill=blue!30, draw, minimum size=1.5cm, align=center] at (8, 6.5) {Instructor};

        % Admin: High Power, Medium Interest -> High Power, Low Interest (Keep Satisfied)
        \node[circle, fill=green!30, draw, minimum size=1.5cm, align=center] at (2.5, 8.5) {Admin};

        % Architect: High Power, Medium Interest -> Keep Satisfied
        \node[circle, fill=green!30, draw, minimum size=1.5cm, align=center] at (4, 6.5) {Architect};

        % AI Engineer: Medium Power, Medium Interest -> Middle
        \node[circle, fill=yellow!30, draw, minimum size=1.5cm, align=center] at (5, 5) {AI Engineer};

    \end{tikzpicture}
    \caption{Ma trận Quyền lực/Sự quan tâm (Power/Interest Grid)}
    \label{fig:stakeholder_grid}
\end{figure}

\noindent\textbf{Ý nghĩa liên kết:}

Các stakeholder trên tương ứng trực tiếp với các ``actors'' đã được mô tả trong Use Case Diagram và quyết định phạm vi kiến trúc ITS ở nhiều cấp độ khác nhau:
\begin{itemize}
    \item Learner và Instructor là trung tâm của Vision Statement -- hệ thống được thiết kế xoay quanh họ.
    \item Admin và Architect đảm bảo các mục tiêu phi chức năng.
    \item AI Engineer đóng vai trò cầu nối giữa Domain Model và Adaptive Engine, giúp hệ thống mở rộng thuật toán mà không vi phạm nguyên tắc SRP/DIP (Single Responsibility Principle/Dependency Inversion Principle).
\end{itemize}

\subsubsection{Yêu cầu và Mối quan hệ}

\begin{table}[H]
    \centering
    \footnotesize
    \renewcommand{\tabularxcolumn}[1]{m{#1}}
    \renewcommand{\arraystretch}{1.4}
    \begin{tabularx}{\textwidth}{|>{\centering\arraybackslash}m{2cm}|>{\centering\arraybackslash}X|>{\centering\arraybackslash}X|>{\centering\arraybackslash}X|>{\centering\arraybackslash}X|>{\centering\arraybackslash}X|}
        \hline
        \textbf{Stakeholder}                                                                                                                                 & \textbf{Nhu cầu chính} & \textbf{Kỳ vọng} & \textbf{Ràng buộc} & \textbf{Liên kết} & \textbf{Chỉ số thành công} \\
        \hline
        \textbf{Learner} (Người học)                                                                                                                         &
        Cá nhân hóa lộ trình học theo năng lực và tiến độ. Phản hồi tức thì sau khi nộp bài. Giao diện thân thiện, dễ sử dụng trên mọi thiết bị.             &
        Tốc độ phản hồi $< 500$ms (AC3 -- Performance). Gợi ý học lại (Remediation) chính xác $\geq 80\%$.                                                   &
        Không có kỹ năng kỹ thuật, cần UI/UX trực quan.                                                                                                      &
        Ánh xạ thành User Stories US0--US3. Hiện thực trong Adaptive Engine Service và Feedback Service.                                                     &
        $\geq 85\%$ người học hoàn thành lộ trình gợi ý. Mức hài lòng $\geq 4.5/5$.                                                                                                                                                                                            \\
        \hline
        \textbf{Instructor} (Giảng viên)                                                                                                                     &
        Quản lý và gắn thẻ metadata nội dung học tập. Xem báo cáo tổng hợp và chi tiết học sinh. Hỗ trợ đánh giá và cập nhật lộ trình giảng dạy.             &
        Dashboard phản hồi trong $< 1$ giây. Hệ thống tự động phát hiện học viên yếu để gợi ý can thiệp.                                                     &
        Phải thao tác được trên giao diện quản trị, không can thiệp backend.                                                                                 &
        Phản ánh trong User Stories US4--US6 và Functional Requirements FR3, FR8. Liên kết với Content Management Service và Dashboard module.               &
        $\geq 90\%$ nội dung học tập có metadata đầy đủ. Thời gian tạo báo cáo lớp học $< 1$ giây.                                                                                                                                                                             \\
        \hline
        \textbf{Admin} (Quản trị viên)                                                                                                                       &
        Quản lý người dùng, vai trò, bảo mật và logs. Quản lý vận hành mô hình AI, deploy blue/green không downtime.                                         &
        SLA $\geq 99.9\%$ uptime. Mọi thay đổi được ghi lại trong audit logs.                                                                                &
        Giới hạn tài nguyên (K8s cluster), yêu cầu chi phí thấp.                                                                                             &
        Mô hình hóa trong User Stories US7--US8 và Non-functional Requirements AC6--AC7. Tác động đến Deployment Diagram (phần $2.4$), CI/CD và autoscaling. &
        Không gián đoạn dịch vụ trong $100\%$ các đợt deploy AI mới. Thời gian rollback $< 5$ phút.                                                                                                                                                                            \\
        \hline
        \textbf{AI Engineer} (Kỹ sư AI)                                                                                                                      &
        Dễ dàng tích hợp hoặc thay thế mô hình gợi ý học tập (Adaptive Algorithm). Có môi trường test độc lập cho A/B testing.                               &
        ---                                                                                                                                                  &
        ---                                                                                                                                                  &
        Liên quan tới Architecture Decision -- AdaptivePathGenerator (phần Domain Services). Ảnh hưởng tới AC1 (Extensibility) và AC4 (Testability).         &
        Thêm hoặc thay thế thuật toán mới mà không cần sửa code service khác.                                                                                                                                                                                                  \\
        \hline
        \textbf{System Architect} (Kiến trúc sư)                                                                                                             &
        Đảm bảo Clean Architecture, tuân thủ SOLID. Giám sát hệ thống qua Prometheus/Grafana/Jaeger.                                                         &
        ---                                                                                                                                                  &
        ---                                                                                                                                                  &
        Quyết định đến thiết kế C4 Diagram, Module Diagram, Sequence Diagram (phần $2.x$). Đảm bảo mọi thành phần triển khai độc lập (Microservices).        &
        Code coverage $\geq 80\%$. Mỗi service deploy độc lập không ảnh hưởng hệ thống.                                                                                                                                                                                        \\
        \hline
    \end{tabularx}
    \renewcommand{\arraystretch}{1.0}
    \caption{Yêu cầu và Mối quan hệ}
    \label{tab:stakeholder_requirements}
\end{table}
\FloatBarrier

\newpage
\subsection{Yêu Cầu Chứng Năng}

\indentpar \indentpar Phần này mô tả chi tiết các yêu cầu chức năng của hệ thống ITS (Intelligent Tutoring System), được hình thành từ User Stories, Use Cases và Domain Model đã phân tích ở các phần trước.
Mục tiêu là đảm bảo mỗi chức năng đều phản ánh nhu cầu của stakeholder, liên kết trực tiếp với Vision Statement.

\subsubsection{User Stories}

\indentpar \indentpar Các User Stories mô tả hành vi mong đợi của người dùng theo dạng ``As a [role], I want [goal] so that [benefit]''.
Mỗi story được gắn kèm tiêu chí chấp nhận (Acceptance Criteria) để đảm bảo có thể kiểm chứng được trong kiểm thử và đánh giá. 

Các User Stories này được ánh xạ trực tiếp sang các Use Case và được hiện thực hóa trong Domain Model.

\small
\setlength{\tabcolsep}{3pt}
\begin{longtable}{|>{\centering\arraybackslash}m{0.8cm}|>{\centering\arraybackslash}m{1.2cm}|>{\centering\arraybackslash}m{1.6cm}|>{\noindent\justifying\arraybackslash}p{5.8cm}|>{\noindent\justifying\arraybackslash}p{5.8cm}|}
\caption{User Stories}
\label{tab:user_stories}
\\
\hline
\textbf{STT} & \textbf{Actor} & \textbf{Phạm vi} & \textbf{User Story} & \textbf{Tiêu chí Chấp nhận} \\
\hline
\endfirsthead
\caption[]{User Stories (tiếp theo)}
\\
\hline
\textbf{STT} & \textbf{Actor} & \textbf{Phạm vi} & \textbf{User Story} & \textbf{Tiêu chí Chấp nhận} \\
\hline
\endhead
\hline
\endfoot
\hline
\endlastfoot
\textbf{US0} & \textbf{Learner} & \textbf{Cá nhân hóa} & 
Là một \textbf{Học sinh}, tôi muốn \textbf{hệ thống đánh giá kiến thức hiện tại của tôi} để nó có thể \textbf{đề xuất lộ trình học tập tối ưu}, không lãng phí thời gian vào những gì tôi đã biết. & 
$1.$ Hệ thống phải cung cấp một bài kiểm tra đầu vào (diagnostic test).\newline
$2.$ Lộ trình học tập được tạo ra phải bỏ qua các chủ đề mà học sinh đã đạt $> 90\%$ trong bài kiểm tra.\newline
$3.$ Lộ trình phải ưu tiên các chủ đề mà học sinh yếu nhất (ví dụ: $< 30\%$ điểm). \\
\hline
\textbf{US1} & \textbf{Learner} & \textbf{Phản hồi} & 
Là một \textbf{Học sinh}, tôi muốn \textbf{nhận được gợi ý (hints) và giải thích ngay lập tức} sau khi tôi mắc lỗi trong bài tập, để tôi có thể \textbf{tự sửa chữa và hiểu được khái niệm đó ngay lập tức}. & 
$1.$ Khi trả lời sai một câu hỏi, một nút ``Gợi ý'' (Hint) xuất hiện.\newline
$2.$ Gợi ý phải liên quan trực tiếp đến lỗi sai.\newline
$3.$ Sau khi nộp bài, hệ thống hiển thị giải thích chi tiết cho từng câu trả lời sai. \\
\hline
\textbf{US2} & \textbf{Learner} & \textbf{Đánh giá} & 
Là một \textbf{Học sinh}, tôi muốn \textbf{xem tiến trình học tập của mình} (\textbf{điểm}, \textbf{thời gian hoàn thành}, \textbf{các kỹ năng đã thành thạo}) để tôi có thể \textbf{theo dõi sự cải thiện của bản thân}. & 
$1.$ Dashboard cá nhân hiển thị điểm trung bình chung.\newline
$2.$ Có một biểu đồ trực quan hóa mức độ thành thạo của từng kỹ năng (ví dụ: ``Đại số: $80\%$'').\newline
$3.$ Lịch sử làm bài (thời gian, điểm số) được ghi lại và có thể xem lại. \\
\hline
\textbf{US3} & \textbf{Learner} & \textbf{Vòng lặp học tập} & 
Là một \textbf{Học sinh}, tôi muốn \textbf{hệ thống tự động đưa lại bài tập về các kỹ năng tôi chưa thành thạo sau một khoảng thời gian}, để \textbf{củng cố kiến thức đã học}. & 
$1.$ Hệ thống phải theo dõi ngày cuối cùng học sinh luyện tập một kỹ năng.\newline
$2.$ Nếu một kỹ năng $< 70\%$ và đã $> 7$ ngày chưa luyện tập, hệ thống tự động thêm bài ôn tập vào lộ trình.\newline
$3.$ Các bài tập ôn tập được đánh dấu là ``Ôn tập'' (Review). \\
\hline
\textbf{US4} & \textbf{Instructor} & \textbf{Nội dung} & 
Là một \textbf{Giảng viên}, tôi muốn \textbf{gắn metadata} (\textbf{độ khó}, \textbf{kỹ năng}, \textbf{chủ đề}) cho mỗi bài tập mới để \textbf{thuật toán cá nhân hóa có thể sử dụng chúng một cách chính xác}. & 
$1.$ Form tạo bài tập mới phải có các trường bắt buộc: ``Độ khó'' (Dropdown: Dễ, TB, Khó), ``Kỹ năng liên quan'' (Tag input).\newline
$2.$ Không thể lưu bài tập nếu thiếu metadata bắt buộc.\newline
$3.$ Giảng viên có thể tạo/thêm các ``Kỹ năng'' mới vào hệ thống. \\
\hline
\textbf{US5} & \textbf{Instructor} & \textbf{Giám sát} & 
Là một \textbf{Giảng viên}, tôi muốn \textbf{xem báo cáo tổng hợp về hiệu suất của cả lớp} để tôi có thể \textbf{xác định những chủ đề mà đa số học sinh đang gặp khó khăn}. & 
$1.$ Báo cáo hiển thị điểm trung bình của cả lớp cho từng chủ đề.\newline
$2.$ Báo cáo làm nổi bật $3$ kỹ năng/chủ đề có tỷ lệ làm sai cao nhất.\newline
$3.$ Dữ liệu báo cáo có thể được xuất ra file CSV. \\
\hline
\textbf{US6} & \textbf{Instructor} & \textbf{Báo cáo chi tiết} & 
Là một \textbf{Giảng viên}, tôi muốn \textbf{tạo báo cáo chi tiết về hiệu suất và lộ trình học tập của một học sinh cụ thể}, để tôi có thể \textbf{tư vấn cá nhân hóa (one-on-one)}. & 
$1.$ Giảng viên có thể chọn một học sinh từ danh sách lớp.\newline
$2.$ Báo cáo hiển thị lộ trình học tập đầy đủ của học sinh đó.\newline
$3.$ Báo cáo so sánh thời gian học sinh dành cho một chủ đề so với trung bình của lớp. \\
\hline
\textbf{US7} & \textbf{Admin} & \textbf{Quản trị} & 
Là một \textbf{Quản trị viên}, tôi muốn \textbf{quản lý các tài khoản Giảng viên} và \textbf{phân quyền truy cập nội dung} để đảm bảo \textbf{tính bảo mật} và \textbf{kiểm soát hệ thống}. & 
$1.$ Admin có thể tạo, vô hiệu hóa, hoặc xóa tài khoản Giảng viên.\newline
$2.$ Admin có thể gán vai trò (ví dụ: Giảng viên, TA) cho tài khoản.\newline
$3.$ Admin có thể thiết lập quyền truy cập của một Giảng viên vào một khóa học cụ thể. \\
\hline
\textbf{US8} & \textbf{Admin} & \textbf{Quản lý Vận hành} & 
Là một \textbf{Quản trị viên}, tôi muốn \textbf{có khả năng deploy/swap (thay đổi) các phiên bản mới của Mô hình AI (ví dụ: thuật toán gợi ý mới) mà không cần downtime hệ thống chính}, để \textbf{đảm bảo Modularity và Deployability (ACs quan trọng cho ITS)}. & 
$1.$ Hệ thống hỗ trợ triển khai Blue/Green hoặc Canary cho service AI.\newline
$2.$ Hệ thống chính (Học sinh) không bị gián đoạn (lỗi $503$) trong quá trình deploy.\newline
$3.$ Admin có thể rollback về phiên bản AI trước đó trong vòng $5$ phút nếu có lỗi. \\
\hline
\end{longtable}
\normalsize

\subsubsection{Use Cases}

\indentpar \indentpar Các Use Case cụ thể hóa cách người dùng tương tác với hệ thống để đạt được mục tiêu nghiệp vụ.
Chúng đóng vai trò là cầu nối giữa yêu cầu người dùng và kiến trúc kỹ thuật, giúp xác định rõ:
\begin{itemize}
    \item Actor nào tham gia,
    \item Điều kiện trước/sau khi thực hiện,
    \item Luồng chính và các luồng thay thế.
\end{itemize}

\small
\setlength{\tabcolsep}{2pt}
\renewcommand{\tabularxcolumn}[1]{m{#1}}
\begin{longtable}{|>{\centering\arraybackslash}m{1cm}|>{\centering\arraybackslash}m{2.5cm}|>{\centering\arraybackslash}m{2.5cm}|>{\centering\arraybackslash}m{2cm}|>{\centering\arraybackslash}m{1.5cm}|>{\centering\arraybackslash}m{5.5cm}|}
\caption{Use Cases}
\label{tab:use_cases}
\\
\hline
\textbf{Usecase ID} & \textbf{Tên Usecase} & \textbf{Mục đích} & \textbf{Tác nhân} & \textbf{FR liên quan} & \textbf{Luồng Cơ bản (Basic Flow)} \\
\hline
\endfirsthead
\caption[]{Use Cases (tiếp theo)}
\\
\hline
\textbf{Usecase ID} & \textbf{Tên Usecase} & \textbf{Mục đích} & \textbf{Tác nhân} & \textbf{FR liên quan} & \textbf{Luồng Cơ bản (Basic Flow)} \\
\hline
\endhead
\hline
\endfoot
\hline
\endlastfoot
\textbf{UC-01} & Đăng ký Tài khoản & Tạo tài khoản mới trong hệ thống. & Learner, Instructor, Admin & FR1 & $1.$ User truy cập trang đăng ký. $2.$ Nhập email, password, chọn role (Learner/Instructor). $3.$ Hệ thống validate và tạo tài khoản. $4.$ Gửi email xác nhận. $5.$ User xác nhận email và kích hoạt tài khoản. \\
\hline
\textbf{UC-02} & Đăng nhập \& Xác thực & Cho phép người dùng đăng nhập và truy cập hệ thống. & Learner, Instructor, Admin & FR1, FR11 & $1.$ User nhập email và password. $2.$ Hệ thống xác thực thông tin. $3.$ Kiểm tra role và phân quyền (RBAC). $4.$ Tạo session và chuyển đến dashboard tương ứng với role. \\
\hline
\textbf{UC-03} & Cập nhật Hồ sơ \& Cài đặt Học tập & Learner cập nhật thông tin cá nhân và tùy chọn học tập. & Learner & FR2 & $1.$ Learner truy cập trang hồ sơ. $2.$ Cập nhật tên, tuổi, trình độ, sở thích, mục tiêu, lịch học. $3.$ Thiết lập nhắc nhở (email/push). $4.$ Hệ thống lưu thông tin vào LearnerProfile. \\
\hline
\textbf{UC-04} & Thực hiện Bài kiểm tra Đầu vào & Đánh giá kiến thức ban đầu để xây dựng Learner Model. & Learner & FR2, FR5 & $1.$ Learner bắt đầu diagnostic test. $2.$ Hệ thống hiển thị câu hỏi đa dạng về kỹ năng. $3.$ Learner trả lời. $4.$ Hệ thống chấm điểm và tạo SkillMasteryScore. $5.$ Kết quả lưu vào LearnerModel. \\
\hline
\textbf{UC-05} & Tạo Khóa học \& Nội dung Học tập & Instructor tạo khóa học, chương, bài học với đa dạng định dạng. & Instructor & FR3 & $1.$ Instructor tạo khóa học mới. $2.$ Tạo chương và bài học (text, video, slide, quiz, coding task). $3.$ Cấu hình versioning và phân quyền (public/private/group). $4.$ Lưu vào ContentAggregate. \\
\hline
\textbf{UC-06} & Gắn Metadata \& Tagging cho Nội dung & Instructor gắn metadata để hỗ trợ thuật toán AI. & Instructor & FR3, FR4 & $1.$ Instructor chọn nội dung đã tạo. $2.$ Gắn tags: kỹ năng, độ khó, chủ đề. $3.$ Hệ thống lưu MetadataTag. $4.$ ContentMetadata có sẵn cho Adaptive Engine. \\
\hline
\textbf{UC-07} & Cấu hình Lộ trình Khóa học & Instructor thiết lập mục tiêu, milestones, điều kiện mở khóa bài học. & Instructor & FR4 & $1.$ Instructor định nghĩa mục tiêu khóa học và kỹ năng yêu cầu. $2.$ Thiết lập pre-test, post-test. $3.$ Cấu hình điều kiện mở khóa (ví dụ: $\geq 70\%$ điểm quiz). $4.$ Lưu cấu trúc lộ trình. \\
\hline
\textbf{UC-08} & Bắt đầu/Tiếp tục Học tập Thích ứng & Cung cấp bài học tiếp theo tối ưu dựa trên Learner Model. & Learner & FR7, FR4 & $1.$ Learner yêu cầu bài học tiếp theo. $2.$ Hệ thống gọi Adaptive Engine (FR7). $3.$ Engine đọc LearnerModel và ContentMetadata. $4.$ Đề xuất ContentID tối ưu (spaced repetition, mastery-based). $5.$ Hiển thị nội dung. \\
\hline
\textbf{UC-09} & Làm Bài tập \& Assessment & Learner thực hiện bài tập (MCQ, essay, coding, upload, project). & Learner & FR5 & $1.$ Learner mở bài tập. $2.$ Đọc đề và trả lời (trong thời gian giới hạn nếu có). $3.$ Submit câu trả lời. $4.$ Hệ thống lưu vào gradebook. \\
\hline
\textbf{UC-10} & Chấm điểm \& Phản hồi Tức thì & Hệ thống chấm điểm và cung cấp phản hồi/gợi ý ngay lập tức. & Learner & FR5, FR6 & $1.$ Learner gửi câu trả lời (FR5). $2.$ Scoring/Feedback Service chấm điểm (auto-grading hoặc manual review). $3.$ Tạo feedback, hints, giải thích đáp án. $4.$ Hiển thị Score và Hint ($< 500$ms). $5.$ Cập nhật LearnerModel. \\
\hline
\textbf{UC-11} & Gợi ý Bài học Bù (Remediation) & Đề xuất bài học bổ sung khi Learner yếu kỹ năng. & Learner & FR6, FR7 & $1.$ Hệ thống phát hiện kỹ năng yếu từ LearnerModel. $2.$ FeedbackGenerator tạo gợi ý bài học liên quan. $3.$ Hiển thị danh sách bài học bù với hướng dẫn step-by-step. $4.$ Learner chọn bài học để học lại. \\
\hline
\textbf{UC-12} & Xem Dashboard \& Tiến độ Học tập & Learner xem tiến độ, điểm số, milestones. & Learner & FR8 & $1.$ Learner truy cập dashboard. $2.$ Hệ thống hiển thị tiến độ, điểm số, lịch học, milestones, skill mastery. $3.$ Learner có thể xuất báo cáo (CSV/PDF). \\
\hline
\textbf{UC-13} & Xem Báo cáo Tổng hợp Lớp & Instructor xem tổng quan hiệu suất của cả lớp. & Instructor & FR8 & $1.$ Instructor chọn lớp. $2.$ Hệ thống tạo báo cáo tổng hợp: điểm trung bình, điểm yếu phổ biến, phân bố kỹ năng. $3.$ Instructor phân tích và điều chỉnh nội dung. \\
\hline
\textbf{UC-14} & Tạo Báo cáo Chi tiết Học sinh & Instructor tạo báo cáo cá nhân hóa cho một học sinh. & Instructor & FR8 & $1.$ Instructor chọn học sinh. $2.$ Hệ thống truy xuất LearnerModel, ProgressRecord. $3.$ Tạo báo cáo: lộ trình học, điểm mạnh/yếu, thời gian học. $4.$ Xuất PDF/CSV để tư vấn one-on-one. \\
\hline
\textbf{UC-15} & Tương tác \& Thảo luận & Learner/Instructor tham gia thảo luận, chat, bình luận. & Learner, Instructor & FR9 & $1.$ User truy cập diễn đàn hoặc bài học. $2.$ Gửi comment/câu hỏi. $3.$ Hệ thống gửi thông báo realtime (in-app/email/push) cho người liên quan. $4.$ User khác trả lời. \\
\hline
\textbf{UC-16} & Quản lý Lớp \& Phân nhóm & Instructor tạo lớp, mời học sinh, chia nhóm. & Instructor & FR10 & $1.$ Instructor tạo lớp mới. $2.$ Mời học sinh qua email/link. $3.$ Phân vai trò (TA, student, observer). $4.$ Chia nhóm cho project. $5.$ Giao bài nhóm và đánh giá theo nhóm. \\
\hline
\textbf{UC-17} & Quản lý Người dùng \& Phân quyền (RBAC) & Admin quản lý tài khoản và phân quyền chi tiết. & Admin & FR1, FR11 & $1.$ Admin truy cập trang quản lý users. $2.$ Tạo/sửa/xóa tài khoản. $3.$ Gán role và permissions. $4.$ Mọi thao tác ghi vào audit logs. $5.$ User chỉ truy cập tính năng được phép. \\
\hline
\textbf{UC-18} & Hoán đổi Mô hình AI (Live Swap) & Triển khai phiên bản AI mới không downtime. & Admin & FR12 & $1.$ Admin yêu cầu triển khai Model V2. $2.$ Deployment Service chạy V2 song song với V1. $3.$ Traffic chuyển dần sang V2 (Blue/Green/Canary). $4.$ Monitoring kiểm tra health. $5.$ Ngừng V1 khi V2 ổn định. \\
\hline
\textbf{UC-19} & Giám sát \& Vận hành Hệ thống & Admin quản lý cấu hình, backup, logs, moderation. & Admin & FR12 & $1.$ Admin truy cập admin panel. $2.$ Kiểm tra health checks, logs hệ thống. $3.$ Thực hiện backup/restore dữ liệu. $4.$ Xử lý báo cáo vi phạm (moderation). $5.$ Cấu hình hệ thống (feature flags, limits). \\
\hline
\textbf{UC-20} & Nhận Phần thưởng \& Gamification & Learner nhận XP, badges, tham gia leaderboard. & Learner & FR13 & $1.$ Learner hoàn thành bài học/milestone. $2.$ Hệ thống tính XP, trao badge. $3.$ Cập nhật leaderboard. $4.$ Hiển thị streak learning, challenge mode. $5.$ Learner được động viên tiếp tục học. \\
\hline
\end{longtable}
\normalsize

\FloatBarrier

\noindent\textbf{Sơ đồ minh họa Use Case chính:}

\begin{figure}[ht]
    \centering
    \begin{minipage}{0.48\textwidth}
        \centering
        \includegraphics[width=\textwidth]{images/usecase_9.png}
    \end{minipage}
    \hfill
    \begin{minipage}{0.48\textwidth}
        \centering
        \includegraphics[width=\textwidth]{images/usecase_10.png}
    \end{minipage}
    \caption{Use Case: Làm Bài tập \& Assessment (trái) và Chấm điểm \& Phản hồi Tức thì (phải)}
    \label{fig:usecase-9-10}
\end{figure}

\begin{figure}[ht]
    \centering
    \includegraphics[width=0.6\textwidth]{images/usecase_11.png}
    \caption{Use Case: Gợi ý Bài học Bù}
    \label{fig:usecase-11}
\end{figure}

\FloatBarrier

\subsubsection{Domain Model}

\indentpar \indentpar Phần Domain Model mô tả cấu trúc logic nghiệp vụ cốt lõi của Hệ thống Gia sư Thông minh (Intelligent Tutoring System -- ITS).
Nó là cầu nối giữa phân tích yêu cầu (Use Cases) và thiết kế kiến trúc (Architecture Design), giúp xác định các thực thể (entities), ranh giới (aggregates) và dịch vụ miền (domain services).

\noindent\textbf{Aggregates}

Trong ITS, mỗi Aggregate đại diện cho một nhóm thực thể có quan hệ nghiệp vụ chặt chẽ, được quản lý bởi một Aggregate Root duy nhất.
Các Aggregates được xác định dựa trên hành vi nghiệp vụ và tần suất thay đổi, nhằm đảm bảo tính mô-đun (modularity), tính mở rộng (scalability) và tính tách biệt (separation of concerns).

\begin{table}[ht]
\centering
\small
\renewcommand{\tabularxcolumn}[1]{m{#1}}
\renewcommand{\arraystretch}{1.65}
\begin{tabularx}{\textwidth}{|>{\centering\arraybackslash}m{4cm}|>{\noindent\justifying\arraybackslash}X|}
\hline
\textbf{Aggregate} & \textbf{Trách nhiệm chính (Responsibility)} \\
\hline
LearnerAggregate & Quản lý thông tin hồ sơ cá nhân, tiến trình học tập, lịch sử hoạt động của người học. \\
\hline
LearnerModelAggregate & Đại diện cho mô hình tri thức (AI Model) của từng học viên -- lưu trữ điểm thành thạo kỹ năng, lịch sử đánh giá, trạng thái BKT. \\
\hline
ContentAggregate & Quản lý nội dung học tập, khóa học, chương, bài học và metadata phục vụ thuật toán cá nhân hóa. \\
\hline
AdaptivePathAggregate & Đại diện cho lộ trình học tập được tạo động bởi Adaptive Engine dựa trên LearnerModel. \\
\hline
UserManagementAggregate & Quản lý người dùng, xác thực (AuthN), phân quyền (AuthZ), và nhật ký hoạt động (audit logs). \\
\hline
\end{tabularx}
\renewcommand{\arraystretch}{1.0}
\caption{Aggregates}
\label{tab:aggregates}
\end{table}

\noindent\textbf{Nguyên tắc phân tách Aggregates:}
\begin{itemize}
    \item Mỗi Aggregate có transaction boundary riêng biệt.
    \item Giao tiếp giữa các Aggregate thực hiện qua Domain Events hoặc Message Queue (Kafka/RabbitMQ) để đảm bảo eventual consistency.
    \item Các Aggregate có tần suất cập nhật cao (như LearnerModelAggregate) được tách riêng để giảm độ coupling với phần dữ liệu ít thay đổi.
\end{itemize}

\noindent\textbf{Entities}

Các Entity là các đối tượng nghiệp vụ có định danh duy nhất và được quản lý trong phạm vi của một Aggregate.

\begin{table}[ht]
\centering
\small
\renewcommand{\tabularxcolumn}[1]{m{#1}}
\renewcommand{\arraystretch}{1.75} % <-- Tăng khoảng cách dòng trong bảng này
\begin{tabularx}{\textwidth}{|>{\centering\arraybackslash}m{3.5cm}|>{\centering\arraybackslash}m{3.5cm}|>{\noindent\justifying\arraybackslash}X|}
\hline
\textbf{Aggregate thuộc về} & \textbf{Entity chính} & \textbf{Mô tả ngắn gọn} \\
\hline
LearnerAggregate & Learner, LearnerProfile, ProgressRecord & Lưu thông tin người học, mục tiêu học tập và tiến trình hoàn thành nội dung. \\
\hline
LearnerModelAggregate & LearnerModel, SkillMasteryScore, DiagnosticResult & Mô tả trạng thái kiến thức hiện tại và mức độ thành thạo kỹ năng theo mô hình Bayesian Knowledge Tracing. \\
\hline
ContentAggregate & Course, Chapter, ContentUnit, MetadataTag, Assessment & Cấu trúc khóa học và các bài tập tương ứng với từng kỹ năng. \\
\hline
AdaptivePathAggregate & AdaptivePath, PathNode, RecommendationScore & Biểu diễn lộ trình học tập cá nhân hóa gồm nhiều nội dung được sắp xếp dựa trên điểm yếu của người học. \\
\hline
UserManagementAggregate & User, Role, Permission, AuditLog & Đảm bảo bảo mật và phân quyền truy cập trong toàn hệ thống. \\
\hline
\end{tabularx}
\renewcommand{\arraystretch}{1.0}
\caption{Entities}
\label{tab:entities}
\end{table}

\noindent\textbf{Lưu ý:} Tất cả các Entity đều tuân thủ nguyên tắc SRP (Single Responsibility Principle) -- mỗi lớp chỉ chịu trách nhiệm một phần của nghiệp vụ, tránh chồng chéo dữ liệu.

\noindent\textbf{Value Objects}

Value Objects là các đối tượng bất biến (immutable), được so sánh theo giá trị thay vì danh tính.
Trong ITS, chúng được sử dụng để tăng tính an toàn dữ liệu và tính tái sử dụng trong nhiều Aggregates.

\begin{table}[ht]
\centering
\small
\renewcommand{\tabularxcolumn}[1]{m{#1}}
\renewcommand{\arraystretch}{1.7} % <-- tăng độ cao cho mỗi hàng
\begin{tabularx}{\textwidth}{|>{\centering\arraybackslash}m{3.5cm}|>{\noindent\justifying\arraybackslash}X|}
\hline
\textbf{Value Object} & \textbf{Vai trò} \\
\hline
MetadataTag & Đại diện cho thẻ (tag) mô tả kỹ năng, chủ đề hoặc độ khó của nội dung học tập. \\
\hline
RecommendationScore & Điểm đánh giá đề xuất nội dung dựa trên sự phù hợp với LearnerModel. \\
\hline
ProgressRecord & Ghi lại tiến trình học tập, trạng thái hoàn thành, và thời gian học của người học. \\
\hline
DiagnosticResult & Kết quả của bài kiểm tra đầu vào (diagnostic test), dùng để khởi tạo LearnerModel. \\
\hline
\end{tabularx}
\renewcommand{\arraystretch}{1.0} % khôi phục arraystretch mặc định cho các bảng sau
\caption{Value Objects}
\label{tab:value_objects}
\end{table}

\noindent\textbf{Lưu ý:} Các Value Object được thiết kế theo nguyên tắc OCP (Open-Closed Principle): có thể mở rộng để thêm thuộc tính (ví dụ thêm loại thẻ metadata mới) mà không cần sửa đổi lớp gốc.

\noindent\textbf{Domain Services}

Domain Services là nơi chứa logic nghiệp vụ phức tạp không thuộc riêng một Entity hoặc Aggregate nào, được thiết kế theo DIP (Dependency Inversion Principle) để có thể hoán đổi dễ dàng các thuật toán hoặc cách triển khai.

\begin{table}[ht]
\centering
\small
\renewcommand{\tabularxcolumn}[1]{m{#1}}
\renewcommand{\arraystretch}{1.7} % <-- tăng độ cao cho mỗi hàng
\begin{tabularx}{\textwidth}{|>{\centering\arraybackslash}m{3.5cm}|>{\noindent\justifying\arraybackslash}X|}
\hline
\textbf{Domain Service} & \textbf{Mô tả \& Vai trò nghiệp vụ} \\
\hline
AdaptivePathGenerator & Tạo lộ trình học tập cá nhân hóa dựa trên dữ liệu của LearnerModel và metadata của nội dung. Cho phép thay đổi thuật toán (rule-based / ML-based) mà không ảnh hưởng service khác. \\
\hline
ScoringEngine & Xử lý việc chấm điểm tự động cho nhiều loại bài tập (quiz, essay, coding). Có thể tích hợp AI model (như BERT) cho essay grading. \\
\hline
FeedbackGenerator & Tạo phản hồi (hints, giải thích, bài học bù) dựa trên mẫu lỗi học sinh thường gặp. Sử dụng NLP hoặc rule-based engine. \\
\hline
RemediationEngine & Phân tích kỹ năng yếu, đề xuất bài học bù hoặc bài ôn tập theo nguyên tắc spaced repetition. \\
\hline
AuthenticationService & Quản lý xác thực (login, JWT issuance) và phân quyền truy cập. \\
\hline
\end{tabularx}
\renewcommand{\arraystretch}{1.0} % khôi phục arraystretch mặc định cho các bảng sau
\caption{Domain Services}
\label{tab:domain_services}
\end{table}

\noindent\textbf{Lưu ý:} Mỗi Domain Service tương ứng với một Microservice độc lập trong kiến trúc ITS, đảm bảo khả năng triển khai độc lập (deployability) và test độc lập (testability).

Ví dụ: AdaptivePathGenerator nằm trong Adaptive Engine Service, ScoringEngine và FeedbackGenerator nằm trong Evaluation \& Feedback Services.

\noindent\textbf{Domain Events}

Domain Events phản ánh các thay đổi quan trọng trong trạng thái hệ thống.
Chúng là cơ chế giúp các microservices giao tiếp theo mô hình event-driven, giảm phụ thuộc trực tiếp giữa các module.

\begin{table}[ht]
\centering
\small
\renewcommand{\tabularxcolumn}[1]{m{#1}}
\renewcommand{\arraystretch}{1.7} % <-- tăng độ cao cho mỗi hàng
\begin{tabularx}{\textwidth}{|>{\centering\arraybackslash}m{3cm}|>{\centering\arraybackslash}m{2.5cm}|>{\centering\arraybackslash}m{2.5cm}|>{\noindent\justifying\arraybackslash}X|}
\hline
\textbf{Domain Event} & \textbf{Nguồn phát sinh (Publisher)} & \textbf{Người tiêu thụ (Consumer)} & \textbf{Mục đích} \\
\hline
SubmissionCompleted & ScoringEngine & LearnerModel Service & Cập nhật kỹ năng học viên sau khi nộp bài. \\
\hline
LearnerModelUpdated & LearnerModel Service & Adaptive Engine & Kích hoạt quá trình tạo lại lộ trình học mới. \\
\hline
FeedbackGenerated & Feedback Service & Dashboard Service & Hiển thị phản hồi tức thì trên giao diện học viên. \\
\hline
PathGenerated & Adaptive Engine & Cache/Redis + Dashboard & Lưu trữ và hiển thị lộ trình học mới. \\
\hline
UserCreated / RoleAssigned & Auth Service & Admin Service & Đồng bộ quyền truy cập và nhật ký hệ thống. \\
\hline
\end{tabularx}
\renewcommand{\arraystretch}{1.0} % khôi phục lại mặc định cho các bảng sau
\caption{Domain Events}
\label{tab:domain_events}
\end{table}

\noindent\textbf{Lưu ý:} Các sự kiện này được truyền qua Kafka hoặc RabbitMQ, đảm bảo tính asynchronous communication và eventual consistency giữa các dịch vụ.

Ví dụ: khi SubmissionCompleted được phát đi, LearnerModel Service sẽ tự động cập nhật điểm thành thạo kỹ năng mà không cần gọi API đồng bộ.

\noindent\textbf{Domain Model Class Diagram}

Sơ đồ dưới đây minh họa mối quan hệ giữa các Aggregate và Entity chính, thể hiện cách Domain Model được tổ chức logic theo DDD và chuẩn bị cho việc phân rã thành Microservices trong kiến trúc tổng thể.

\begin{figure}[ht]
    \centering
    \includegraphics[width=1.0\textwidth]{images/domain_model_class_diagram.png}
    \caption{Domain Model Class Diagram}
    \label{fig:domain-model-class-diagram}
\end{figure}

\FloatBarrier

\newpage
\subsection{Yêu Cầu Phi Chức Năng}

\indentpar \indentpar Các yêu cầu phi chức năng đóng vai trò định hình “chất lượng kiến trúc” của hệ thống ITS.
Chúng là nền tảng cho mọi quyết định thiết kế — từ lựa chọn công nghệ, mô hình triển khai, cho đến phương thức kiểm thử và giám sát.
Phần này tập trung làm rõ những đặc tính kiến trúc (Architecture Characteristics) và thuộc tính chất lượng (Quality Attributes) giúp ITS vận hành hiệu quả, ổn định và dễ mở rộng trong môi trường học tập quy mô lớn.
\subsubsection{Đặc Tính Kiến Trúc}

\indentpar \indentpar Phần này trình bày chi tiết các đặc tính kiến trúc (Architecture Characteristics) cốt lõi được rút ra từ yêu cầu nghiệp vụ, đóng vai trò nền tảng cho mọi quyết định kỹ thuật trong giai đoạn thiết kế và triển khai.
Mỗi đặc tính được mô tả dưới dạng mục tiêu định lượng (SLO) và hàm kiểm thử tự động (Fitness Function) để bảo đảm chúng có thể được đo lường và duy trì trong suốt vòng đời dự án.

\textbf{a. Các Đặc tính Kiến trúc Chính (Primary Architecture Characteristics)}

\newcolumntype{Y}{>{\raggedright\arraybackslash}X}
\renewcommand{\arraystretch}{1.35}
\begin{table}[ht]
    \centering
    \small
    \begin{tabularx}{\textwidth}{|>{\centering\arraybackslash}p{1.1cm}|>{\centering\arraybackslash}p{3.1cm}|Y|Y|Y|}
        \hline
        \textbf{AC}   & \textbf{Đặc tính} & {\centering\textbf{Định nghĩa \& Tầm quan trọng}\par}                                                                                                                                                                  & {\centering\textbf{Metrics \& Mục tiêu}\par}                                                                                                                                                                                         & {\centering\textbf{Fitness Function}\par}                                                                                                                                      \\
        \hline
        \textbf{AC-1} & Modularity        & \textbf{Định nghĩa:} Phân rã hệ thống thành các module độc lập với low coupling, high cohesion.\newline \textbf{Tầm quan trọng:} Bảo đảm live AI model swapping không downtime, cho phép cập nhật thuật toán liên tục. & \textbf{Metrics:} $I \approx 0$, $C_{\text{e}} < 5$, LCOM $< 1$.\newline \textbf{Mục tiêu:} Thay module $< 5$ phút, zero downtime khi swap AI, $\leq 3$ dependencies/module lõi.                                                     & \textbf{ArchUnit Test:} Fail nếu domain phụ thuộc repository cụ thể.\newline \textbf{Dependency Check:} Fail khi build phát hiện circular dependencies.                        \\
        \hline
        \textbf{AC-2} & Scalability       & \textbf{Định nghĩa:} Đáp ứng tải tăng bằng horizontal scaling mà không tái thiết kế kiến trúc.\newline \textbf{Tầm quan trọng:} Phục vụ người học tăng trưởng theo khu vực, giờ cao điểm.                              & \textbf{Metrics:} Concurrent users, auto-scaling response time, CPU/Memory per request.\newline \textbf{Mục tiêu:} $\geq 5{,}000$ concurrent users, auto-scale $< 60$s, tăng gấp đôi resource $\rightarrow$ gấp đôi throughput.      & \textbf{K6 Load Test:} Staging đạt p99 $< 1$s, error rate $< 0.1\%$ tại $5{,}000$ users.\newline \textbf{HPA Monitoring:} Alert nếu CPU $> 70\%$ trong $> 30$s mà không scale. \\
        \hline
        \textbf{AC-3} & Performance       & \textbf{Định nghĩa:} Đáp ứng yêu cầu nhanh để giữ trải nghiệm học tập mượt mà.\newline \textbf{Tầm quan trọng:} Critical cho phản hồi remedial real-time trong phiên 1-kèm-1.                                          & \textbf{Metrics:} Response time percentiles, throughput, cache hit ratio, query time.\newline \textbf{Mục tiêu:} p95 $< 500$ms, heavy ops p99 $< 1.5$s, cache hit $> 90\%$, $95\%$ query $< 100$ms.                                  & \textbf{CI Latency Test:} Fail nếu /api/feedback p95 $> 500$ms.\newline \textbf{Performance Budget:} Alert khi page load $> 2$s hoặc TTI $> 3$s.                               \\
        \hline
        \textbf{AC-4} & Testability       & \textbf{Định nghĩa:} Dễ xác minh tính đúng đắn của AI và business logic phức tạp.\newline \textbf{Tầm quan trọng:} Bảo đảm độ tin cậy cho adaptive learning và scoring engine.                                         & \textbf{Metrics:} Code coverage, test execution time, test cases/feature, cyclomatic complexity.\newline \textbf{Mục tiêu:} Unit coverage $\geq 80\%$, integration $\geq 70\%$, full E2E cho critical paths, test suite $< 10$ phút. & \textbf{Coverage Gate:} Fail nếu ScoringEngine hoặc AdaptiveEngine $< 80\%$.\newline \textbf{Mutation Testing:} Mutation score $> 70\%$ cho module trọng yếu.                  \\
        \hline
    \end{tabularx}
    \caption{Các Đặc tính Kiến trúc Chính}
    \label{tab:primary-architecture-characteristics}
\end{table}
\renewcommand{\arraystretch}{1.0}

Sau khi xác định được các đặc tính trọng yếu ảnh hưởng đến hình dạng kiến trúc tổng thể, phần tiếp theo trình bày những đặc tính bổ trợ (Secondary ACs) — tập trung vào khía cạnh vận hành, bảo trì và mở rộng tính năng trong môi trường thực tế.
Các đặc tính này không thay đổi cấu trúc hệ thống, nhưng quyết định trải nghiệm triển khai, vận hành và giám sát của đội ngũ kỹ thuật.

\vspace{7em}

\textbf{b. Các Đặc tính Kiến trúc Bổ trợ (Secondary Architecture Characteristics)}

\renewcommand{\arraystretch}{1.35}
\begin{table}[ht]
    \centering
    \small
    \begin{tabularx}{\textwidth}{|>{\centering\arraybackslash}p{1.1cm}|>{\centering\arraybackslash}p{3.3cm}|Y|Y|Y|}
        \hline
        \textbf{AC}   & \textbf{Đặc tính} & {\centering\textbf{Định nghĩa \& Tầm quan trọng}\par}                                                                                                                                       & {\centering\textbf{Metrics \& Mục tiêu}\par}                                                                                                                                                                 & {\centering\textbf{Fitness Function}\par}                                                                                                                                    \\
        \hline
        \textbf{AC-5} & Deployability     & \textbf{Định nghĩa:} Khả năng triển khai phiên bản mới nhanh và độc lập.\newline \textbf{Tầm quan trọng:} Hỗ trợ continuous delivery, live AI model swapping mà không gián đoạn người dùng. & \textbf{Metrics:} Deployment frequency, lead time, rollback time, success rate.\newline \textbf{Mục tiêu:} Deploy $< 15$ phút, rollback $< 5$ phút, zero downtime, success rate $> 95\%$.                    & \textbf{Canary Deployment Test:} 1\% user, tiếp tục nếu error rate $=0\%$ sau 5 phút.\newline \textbf{Health Check:} Tự động rollback nếu health endpoint fail trong 2 phút. \\
        \hline
        \textbf{AC-6} & Security          & \textbf{Định nghĩa:} Bảo vệ dữ liệu và hệ thống khỏi các mối đe doạ.\newline \textbf{Tầm quan trọng:} Đảm bảo an toàn PII, kết quả học tập, nội dung bản quyền.                             & \textbf{Metrics:} Critical vulnerabilities, patch time, auth failure rate, encryption coverage.\newline \textbf{Mục tiêu:} Zero critical vulns, TLS~1.3/AES-256, MFA, vá lỗi $< 24$ giờ.                     & \textbf{OWASP Dependency Check:} Fail build nếu phát hiện critical vulnerability.\newline \textbf{Penetration Test:} Kiểm thử tự động hàng quý, zero critical findings.      \\
        \hline
        \textbf{AC-7} & Maintainability   & \textbf{Định nghĩa:} Dễ hiểu, sửa đổi, cải tiến hệ thống.\newline \textbf{Tầm quan trọng:} Giảm chi phí vận hành, tăng tốc fix bug và bổ sung tính năng.                                    & \textbf{Metrics:} Cyclomatic complexity, technical debt, MTTR, code duplication.\newline \textbf{Mục tiêu:} Complexity $< 10$ (90\% methods), debt $< 5\%$, duplication $< 3\%$, MTTR $< 4$ giờ.             & \textbf{SonarQube Quality Gate:} Fail nếu complexity $> 10$ hoặc duplication $> 5\%$.\newline \textbf{Code Review Requirement:} Merge cần $\geq 2$ approvals.                \\
        \hline
        \textbf{AC-8} & Extensibility     & \textbf{Định nghĩa:} Thêm tính năng mới mà không sửa core code.\newline \textbf{Tầm quan trọng:} Thích ứng với nhu cầu giáo dục và công nghệ mới.                                           & \textbf{Metrics:} Extension points, time to add feature, breaking changes.\newline \textbf{Mục tiêu:} Plugin cho content types, thêm assessment mới không chạm core, breaking changes $< 5\%$/major release. & \textbf{OCP Compliance Test:} Xác nhận mở rộng qua config/plugins.\newline \textbf{API Compatibility:} Test tự động bảo đảm backward compatibility.                          \\
        \hline
        \textbf{AC-9} & Observability     & \textbf{Định nghĩa:} Hiểu trạng thái hệ thống qua logs, metrics, traces.\newline \textbf{Tầm quan trọng:} Phát hiện sớm sự cố trong hệ thống phân tán.                                      & \textbf{Metrics:} Log coverage, MTTD, dashboard coverage, alert accuracy.\newline \textbf{Mục tiêu:} 100\% requests có trace ID, MTTD $< 5$ phút, đủ dashboards, alert noise $< 10\%$.                       & \textbf{Trace Coverage Test:} Alert nếu $< 95\%$ requests có trace ID.\newline \textbf{SLO Monitoring:} Tự động on-call khi SLO breach detected.                             \\
        \hline
    \end{tabularx}
    \caption{Các Đặc tính Kiến trúc Bổ trợ}
    \label{tab:secondary-architecture-characteristics}
\end{table}
\renewcommand{\arraystretch}{1.0}

Những đặc tính kiến trúc ở trên định nghĩa “chất lượng bên trong” của hệ thống (internal quality).
Phần tiếp theo – Quality Attributes – tập trung vào “chất lượng bên ngoài” (external quality) mà người dùng và tổ chức có thể cảm nhận trực tiếp, như tốc độ phản hồi, tính ổn định, độ tin cậy, bảo mật, khả năng sử dụng và tương thích.

\subsubsection{Thuộc Tính Chất Lượng}

\indentpar \indentpar Các thuộc tính chất lượng (Quality Attributes) được nhóm theo từng khía cạnh đo lường cụ thể,
phản ánh mức độ mà hệ thống ITS đáp ứng các Service-Level Objectives (SLOs) về hiệu năng, khả năng mở rộng, bảo mật, độ tin cậy và trải nghiệm người dùng.

\vspace{12em}

\textbf{a. Yêu cầu Hiệu năng (Performance Requirements)}%
\renewcommand{\arraystretch}{1.3}
\begin{table}[H]
    \centering
    \small
    \begin{tabularx}{\textwidth}{|>{\centering\arraybackslash}m{3.2cm}|>{\centering\arraybackslash}X|>{\centering\arraybackslash}X|}
        \hline
        \textbf{Nhóm}                         & \textbf{Hạng mục}             & \textbf{Mục tiêu định lượng}  \\
        \hline
        \multirow{7}{*}{Response Time}        & Login/Authentication          & $< 200$ms (p95)               \\
        \cline{2-3}
                                              & Content Loading               & $< 500$ms (p95)               \\
        \cline{2-3}
                                              & Quiz/Assessment Submission    & $< 300$ms (p95)               \\
        \cline{2-3}
                                              & Real-time Feedback Generation & $< 500$ms (p95)               \\
        \cline{2-3}
                                              & Report Generation             & $< 2$ giây (p95)              \\
        \cline{2-3}
                                              & Search Operations             & $< 400$ms (p95)               \\
        \cline{2-3}
                                              & Dashboard Loading             & $< 1$ giây (p95)              \\
        \hline
        \multirow{5}{*}{Throughput}           & API Gateway                   & $5{,}000$ requests/giây       \\
        \cline{2-3}
                                              & Assessment Service            & $1{,}000$ submissions/giây    \\
        \cline{2-3}
                                              & Content Delivery              & $10{,}000$ concurrent streams \\
        \cline{2-3}
                                              & Analytics Processing          & $100{,}000$ events/phút       \\
        \cline{2-3}
                                              & Notification Service          & $50{,}000$ messages/phút      \\
        \hline
        \multirow{5}{*}{Resource Utilization} & CPU Usage                     & $< 70\%$ ở tải bình thường    \\
        \cline{2-3}
                                              & Memory Usage                  & $< 80\%$ dung lượng cấp phát  \\
        \cline{2-3}
                                              & Database Connection Pool      & $< 80\%$ sử dụng              \\
        \cline{2-3}
                                              & Network Bandwidth             & $< 60\%$ công suất            \\
        \cline{2-3}
                                              & Storage I/O                   & $< 10$ms latency trung bình   \\
        \hline
    \end{tabularx}
    \caption{Yêu cầu Hiệu năng}
    \label{tab:performance-requirements}
\end{table}
\renewcommand{\arraystretch}{1.0}

\textbf{b. Yêu cầu Khả năng Mở rộng}%
\renewcommand{\arraystretch}{1.3}
\begin{table}[H]
    \centering
    \small
    \begin{tabularx}{\textwidth}{|>{\centering\arraybackslash}m{3.3cm}|>{\centering\arraybackslash}X|>{\centering\arraybackslash}X|}
        \hline
        \textbf{Nhóm}                       & \textbf{Hạng mục}          & \textbf{Yêu cầu/Mục tiêu}                     \\
        \hline
        \multirow{5}{*}{Horizontal Scaling} & Auto-scaling trigger       & CPU $70\%$ hoặc memory $75\%$                 \\
        \cline{2-3}
                                            & Scale units                & 2 $\rightarrow$ 20 instance trong $< 60$ giây \\
        \cline{2-3}
                                            & Triển khai                 & Containerized (Docker/Kubernetes)             \\
        \cline{2-3}
                                            & Kiến trúc service          & Stateless để scale dễ dàng                    \\
        \cline{2-3}
                                            & Data tier                  & Database read replicas cho workloads đọc      \\
        \hline
        \multirow{4}{*}{Vertical Scaling}   & Instance upgrade           & Không downtime                                \\
        \cline{2-3}
                                            & Database vertical scaling  & Trong maintenance window                      \\
        \cline{2-3}
                                            & Cache layer                & Mở rộng đến $128$GB RAM                       \\
        \cline{2-3}
                                            & Storage                    & Auto-expansion khi đạt $80\%$ dung lượng      \\
        \hline
        \multirow{4}{*}{Load Distribution}  & Geo load balancing         & Đa vùng                                       \\
        \cline{2-3}
                                            & CDN                        & Cho static assets                             \\
        \cline{2-3}
                                            & Database sharding          & Theo trường/khu vực                           \\
        \cline{2-3}
                                            & Async operations           & Queue-based load leveling                     \\
        \hline
        \multirow{5}{*}{Capacity Planning}  & Concurrent users (initial) & $5{,}000$                                     \\
        \cline{2-3}
                                            & Concurrent users (year 2)  & $50{,}000$                                    \\
        \cline{2-3}
                                            & Tổng số người dùng         & $1$ triệu                                     \\
        \cline{2-3}
                                            & Lưu trữ nội dung           & $100$TB                                       \\
        \cline{2-3}
                                            & Đánh giá/tháng             & $10$ triệu bài                                \\
        \hline
    \end{tabularx}
    \caption{Yêu cầu Khả năng Mở Rộng}
    \label{tab:scalability-requirements}
\end{table}
\renewcommand{\arraystretch}{1.0}

\vspace{9em}

\textbf{c. Yêu cầu Bảo mật}%
\renewcommand{\arraystretch}{1.3}
\begin{table}[H]
    \centering
    \small
    \begin{tabularx}{\textwidth}{|>{\centering\arraybackslash}m{3.5cm}|>{\centering\arraybackslash}X|>{\centering\arraybackslash}X|}
        \hline
        \textbf{Nhóm}                                    & \textbf{Hạng mục}     & \textbf{Yêu cầu/Mục tiêu}                       \\
        \hline
        \multirow{5}{*}{Authentication \& Authorization} & SSO                   & OAuth~2.0 / OpenID Connect                      \\
        \cline{2-3}
                                                         & Phiên làm việc        & JWT 15 phút, refresh token an toàn              \\
        \cline{2-3}
                                                         & Multi-factor          & MFA cho admin/instructor                        \\
        \cline{2-3}
                                                         & Phân quyền            & RBAC chi tiết                                   \\
        \cline{2-3}
                                                         & Session timeout       & Tự động sau 30 phút không hoạt động             \\
        \hline
        \multirow{3}{*}{Data Protection}                 & Encryption at rest    & AES-256, database TDE, backup encryption        \\
        \cline{2-3}
                                                         & Encryption in transit & TLS~1.3                                         \\
        \cline{2-3}
                                                         & Key management        & Lưu trong HSM/Key Vault                         \\
        \hline
        \multirow{3}{*}{Compliance \& Privacy}           & Chuẩn tuân thủ        & GDPR, COPPA, FERPA                              \\
        \cline{2-3}
                                                         & Quyền dữ liệu         & Right-to-be-forgotten, retention 3 năm          \\
        \cline{2-3}
                                                         & Audit logs            & Lưu trữ 1 năm                                   \\
        \hline
        \multirow{5}{*}{Application Security}            & Input validation      & Áp dụng cho toàn bộ input                       \\
        \cline{2-3}
                                                         & SQL injection         & Parameterized queries                           \\
        \cline{2-3}
                                                         & XSS / CSRF            & CSP, CSRF tokens cho tác vụ thay đổi trạng thái \\
        \cline{2-3}
                                                         & Rate limiting         & 100 requests/phút/user                          \\
        \cline{2-3}
                                                         & DDoS protection       & WAF đám mây                                     \\
        \hline
    \end{tabularx}
    \caption{Yêu cầu Bảo mật}
    \label{tab:security-requirements}
\end{table}
\renewcommand{\arraystretch}{1.0}

\textbf{d. Yêu cầu Độ tin cậy}%
\renewcommand{\arraystretch}{1.3}
\begin{table}[H]
    \centering
    \small
    \begin{tabularx}{\textwidth}{|>{\centering\arraybackslash}m{3.5cm}|>{\centering\arraybackslash}X|>{\centering\arraybackslash}X|}
        \hline
        \textbf{Nhóm}                       & \textbf{Hạng mục}  & \textbf{Yêu cầu/Mục tiêu}                                    \\
        \hline
        \multirow{4}{*}{Availability}       & Uptime hệ thống    & 99.5\% (downtime $< 44$ giờ/năm)                             \\
        \cline{2-3}
                                            & Core services      & 99.9\% (downtime $< 9$ giờ/năm)                              \\
        \cline{2-3}
                                            & Maintenance window & 2h--4h sáng Chủ nhật                                         \\
        \cline{2-3}
                                            & Redundancy         & Active-active, failover $< 30$ giây                          \\
        \hline
        \multirow{4}{*}{Fault Tolerance}    & Circuit breaker    & Cho external services                                        \\
        \cline{2-3}
                                            & Retry policy       & Exponential backoff                                          \\
        \cline{2-3}
                                            & Degradation        & Graceful khi service lỗi                                     \\
        \cline{2-3}
                                            & Isolation          & Bulkhead, dead letter queue                                  \\
        \hline
        \multirow{3}{*}{Data Integrity}     & Transaction        & ACID cho tác vụ critical                                     \\
        \cline{2-3}
                                            & Consistency        & Eventual cho analytics                                       \\
        \cline{2-3}
                                            & Validation         & Multilayer validation, checksum uploads, conflict resolution \\
        \hline
        \multirow{4}{*}{Backup \& Recovery} & Backup             & Tự động hàng ngày, PITR 7 ngày                               \\
        \cline{2-3}
                                            & Replication        & Đa vùng                                                      \\
        \cline{2-3}
                                            & RTO/RPO            & RTO $< 4$ giờ, RPO $< 1$ giờ                                 \\
        \cline{2-3}
                                            & Testing            & Kiểm thử backup mỗi quý                                      \\
        \hline
    \end{tabularx}
    \caption{Yêu cầu Độ tin cậy}
    \label{tab:reliability-requirements}
\end{table}
\renewcommand{\arraystretch}{1.0}

\vspace{13em}

\textbf{e. Yêu cầu Khả năng Sử dụng}%
\renewcommand{\arraystretch}{1.3}
\begin{table}[H]
    \centering
    \small
    \begin{tabularx}{\textwidth}{|>{\centering\arraybackslash}m{3.5cm}|>{\centering\arraybackslash}X|>{\centering\arraybackslash}X|}
        \hline
        \textbf{Nhóm}                    & \textbf{Hạng mục} & \textbf{Yêu cầu/Mục tiêu}                           \\
        \hline
        \multirow{4}{*}{User Interface}  & Responsive design & Mobile/Tablet/Desktop                               \\
        \cline{2-3}
                                         & Accessibility     & WCAG~2.1 mức AA                                     \\
        \cline{2-3}
                                         & Tuỳ biến          & Đa ngôn ngữ (VN/EN), dark mode, font tuỳ chỉnh      \\
        \cline{2-3}
                                         & Điều hướng        & Hỗ trợ keyboard navigation                          \\
        \hline
        \multirow{3}{*}{User Experience} & Hiệu năng UI      & First meaningful paint $< 1.5$ giây, TTI $< 3$ giây \\
        \cline{2-3}
                                         & Trải nghiệm       & Animation 60~FPS, PWA, offline content              \\
        \cline{2-3}
                                         & Lưu trạng thái    & Auto-save mỗi 30 giây                               \\
        \hline
        \multirow{3}{*}{Learning Curve}  & Onboarding        & Tutorial, help context, tooltip                     \\
        \cline{2-3}
                                         & Hỗ trợ            & Video hướng dẫn, in-app messaging                   \\
        \cline{2-3}
                                         & Tài liệu          & Knowledge base, FAQ (nếu có)                        \\
        \hline
    \end{tabularx}
    \caption{Yêu cầu Khả năng Sử dụng}
    \label{tab:usability-requirements}
\end{table}
\renewcommand{\arraystretch}{1.0}

\textbf{f. Yêu cầu Tương thích}%
\renewcommand{\arraystretch}{1.3}
\begin{table}[H]
    \centering
    \small
    \begin{tabularx}{\textwidth}{|>{\centering\arraybackslash}m{3.5cm}|>{\centering\arraybackslash}X|>{\centering\arraybackslash}X|}
        \hline
        \textbf{Nhóm}                              & \textbf{Hạng mục}       & \textbf{Yêu cầu/Mục tiêu}                                 \\
        \hline
        \multirow{2}{*}{Browser Support}           & Trình duyệt             & Chrome~90+, Firefox~88+, Safari~14+, Edge~90+             \\
        \cline{2-3}
                                                   & Progressive enhancement & Hỗ trợ trình duyệt cũ ở mức tối thiểu                     \\
        \hline
        \multirow{2}{*}{Device Support}            & Di động                 & iOS~13+, Android~8+                                       \\
        \cline{2-3}
                                                   & Desktop                 & Windows~10+, macOS~10.15+, Linux Ubuntu~20.04+/Fedora~34+ \\
        \hline
        \multirow{2}{*}{Integration Compatibility} & Chuẩn tích hợp          & LTI~1.3, SCORM~2004, xAPI, QTI~2.2, Common Cartridge~1.3  \\
        \cline{2-3}
                                                   & Interop                 & Đảm bảo import/export nội dung chuẩn hoá                  \\
        \hline
        \multirow{2}{*}{API Compatibility}         & Giao thức               & REST (OpenAPI~3.0), GraphQL, WebSocket, Webhook           \\
        \cline{2-3}
                                                   & Backward compatibility  & Hỗ trợ 2 phiên bản chính                                  \\
        \hline
    \end{tabularx}
    \caption{Yêu cầu Tương thích}
    \label{tab:compatibility-requirements}
\end{table}
\renewcommand{\arraystretch}{1.0}

\textbf{g. Yêu cầu Giám sát}%
\renewcommand{\arraystretch}{1.3}
\begin{table}[H]
    \centering
    \small
    \begin{tabularx}{\textwidth}{|>{\centering\arraybackslash}m{3.8cm}|>{\centering\arraybackslash}X|>{\centering\arraybackslash}X|}
        \hline
        \textbf{Nhóm}                       & \textbf{Hạng mục}   & \textbf{Yêu cầu/Mục tiêu}                              \\
        \hline
        \multirow{3}{*}{Metrics Collection} & Application metrics & Prometheus/StatsD, dashboard KPI                       \\
        \cline{2-3}
                                            & AI metrics          & Theo dõi hiệu suất model AI                            \\
        \cline{2-3}
                                            & Cost tracking       & Resource usage, budget alerts                          \\
        \hline
        \multirow{2}{*}{Logging}            & Nền tảng            & ELK/EFK, cấu trúc JSON, correlation ID                 \\
        \cline{2-3}
                                            & Retention           & 30 ngày (hot), 1 năm (cold)                            \\
        \hline
        Tracing                             & Distributed tracing & OpenTelemetry, dependency mapping, sampling 10\%       \\
        \hline
        \multirow{3}{*}{Alerting}           & Real-time alert     & Escalation email $\rightarrow$ SMS $\rightarrow$ phone \\
        \cline{2-3}
                                            & Maintenance window  & Suppress khi bảo trì, runbook tích hợp                 \\
        \cline{2-3}
                                            & SLO monitoring      & Error budget burn rate                                 \\
        \hline
    \end{tabularx}
    \caption{Yêu cầu Giám sát}
    \label{tab:monitoring-requirements}
\end{table}
\renewcommand{\arraystretch}{1.0}

\vspace{8em}

\textbf{h. Yêu cầu Tuân thủ}%
\renewcommand{\arraystretch}{1.3}
\begin{table}[H]
    \centering
    \small
    \begin{tabularx}{\textwidth}{|>{\centering\arraybackslash}m{3.8cm}|>{\centering\arraybackslash}X|>{\centering\arraybackslash}X|}
        \hline
        \textbf{Nhóm}                          & \textbf{Hạng mục}    & \textbf{Yêu cầu/Mục tiêu}                                                                \\
        \hline
        \multirow{2}{*}{Educational Standards} & Curriculum alignment & VN National Curriculum, Cambridge, IB                                                    \\
        \cline{2-3}
                                               & Competency-based     & Mapping learning objectives, standards grading                                           \\
        \hline
        Data Governance                        & Quản trị dữ liệu     & Phân loại, lineage, quality monitoring, MDM, privacy impact assessment                   \\
        \hline
        Audit \& Reporting                     & Báo cáo              & Audit trail, activity/security incident report, compliance dashboard, automated checking \\
        \hline
    \end{tabularx}
    \caption{Yêu cầu Tuân thủ}
    \label{tab:compliance-requirements}
\end{table}
\renewcommand{\arraystretch}{1.0}

\textbf{i. Yêu cầu Phục hồi Thảm hoạ}%
\renewcommand{\arraystretch}{1.3}
\begin{table}[H]
    \centering
    \small
    \begin{tabularx}{\textwidth}{|>{\centering\arraybackslash}m{3.8cm}|>{\centering\arraybackslash}X|>{\centering\arraybackslash}X|}
        \hline
        \textbf{Nhóm}                        & \textbf{Hạng mục} & \textbf{Yêu cầu/Mục tiêu}                                            \\
        \hline
        \multirow{3}{*}{Business Continuity} & DR plan           & Tài liệu đầy đủ, kiểm thử định kỳ                                    \\
        \cline{2-3}
                                             & Site dự phòng     & Alternative processing sites, communication plan, vendor contingency \\
        \cline{2-3}
                                             & DR drill          & 2 lần/năm                                                            \\
        \hline
        \multirow{3}{*}{Data Recovery}       & Verification      & Backup verification tự động, nhiều địa điểm (on-prem + cloud)        \\
        \cline{2-3}
                                             & Bảo mật           & Backup encryption, integrity check, test restore hàng tháng          \\
        \cline{2-3}
                                             & Granular recovery & Theo bảng, người dùng                                                \\
        \hline
    \end{tabularx}
    \caption{Yêu cầu Phục hồi Thảm hoạ}
    \label{tab:disaster-requirements}
\end{table}
\renewcommand{\arraystretch}{1.0}

\textbf{j. Yêu cầu Bảo trì \& Hỗ trợ}%
\renewcommand{\arraystretch}{1.3}
\begin{table}[H]
    \centering
    \small
    \begin{tabularx}{\textwidth}{|>{\centering\arraybackslash}m{3.8cm}|>{\centering\arraybackslash}X|>{\centering\arraybackslash}X|}
        \hline
        \textbf{Nhóm}                       & \textbf{Hạng mục} & \textbf{Yêu cầu/Mục tiêu}                              \\
        \hline
        \multirow{3}{*}{System Maintenance} & Triển khai        & Blue-green deployment, automation database maintenance \\
        \cline{2-3}
                                            & Vận hành          & Log rotation/cleanup, certificate renewal automation   \\
        \cline{2-3}
                                            & Cập nhật          & Dependency update hàng tháng                           \\
        \hline
        \multirow{3}{*}{Technical Support}  & Monitoring        & Giám sát 24/7, support L1/L2/L3                        \\
        \cline{2-3}
                                            & Response time     & Critical incidents $< 15$ phút                         \\
        \cline{2-3}
                                            & Knowledge base    & $> 500$ bài viết, hỗ trợ chẩn đoán từ xa               \\
        \hline
    \end{tabularx}
    \caption{Yêu cầu Bảo trì \& Hỗ trợ}
    \label{tab:maintenance-requirements}
\end{table}
\renewcommand{\arraystretch}{1.0}

Các thuộc tính chất lượng trên khép lại phần yêu cầu phi chức năng, đóng vai trò làm cầu nối giữa yêu cầu và thiết kế kiến trúc trong các chương sau.
Từ các tiêu chí định lượng này, nhóm thiết kế có thể xác định kiểu kiến trúc, công cụ giám sát và chiến lược tối ưu phù hợp để đảm bảo hệ thống vận hành ổn định, mở rộng và đáng tin cậy
\newpage
% \subsection{Ràng Buộc và Giả Định}

% \indentpar \indentpar Phần này xác định các ràng buộc (constraints) ảnh hưởng đến phạm vi và quyết định thiết kế của hệ thống ITS, cùng với các giả định (assumptions) được đặt ra trong giai đoạn phân tích và thiết kế kiến trúc.
% Các yếu tố này giúp định hình phạm vi thực thi của dự án, làm cơ sở để đánh giá tính khả thi và xác định giới hạn mở rộng của hệ thống.

% \subsubsection{Technical Constraints (Ràng buộc kỹ thuật)}

% \begin{table}[ht]
% \centering
% \small
% \renewcommand{\tabularxcolumn}[1]{m{#1}}
% \renewcommand{\arraystretch}{1.5}
% \begin{tabularx}{\textwidth}{|>{\centering\arraybackslash}m{3.5cm}|>{\noindent\justifying\arraybackslash}X|}
% \hline
% \textbf{Hạng mục} & \textbf{Mô tả} \\
% \hline
% Quy mô đội ngũ phát triển & 4--5 thành viên (bao gồm Backend Developer, DevOps Engineer, AI Engineer, QA). Giới hạn nguồn lực khiến việc mở rộng chức năng cần được ưu tiên theo mức độ quan trọng (MoSCoW). \\
% \hline
% Công nghệ được phép sử dụng & Backend: Golang (GIN), Java (Spring Boot)\newline Frontend: Next.js 14 + TypeScript\newline Database: PostgreSQL, MongoDB, Redis\newline AI/ML: Python (FastAPI/Whisper/ONNX), Golang ML binding\newline Các công nghệ khác ngoài danh sách này không được đưa vào giai đoạn MVP. \\
% \hline
% Giới hạn hạ tầng triển khai & Sử dụng cluster Kubernetes 3 node (vCPU: 8, RAM: 32GB/node). Không được vượt quá quota tài nguyên đã cấp. \\
% \hline
% Ngân sách hạ tầng & Tối đa $\$300$/tháng, bao gồm chi phí cloud (GKE hoặc self-hosted k8s), MinIO object storage, và Redis caching. \\
% \hline
% Quy định bảo mật dữ liệu & Phải tuân thủ tiêu chuẩn FERPA và tương thích với yêu cầu GDPR khi xử lý dữ liệu người học (profile, điểm số). \\
% \hline
% Giới hạn hiệu năng mô hình AI & Các mô hình AI (Adaptive Engine, Feedback Generator) phải chạy được trên CPU, không yêu cầu GPU để giảm chi phí triển khai. \\
% \hline
% Giới hạn phần mềm phụ thuộc & Không sử dụng framework hoặc SDK thương mại có phí bản quyền. Ưu tiên mã nguồn mở. \\
% \hline
% Hệ thống tích hợp bên ngoài & ITS chỉ tích hợp với Auth Provider (JWT/OAuth2) và LMS API chuẩn LTI 1.3 trong giai đoạn đầu. \\
% \hline
% \end{tabularx}
% \renewcommand{\arraystretch}{1.0}
% \caption{Technical Constraints}
% \label{tab:technical_constraints}
% \end{table}
% \FloatBarrier

% \subsubsection{Business Constraints (Ràng buộc kinh doanh)}

% \begin{table}[ht]
% \centering
% \small
% \renewcommand{\tabularxcolumn}[1]{m{#1}}
% \renewcommand{\arraystretch}{1.5}
% \begin{tabularx}{\textwidth}{|>{\centering\arraybackslash}m{3.5cm}|>{\noindent\justifying\arraybackslash}X|}
% \hline
% \textbf{Hạng mục} & \textbf{Mô tả} \\
% \hline
% Thời gian thực hiện (Timeline) & Tổng thời gian dự án: 12 tuần (3 tháng), trong đó:\newline -- Tuần 1--3: Phân tích và thiết kế kiến trúc\newline -- Tuần 4--8: Phát triển và tích hợp các microservice chính (Auth, Learner, Adaptive Engine)\newline -- Tuần 9--12: Kiểm thử, tối ưu và triển khai MVP. \\
% \hline
% Ngân sách tổng thể & $\$1.000$ cho toàn bộ giai đoạn MVP (bao gồm hạ tầng, CI/CD, tài nguyên lưu trữ, domain và chứng chỉ bảo mật SSL). \\
% \hline
% Phạm vi phiên bản MVP & Chỉ bao gồm các chức năng cốt lõi: Học tập thích ứng (Adaptive Learning), Chấm điểm và Phản hồi tức thì, Báo cáo tiến độ. Các tính năng phụ như Gamification hoặc Chat 1-1 sẽ được lên kế hoạch cho phiên bản mở rộng. \\
% \hline
% Quy định và tuân thủ (Compliance) & Tuân thủ quy định bảo mật FERPA (Mỹ) và GDPR (Châu Âu).\newline Các thành phần xử lý dữ liệu cá nhân phải có cơ chế audit log và role-based access control. \\
% \hline
% Môi trường vận hành & ITS phải hoạt động ổn định trên môi trường cloud (GCP/GKE hoặc AWS EKS) và có khả năng triển khai on-premise cho cơ sở giáo dục nhỏ. \\
% \hline
% Chiến lược mở rộng & Mọi thành phần (service, database) phải được thiết kế theo nguyên tắc cloud-native, đảm bảo khả năng mở rộng ngang (scale-out) mà không cần viết lại logic nghiệp vụ. \\
% \hline
% Phụ thuộc nhân sự và lịch học kỳ & Mốc triển khai phải hoàn thành trước khi bắt đầu học kỳ mới, để có thể demo trong báo cáo môn học Kiến trúc Phần mềm. \\
% \hline
% \end{tabularx}
% \renewcommand{\arraystretch}{1.0}
% \caption{Business Constraints}
% \label{tab:business_constraints}
% \end{table}
% \FloatBarrier

% \subsubsection{Assumptions (Giả định)}

% \begin{table}[ht]
% \centering
% \small
% \renewcommand{\tabularxcolumn}[1]{m{#1}}
% \renewcommand{\arraystretch}{1.5}
% \begin{tabularx}{\textwidth}{|>{\centering\arraybackslash}m{4.5cm}|>{\noindent\justifying\arraybackslash}X|}
% \hline
% \textbf{Giả định} & \textbf{Giải thích} \\
% \hline
% Người dùng có kết nối Internet ổn định $\geq 10$ Mbps & Đảm bảo khả năng stream nội dung học tập (video, quiz, bài tập coding) mà không gián đoạn. \\
% \hline
% Trình duyệt hiện đại (Chrome 90+, Firefox 88+, Safari 14+) & Hệ thống frontend sử dụng WebSocket và API hiện đại, không đảm bảo tương thích với trình duyệt cũ. \\
% \hline
% Người dùng có tài khoản xác thực hợp lệ & Tất cả người học và giảng viên phải đăng ký tài khoản trước khi truy cập hệ thống. \\
% \hline
% Server vận hành 24/7 & Đảm bảo tính sẵn sàng cao (SLA $\geq 99.5\%$) và hỗ trợ học tập không giới hạn thời gian. \\
% \hline
% Cơ sở dữ liệu được sao lưu hàng ngày & Phục hồi dữ liệu nhanh trong trường hợp lỗi hoặc mất kết nối. \\
% \hline
% Các service hoạt động trong cùng một VPC nội bộ & Đảm bảo latency thấp và không bị giới hạn bởi firewall của nhà cung cấp cloud. \\
% \hline
% Người dùng chấp nhận chính sách quyền riêng tư (Privacy Policy) & Việc lưu trữ và xử lý dữ liệu học tập tuân theo quy định về bảo mật. \\
% \hline
% Không có downtime trong giờ học cao điểm (08:00--22:00) & Mọi đợt cập nhật hoặc bảo trì hệ thống được thực hiện sau 22:00. \\
% \hline
% Hệ thống phát triển và kiểm thử trong môi trường staging tương tự production & Đảm bảo tính nhất quán giữa môi trường phát triển và triển khai. \\
% \hline
% \end{tabularx}
% \renewcommand{\arraystretch}{1.0}
% \caption{Assumptions}
% \label{tab:assumptions}
% \end{table}
% \FloatBarrier

% \noindent\textbf{Tóm tắt}

% Các ràng buộc và giả định trên đảm bảo thiết kế kiến trúc ITS:
% \begin{itemize}
%     \item Hiện thực trong giới hạn nguồn lực và ngân sách của nhóm phát triển.
%     \item Tuân thủ tiêu chuẩn bảo mật và quyền riêng tư cho dữ liệu học viên.
%     \item Duy trì tính khả thi của MVP, với nền tảng vững chắc cho các giai đoạn mở rộng sau.
% \end{itemize}


\end{document}