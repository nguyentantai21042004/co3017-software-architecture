\subsection{Architecture Decision Records}

\indentpar \indentpar Sau khi lựa chọn kiến trúc tổng thể là \textbf{Hybrid Microservices + Event-Driven}, bước tiếp theo là ghi lại chi tiết các quyết định kỹ thuật quan trọng giúp hiện thực hóa kiến trúc đó. 
Phần này sử dụng mô hình \textbf{Architecture Decision Record (ADR)} để mô tả bối cảnh, quyết định, lý do và hệ quả theo cách có thể truy vết (traceable) tới các yêu cầu nghiệp vụ và các đặc tính kiến trúc (ACs).

\subsubsection{Tổng quan các quyết định}

\indentpar \indentpar Các quyết định được sắp xếp theo thứ tự từ nền tảng (ADR-1 \textrightarrow{} ADR-3) đến lớp vận hành (ADR-5 \textrightarrow{} ADR-7).

\renewcommand{\arraystretch}{1.3}
\newcolumntype{C}[1]{>{\centering\arraybackslash}p{#1}}
\newcolumntype{Y}{>{\centering\arraybackslash}X}
\begin{table}[H]
\centering
\small
\begin{tabularx}{\textwidth}{|C{1.8cm}|Y|C{3cm}|}
\hline
\textbf{ADR}\rule{0pt}{1.6em} & \textbf{Mục tiêu chính} & \textbf{Liên kết ACs} \\
\hline
ADR-1 & Xác định chiến lược ngôn ngữ đa nền tảng (Polyglot) & AC3, AC7 \\
\hline
ADR-2 & Lựa chọn cơ sở dữ liệu quan hệ chính (PostgreSQL) & AC6, AC7 \\
\hline
ADR-3 & Thiết lập mô hình kiến trúc nội bộ (Clean/Hexagonal) & AC1, AC4 \\
\hline
ADR-4 & Triển khai Repository Pattern để tách tầng dữ liệu & AC1, AC4 \\
\hline
ADR-5 & Định nghĩa chiến lược kiểm thử (Testing Pyramid) & AC4, AC7 \\
\hline
ADR-6 & Thiết kế kiến trúc bảo mật tập trung (AuthN/AuthZ) & AC6 \\
\hline
ADR-7 & Chiến lược bảo vệ dữ liệu cá nhân (Data Privacy \& Compliance) & AC6 \\
\hline
\end{tabularx}
\caption{Tổng quan các Architecture Decision Records}
\label{tab:adr-overview}
\end{table}
\renewcommand{\arraystretch}{1.0}

\subsubsection{ADR-1: Chiến lược lập trình đa ngôn ngữ}

\newcolumntype{M}[1]{>{\centering\arraybackslash}m{#1}}
\newcolumntype{Y}{>{\raggedright\arraybackslash\vspace{0.3em}}X<{\vspace{0.3em}}}
\renewcommand{\arraystretch}{1.3}
\begin{table}[H]
\centering
\small
\begin{tabularx}{\textwidth}{|M{2.4cm}|Y|}
\hline
\textbf{Hạng mục}\rule{0pt}{1.6em} & \multicolumn{1}{c|}{\textbf{Nội dung}} \\
\hline
\textbf{Bối cảnh} & \begin{itemize}[leftmargin=0.4cm]
    \item Hệ thống gồm nhiều nhóm dịch vụ với yêu cầu phi chức năng khác nhau:
    \begin{itemize}[nosep]
        \item Management Services (User, Content) cần logic nghiệp vụ phức tạp, hệ sinh thái phong phú và khả năng bảo trì cao.
        \item Computation Services (Scoring, Feedback) yêu cầu hiệu năng cao, độ trễ thấp ($\leq 500\,\text{ms}$) và khả năng xử lý đồng thời lớn.
        \item AI/ML Services đòi hỏi xử lý CPU chuyên sâu và chu kỳ lặp nhanh.
    \end{itemize}
    \item Đội ngũ phát triển có kinh nghiệm với cả Java và Golang.
\end{itemize} \\
\hline
\textbf{Quyết định} & \begin{itemize}[leftmargin=0.4cm]
    \item Áp dụng chiến lược Polyglot Programming cho từng nhóm dịch vụ.
    \item Java 17+ (Spring Boot 3.x) cho các dịch vụ thiên về nghiệp vụ và bảo trì: `User Management Service`, `Content Service`.
    \item Golang 1.21+ (Gin/Echo) cho các dịch vụ yêu cầu hiệu năng, concurrency và độ trễ thấp: `Scoring/Feedback Service`, `Adaptive Engine`, `Learner Model Service`, `API Gateway`.
\end{itemize} \\
\hline
\textbf{Lý luận} & \begin{itemize}[leftmargin=0.4cm]
    \item Hỗ trợ AC3 (Performance) \& AC2 (Scalability): Golang hiệu năng cao, khởi động nhanh, mô hình goroutine mạnh.
    \item Hỗ trợ AC7 (Maintainability): Java/Spring Boot có hệ sinh thái trưởng thành cho nghiệp vụ, bảo mật, ORM.
    \item ``Dùng công cụ tốt nhất cho từng bài toán'', tránh tư duy ``một búa cho mọi loại đinh''.
    \item Giảm rủi ro khi chỉ sử dụng một ngôn ngữ duy nhất.
\end{itemize} \\
\hline
\textbf{Hậu quả} & \textbf{Tích cực:}
\begin{itemize}[leftmargin=0.4cm]
    \item Tối ưu hiệu năng cho các dịch vụ thời gian thực.
    \item Nâng cao khả năng bảo trì cho các dịch vụ nghiệp vụ.
    \item Tận dụng thế mạnh của từng hệ sinh thái.
\end{itemize}
\textbf{Tiêu cực:}
\begin{itemize}[leftmargin=0.4cm]
    \item Cần chuyên môn ở hai ngôn ngữ và hai bộ công cụ (Maven/Gradle vs Go modules).
    \item Chiến lược kiểm thử \& CI/CD phức tạp hơn.
\end{itemize} \\
\hline
\textbf{Rủi ro} & \begin{itemize}[leftmargin=0.4cm]
    \item Rủi ro: không nhất quán do vận hành hai hệ sinh thái song song.
    \item Giảm thiểu: đào tạo chéo, chuẩn hóa cấu trúc dự án, chia sẻ template CI/CD, thống nhất chuẩn logging/monitoring.
\end{itemize} \\
\hline
\textbf{Các lựa chọn} & \begin{itemize}[leftmargin=0.4cm]
    \item Tất cả bằng Java: đơn giản về chuyên môn nhưng khó đạt AC3 (Performance).
    \item Tất cả bằng Golang: tooling nhất quán nhưng thiếu hệ sinh thái nghiệp vụ, không đạt AC7 (Maintainability).
\end{itemize} \\
\hline
\textbf{Liên quan} &  Ảnh hưởng tới ADR-3 (Clean Architecture cho Java/Go) và ADR-5 (Testing Pyramid cho hai stack). \\
\hline
\end{tabularx}
\caption{Tóm tắt ADR-1: Chiến lược lập trình đa ngôn ngữ}
\label{tab:adr1-summary}
\end{table}
\renewcommand{\arraystretch}{1.0}

\subsubsection{ADR-2: PostgreSQL là cơ sở dữ liệu quan hệ chính}

\renewcommand{\arraystretch}{1.3}
\begin{table}[H]
\centering
\small
\begin{tabularx}{\textwidth}{|M{2.4cm}|Y|}
\hline
\textbf{Hạng mục}\rule{0pt}{1.6em} & \multicolumn{1}{c|}{\textbf{Nội dung}} \\
\hline
\textbf{Bối cảnh} & `User Management` và `Content Service` yêu cầu cơ sở dữ liệu quan hệ bảo đảm ACID, hỗ trợ truy vấn phức tạp, RBAC, metadata dạng JSON và có cơ chế replication/backup trưởng thành. \\
\hline
\textbf{Quyết định} & \begin{itemize}[leftmargin=0.4cm]
    \item Chọn PostgreSQL 15+ làm cơ sở dữ liệu quan hệ chính.
    \item Cấu hình: primary-standby replication, connection pooling (PgBouncer), WAL archiving để hỗ trợ point-in-time recovery.
\end{itemize} \\
\hline
\textbf{Lý luận} & \begin{itemize}[leftmargin=0.4cm]
    \item Hỗ trợ AC6 (Security) với row-level security và cơ chế bảo mật nâng cao.
    \item Hỗ trợ AC7 (Maintainability) nhờ tuân thủ ACID và quản lý lược đồ mạnh.
    \item Hỗ trợ metadata linh hoạt qua JSON/JSONB.
    \item Giảm rủi ro mất hoặc không nhất quán dữ liệu so với NoSQL.
\end{itemize} \\
\hline
\textbf{Hậu quả} & \textbf{Tích cực:}
\begin{itemize}[leftmargin=0.4cm]
    \item Đảm bảo toàn vẹn dữ liệu, truy vấn phong phú, không bị vendor lock-in.
\end{itemize}
\textbf{Tiêu cực:}
\begin{itemize}[leftmargin=0.4cm]
    \item Hạn chế mở rộng theo chiều dọc, cần tối ưu index.
    \item Di trú schema tốn công sức.
\end{itemize} \\
\hline
\textbf{Rủi ro} & \begin{itemize}[leftmargin=0.4cm]
    \item Rủi ro: hiệu năng suy giảm khi tải nặng hoặc truy vấn chưa tối ưu.
    \item Giảm thiểu: dùng read replica, đánh index phù hợp, connection pooling và giám sát slow query.
\end{itemize} \\
\hline
\textbf{Các lựa chọn} & \begin{itemize}[leftmargin=0.4cm]
    \item MySQL: phổ biến nhưng hỗ trợ JSON và truy vấn phức tạp kém hơn.
    \item NoSQL (MongoDB): mở rộng ngang tốt nhưng thiếu ACID và truy vấn quan hệ.
\end{itemize} \\
\hline
\textbf{Liên quan} & \begin{itemize}[leftmargin=0.4cm]
    \item Tác động tới ADR-4 (Repository Pattern với Postgres) và ADR-7 (Data Privacy tận dụng `pgcrypto`).
\end{itemize} \\
\hline
\end{tabularx}
\caption{Tóm tắt ADR-2: PostgreSQL là cơ sở dữ liệu quan hệ chính}
\label{tab:adr2-summary}
\end{table}
\renewcommand{\arraystretch}{1.0}

\subsubsection{ADR-3: Clean/Hexagonal Architecture for All Services}

\renewcommand{\arraystretch}{1.3}
\begin{table}[H]
\centering
\small
\begin{tabularx}{\textwidth}{|M{2.4cm}|Y|}
\hline
\textbf{Hạng mục}\rule{0pt}{1.6em} & \multicolumn{1}{c|}{\textbf{Nội dung}} \\
\hline
\textbf{Bối cảnh} & \begin{itemize}[leftmargin=0.4cm]
    \item Các AC ưu tiên: AC4 (Testability), AC1 (Modularity), AC7 (Maintainability).
    \item Kiến trúc tầng truyền thống thường kết dính logic nghiệp vụ với hạ tầng, khiến việc kiểm thử độc lập và thay đổi trở nên khó khăn.
\end{itemize} \\
\hline
\textbf{Quyết định} & \begin{itemize}[leftmargin=0.4cm]
    \item Áp dụng Clean/Hexagonal Architecture cho toàn bộ microservice (Java và Go).
    \item Tuân thủ Dependency Rule: phụ thuộc hướng từ Infrastructure \textrightarrow{} Adapters \textrightarrow{} Application \textrightarrow{} Domain.
\end{itemize} \\
\hline
\textbf{Lý luận} & \begin{itemize}[leftmargin=0.4cm]
    \item Đảm bảo AC4: logic nghiệp vụ và use case có thể unit test độc lập.
    \item Đảm bảo AC1/AC7: tách rõ mối quan tâm, cho phép thay đổi framework/DB mà không ảnh hưởng domain.
    \item Thực thi Dependency Inversion Principle (DIP).
    \item Giảm rủi ro vendor lock-in và ``big ball of mud'' khi hệ thống mở rộng.
\end{itemize} \\
\hline
\textbf{Hậu quả} & \textbf{Tích cực:}
\begin{itemize}[leftmargin=0.4cm]
    \item Domain thuần túy, unit test dễ dàng, có thể thay đổi DB mà không sửa logic.
\end{itemize}
\textbf{Tiêu cực:}
\begin{itemize}[leftmargin=0.4cm]
    \item Tăng boilerplate, đường cong học tập cao hơn, số lượng tệp lớn.
\end{itemize} \\
\hline
\textbf{Rủi ro} & \begin{itemize}[leftmargin=0.4cm]
    \item Rủi ro: đội ngũ (đặc biệt junior) áp dụng sai, vi phạm quy tắc phụ thuộc.
    \item Giảm thiểu: cung cấp template chuẩn cho Java/Go, tài liệu chi tiết, code review nghiêm.
\end{itemize} \\
\hline
\textbf{Các lựa chọn} & \begin{itemize}[leftmargin=0.4cm]
    \item Layered Architecture: quen thuộc nhưng kết dính chặt với framework/DB, không đáp ứng AC4.
\end{itemize} \\
\hline
\textbf{Liên quan} & \begin{itemize}[leftmargin=0.4cm]
    \item Phụ thuộc lựa chọn kiến trúc tổng thể; ảnh hưởng trực tiếp ADR-1, ADR-4, ADR-5.
\end{itemize} \\
\hline
\end{tabularx}
\caption{Tóm tắt ADR-3: Clean/Hexagonal Architecture}
\label{tab:adr3-summary}
\end{table}
\renewcommand{\arraystretch}{1.0}

\subsubsection{ADR-4: Repository Pattern with Interface Abstraction}

\renewcommand{\arraystretch}{1.3}
\begin{table}[H]
\centering
\small
\begin{tabularx}{\textwidth}{|M{2.4cm}|Y|}
\hline
\textbf{Hạng mục}\rule{0pt}{1.6em} & \multicolumn{1}{c|}{\textbf{Nội dung}} \\
\hline
\textbf{Bối cảnh} & \begin{itemize}[leftmargin=0.4cm]
    \item ADR-3 yêu cầu tầng application không phụ thuộc trực tiếp vào hạ tầng/ORM.
    \item Mục tiêu: đảo ngược phụ thuộc để đạt AC1 (Modularity) và AC4 (Testability).
\end{itemize} \\
\hline
\textbf{Quyết định} & \begin{enumerate}[leftmargin=0.4cm]
    \item Định nghĩa repository interfaces (ports) trong tầng application.
    \item Cài đặt interfaces trong tầng infrastructure (adapters) bằng ORM/SQL.
    \item Sử dụng Dependency Injection để cung cấp implementation (ví dụ `PostgresUserRepository`) cho các use case.
\end{enumerate} \\
\hline
\textbf{Lý luận} & \begin{itemize}[leftmargin=0.4cm]
    \item Đảm bảo AC4: cho phép unit test use case với mock repository.
    \item Đảm bảo AC1: hoán đổi công nghệ DB mà không sửa logic ứng dụng.
    \item Thực thi DIP theo yêu cầu ADR-3.
    \item Giảm rủi ro logic nghiệp vụ phụ thuộc ORM cụ thể.
\end{itemize} \\
\hline
\textbf{Hậu quả} & \textbf{Tích cực:}
\begin{itemize}[leftmargin=0.4cm]
    \item Kiểm thử không cần DB, dễ hoán đổi triển khai, ranh giới dữ liệu rõ ràng.
\end{itemize}
\textbf{Tiêu cực:}
\begin{itemize}[leftmargin=0.4cm]
    \item Nhiều interface cần bảo trì, phải ánh xạ domain entity với DB entity.
\end{itemize} \\
\hline
\textbf{Rủi ro} & \begin{itemize}[leftmargin=0.4cm]
    \item Rủi ro: việc mapping giữa domain object và DB object tốn thời gian.
    \item Giảm thiểu: dùng thư viện mapping (ví dụ MapStruct) hoặc code generation.
\end{itemize} \\
\hline
\textbf{Các lựa chọn} & \begin{itemize}[leftmargin=0.4cm]
    \item Dùng ORM trực tiếp trong use case: giảm boilerplate nhưng vi phạm ADR-3, không đạt AC4.
\end{itemize} \\
\hline
\textbf{Liên quan} & \begin{itemize}[leftmargin=0.4cm]
    \item Phụ thuộc ADR-3; ảnh hưởng ADR-5 (mock repository cho unit test) và ADR-2 (triển khai Postgres cụ thể).
\end{itemize} \\
\hline
\end{tabularx}
\caption{Tóm tắt ADR-4: Repository Pattern with Interface Abstraction}
\label{tab:adr4-summary}
\end{table}
\renewcommand{\arraystretch}{1.0}

\subsubsection{ADR-5: Testing Strategy (Testing Pyramid)}

\renewcommand{\arraystretch}{1.3}
\begin{table}[H]
\centering
\small
\begin{tabularx}{\textwidth}{|M{2.4cm}|Y|}
\hline
\textbf{Hạng mục}\rule{0pt}{1.6em} & \multicolumn{1}{c|}{\textbf{Nội dung}} \\
\hline
\textbf{Bối cảnh} & \begin{itemize}[leftmargin=0.4cm]
    \item Cần chiến lược kiểm thử rõ ràng để đạt AC4 (Testability) trong môi trường Microservices + Polyglot.
    \item Logic AI yêu cầu độ chính xác và độ tin cậy cao.
\end{itemize} \\
\hline
\textbf{Quyết định} & \begin{enumerate}[leftmargin=0.4cm]
    \item Unit Tests ($\geq 80\%$ coverage): kiểm thử logic domain/application độc lập; dùng JUnit 5/Mockito, `go test`/`testify`; mock toàn bộ I/O.
    \item Integration Tests: kiểm thử tương tác với DB/message broker bằng `@SpringBootTest` và Testcontainers.
    \item End-to-End Tests: xác thực luồng nghiệp vụ qua API Gateway bằng Cypress/Playwright/Postman/K6, chỉ tập trung happy path quan trọng.
\end{enumerate} \\
\hline
\textbf{Lý luận} & \begin{itemize}[leftmargin=0.4cm]
    \item Đảm bảo AC4 dựa trên nền tảng kỹ thuật ADR-3 và ADR-4.
    \item Đảm bảo độ tin cậy cao cho các thuật toán AI/Scoring.
    \item Phát hiện lỗi sớm ở tầng Unit Test, giảm chi phí so với phát hiện ở E2E.
\end{itemize} \\
\hline
\textbf{Hậu quả} & \textbf{Tích cực:}
\begin{itemize}[leftmargin=0.4cm]
    \item Chất lượng code đáng tin cậy, phát hiện lỗi sớm, logic AI được kiểm thử kỹ.
\end{itemize}
\textbf{Tiêu cực:}
\begin{itemize}[leftmargin=0.4cm]
    \item Testcontainers tăng thời gian CI/CD; E2E dễ flaky; đội ngũ cần học Testcontainers.
\end{itemize} \\
\hline
\textbf{Rủi ro} & \begin{itemize}[leftmargin=0.4cm]
    \item Rủi ro: CI chạy chậm do integration test; giảm thiểu bằng cách tách pipeline (Unit mỗi commit, Integration/E2E khi PR/merge main).
    \item Rủi ro: E2E không ổn định; giảm thiểu bằng cách giới hạn E2E cho happy path quan trọng.
\end{itemize} \\
\hline
\textbf{Các lựa chọn} & \begin{itemize}[leftmargin=0.4cm]
    \item Chỉ Unit Test: nhanh nhưng bỏ lỡ lỗi tích hợp.
    \item Chỉ E2E (Ice Cream Cone): chậm, đắt đỏ, khó xác định nguyên nhân lỗi.
    \item Contract Tests: hữu ích nhưng quá phức tạp cho giai đoạn MVP hiện tại.
\end{itemize} \\
\hline
\textbf{Liên quan} & \begin{itemize}[leftmargin=0.4cm]
    \item Phụ thuộc ADR-1, ADR-3, ADR-4; định hình cách thiết kế unit test cho cả hai stack.
\end{itemize} \\
\hline
\end{tabularx}
\caption{Tóm tắt ADR-5: Testing Strategy}
\label{tab:adr5-summary}
\end{table}
\renewcommand{\arraystretch}{1.0}

\subsubsection{ADR-6: Security Architecture (AuthN \& AuthZ)}

\renewcommand{\arraystretch}{1.3}
\begin{table}[H]
\centering
\small
\begin{tabularx}{\textwidth}{|M{2.4cm}|Y|}
\hline
\textbf{Hạng mục}\rule{0pt}{1.6em} & \multicolumn{1}{c|}{\textbf{Nội dung}} \\
\hline
\textbf{Bối cảnh} & \begin{itemize}[leftmargin=0.4cm]
    \item Cần cơ chế bảo mật mạnh mẽ cho hệ thống microservices đáp ứng AC6 (Security), FR11 (RBAC) và AC2 (Scalability).
\end{itemize} \\
\hline
\textbf{Quyết định} & \begin{enumerate}[leftmargin=0.4cm]
    \item Authentication: triển khai Auth Service (Java/Spring Security) như Identity Provider tuân thủ OAuth~2.0/OIDC, phát hành JWT (access + refresh).
    \item Authorization ở rìa hệ thống: API Gateway (Go) xác thực JWT cho mọi request bên ngoài.
    \item Authorization nội bộ: Gateway chuyển tiếp `X-User-ID`, `X-User-Roles`; các service nội bộ tin tưởng header và thực thi RBAC.
\end{enumerate} \\
\hline
\textbf{Lý luận} & \begin{itemize}[leftmargin=0.4cm]
    \item Đáp ứng AC6 với thiết kế theo chuẩn OAuth 2.0/OIDC.
    \item Duy trì AC1 bằng cách tách AuthN/AuthZ khỏi dịch vụ nghiệp vụ.
    \item Đảm bảo AC2 nhờ JWT stateless, hỗ trợ mở rộng ngang.
    \item Tránh lặp lại logic xác thực ở từng service, giảm rủi ro không nhất quán.
\end{itemize} \\
\hline
\textbf{Hậu quả} & \textbf{Tích cực:}
\begin{itemize}[leftmargin=0.4cm]
    \item Bảo mật tập trung, đơn giản hóa dịch vụ nghiệp vụ, dễ dàng scale.
\end{itemize}
\textbf{Tiêu cực:}
\begin{itemize}[leftmargin=0.4cm]
    \item Auth Service và Gateway là điểm lỗi đơn, yêu cầu độ sẵn sàng cao.
    \item Mô hình tin tưởng Gateway kém an toàn hơn Zero Trust; JWT cần TTL ngắn và cơ chế refresh phức tạp.
\end{itemize} \\
\hline
\textbf{Rủi ro} & \begin{itemize}[leftmargin=0.4cm]
    \item Rủi ro: Auth Service/Gateway sự cố gây downtime toàn hệ thống. Giảm thiểu: triển khai HA với nhiều replica trên Kubernetes.
    \item Rủi ro: Tấn công nội bộ bypass Gateway và giả mạo header. Giảm thiểu: sử dụng VPC riêng và Network Policies để chỉ Gateway gọi được service nội bộ.
\end{itemize} \\
\hline
\textbf{Các lựa chọn} & \begin{itemize}[leftmargin=0.4cm]
    \item mTLS (Zero Trust): bảo mật cao nhưng quá phức tạp cho giai đoạn hiện tại.
    \item Session Cookies: đơn giản nhưng không phù hợp microservices, không đáp ứng AC2.
    \item Mỗi service tự validate JWT: bảo mật hơn nhưng tăng latency và lặp lại logic, vi phạm SRP.
\end{itemize} \\
\hline
\textbf{Liên quan} & \begin{itemize}[leftmargin=0.4cm]
    \item Phụ thuộc ADR-1 (API Gateway dùng Go, Auth Service dùng Java); ảnh hưởng ADR-7 (JWT cung cấp `LearnerID` ẩn danh).
\end{itemize} \\
\hline
\end{tabularx}
\caption{Tóm tắt ADR-6: Security Architecture}
\label{tab:adr6-summary}
\end{table}
\renewcommand{\arraystretch}{1.0}

\subsubsection{ADR-7: Data Privacy \& Compliance (GDPR/FERPA)}

\renewcommand{\arraystretch}{1.3}
\begin{table}[H]
\centering
\small
\begin{tabularx}{\textwidth}{|M{2.4cm}|Y|}
\hline
\textbf{Hạng mục}\rule{0pt}{1.6em} & \multicolumn{1}{c|}{\textbf{Nội dung}} \\
\hline
\textbf{Bối cảnh} & \begin{itemize}[leftmargin=0.4cm]
    \item ITS xử lý dữ liệu cá nhân nhạy cảm (PII) của học sinh và phải tuân thủ GDPR/FERPA, đáp ứng AC6 (Security).
\end{itemize} \\
\hline
\textbf{Quyết định} & \begin{enumerate}[leftmargin=0.4cm]
    \item Phân tách PII: chỉ `User Management Service` lưu trữ PII (PostgreSQL theo ADR-2).
    \item Ẩn danh hóa: các service khác chỉ tham chiếu `LearnerID` (UUID) đã ẩn danh.
    \item Mã hóa PII khi lưu trữ bằng `pgcrypto`.
    \item Mã hóa khi truyền (TLS cho mọi giao tiếp).
    \item Thực thi ``Right to be Forgotten'' qua API dành cho Admin, phát sự kiện để các service khác xóa dữ liệu liên quan.
\end{enumerate} \\
\hline
\textbf{Lý luận} & \begin{itemize}[leftmargin=0.4cm]
    \item Bảo vệ PII theo nguyên tắc đặc quyền tối thiểu: dịch vụ chấm điểm không cần biết danh tính thật.
    \item Đáp ứng yêu cầu tuân thủ GDPR/FERPA và giảm rủi ro nếu một service bị xâm nhập.
\end{itemize} \\
\hline
\textbf{Hậu quả} & \textbf{Tích cực:}
\begin{itemize}[leftmargin=0.4cm]
    \item Bảo vệ PII ở mức cao, giảm bề mặt tấn công, tuân thủ pháp lý.
\end{itemize}
\textbf{Tiêu cực:}
\begin{itemize}[leftmargin=0.4cm]
    \item Việc ``join'' dữ liệu phức tạp hơn (cần gọi thêm User Service hoặc cache PII).
    \item Mã hóa cấp độ cột ảnh hưởng hiệu năng truy vấn và indexing.
    \item Thực thi Right to be Forgotten trong hệ thống phân tán đòi hỏi cơ chế như Saga pattern.
\end{itemize} \\
\hline
\textbf{Rủi ro} & \begin{itemize}[leftmargin=0.4cm]
    \item Rủi ro: tăng độ trễ khi phải phối hợp nhiều service để tổng hợp dữ liệu. Giảm thiểu: sử dụng API Gateway để aggregate hoặc cache PII ít thay đổi (Redis).
    \item Rủi ro: quy trình xóa không nhất quán, PII còn sót ở service khác. Giảm thiểu: áp dụng Saga pattern dựa trên event để đảm bảo yêu cầu xóa lan tỏa.
\end{itemize} \\
\hline
\textbf{Các lựa chọn} & \begin{itemize}[leftmargin=0.4cm]
    \item Lưu PII ở mọi nơi: đơn giản nhưng vi phạm pháp lý, rủi ro bảo mật cao.
    \item Chỉ mã hóa toàn bộ database: dễ triển khai nhưng không bảo vệ nếu ứng dụng bị xâm nhập; cần mã hóa cấp cột.
\end{itemize} \\
\hline
\textbf{Liên quan} & \begin{itemize}[leftmargin=0.4cm]
    \item Phụ thuộc ADR-2 (PostgreSQL + `pgcrypto`) và ADR-6 (JWT cung cấp khóa ẩn danh `LearnerID`).
\end{itemize} \\
\hline
\end{tabularx}
\caption{Tóm tắt ADR-7: Data Privacy \& Compliance}
\label{tab:adr7-summary}
\end{table}
\renewcommand{\arraystretch}{1.0}

\subsubsection{Kết luận}

\indentpar \indentpar Chuỗi ADR trên được thiết kế xoay quanh các Architecture Characteristics đã được ưu tiên trong Mục~3.1, hiện thực hóa thông qua kiến trúc lai đã chọn ở Mục~3.2. Nói cách khác, các AC xác định ``chúng ta cần gì'', phần lựa chọn kiến trúc quyết định ``chúng ta dùng kiến trúc nào'', và các ADR mô tả ``chúng ta sẽ xây dựng nó như thế nào''. 