\section{Tính Giả Ngẫu Nhiên (Pseudorandomness)}

\subsection{Câu 1. (10 điểm) Bộ Sinh Giả Ngẫu Nhiên (Pseudorandom Generators - PRGs)}

\subsubsection{(a) (5 điểm) Giải thích các hạn chế của mã hóa một lần (one-time pad) đối với việc mã hóa thực tế và tại sao các bộ sinh giả ngẫu nhiên (PRGs) lại cần thiết trong các hệ thống mật mã hiện đại}

\begin{center}
    \textbf{Trả lời:}
\end{center}
\textbf{Hạn chế của Mã Một Lần (OTP):}
OTP cung cấp tính bí mật hoàn hảo (perfect secrecy). Tuy nhiên, hạn chế cốt lõi là khóa ($K$) phải dài ít nhất bằng văn bản gốc ($M$). Điều này tạo ra một nghịch lý hậu cần ("trứng có trước hay gà có trước") vì để truyền thông tin mật ($n$ bit), người dùng đã phải chia sẻ $n$ bit khóa an toàn trước đó. \\
\textbf{Sự cần thiết của PRGs:}
Mật mã học thực tế cần các lược đồ mà khóa ($k_s$) có thể nhỏ hơn nhiều so với thông điệp ($|k_s| < |m|$). Bộ Sinh Giả Ngẫu Nhiên (PRG) là giải pháp: nó nhận một seed ngắn ($S$) và mở rộng nó thành một luồng khóa dài hơn $G(S)$. Mặc dù đầu ra của PRG không thể đạt được phân phối đồng nhất thực sự (do số lượng đầu vào nhỏ hơn không gian đầu ra), nó vẫn được coi là an toàn nếu đầu ra đó không thể phân biệt được về mặt tính toán (computationally indistinguishable) với một chuỗi ngẫu nhiên thực sự.

\subsubsection{(b) (5 điểm) Phân tích các tác động bảo mật của cấu trúc PRG sau đây, trong đó G là một PRG an toàn}

\begin{center}
    H(S) = A‖B‖C‖D trong đó A‖B = G(S) và C‖D = G(B)
\end{center}

Xác định xem H có phải là một PRG an toàn hay không. Nếu không, hãy cung cấp một bộ phân biệt (distinguisher) có thể phân biệt H(S) với một chuỗi thực sự ngẫu nhiên với lợi thế không đáng kể (non-negligible advantage).

\begin{center}
    \textbf{Trả lời:}
\end{center}

\textbf{1) Cấu trúc PRG nối tiếp $H(S)$.}

\[
H(S) = A\|B\|C\|D,\quad A\|B = G(S),\quad C\|D = G(B)
\]

Trong đó $G$ là một PRG an toàn. Điểm yếu cốt lõi nằm ở việc tái sử dụng nửa sau $B$ làm seed mới cho lần mở rộng tiếp theo.

\textbf{2) Ý tưởng chính: Mối liên hệ nội tại.}

Nếu $H(S)$ là chuỗi ngẫu nhiên thực sự thì bốn phần $A,B,C,D$ sẽ độc lập. Tuy nhiên, trong $H(S)$ thật ta có ràng buộc tất định $C\|D = G(B)$, tạo nên mối liên hệ kiểm chứng được giữa nửa trước và nửa sau.

\textbf{3) Bộ phân biệt $D$ hoạt động như thế nào.}

Khi nhận chuỗi thử thách $Y = Y_1\|Y_2\|Y_3\|Y_4$, bộ phân biệt $D$ thực hiện:
\begin{center}
\begin{tabular}{|c|l|l|}
\hline
\textbf{Bước} & \textbf{Hành động} & \textbf{Mục đích} \\
\hline
1 & Lấy $Y_2$ (ứng với $B$) & Xác định seed thứ hai \\
2 & Tính $G(Y_2)$ & Dự đoán giá trị $Y_3\|Y_4$ nếu là thật \\
3 & So sánh $G(Y_2) \stackrel{?}{=} Y_3\|Y_4$ & Kiểm tra quan hệ nội tại \\
4 & Khớp: đoán REAL; ngược lại: RAND & Phân biệt nguồn chuỗi \\
\hline
\end{tabular}
\end{center}

\textbf{4) Phân tích xác suất và lợi thế.}

Trường hợp chuỗi thật $\left( L^{H}_{\text{real}} \right)$: quan hệ $G(B)=C\|D$ luôn đúng nên $\Pr[\text{$D$ đoán đúng}] = 1$.

Trường hợp chuỗi ngẫu nhiên $\left( L^{H}_{\text{rand}} \right)$: $Y_2, Y_3, Y_4$ độc lập, do đó
\[
\Pr\big[ G(Y_2) = Y_3\|Y_4 \big] = \frac{1}{2^{2\lambda}}\,.
\]

\textbf{5) Kết luận lợi thế.}

\[
\mathrm{Adv}(D) = 1 - \frac{1}{2^{2\lambda}} \approx 1\,.
\]

Lợi thế xấp xỉ 1 (\textit{không thể bỏ qua, non-negligible}), do đó $D$ phân biệt dễ dàng giữa $H(S)$ thật và chuỗi ngẫu nhiên. Suy ra $H(S)$ \textbf{không} phải là một PRG an toàn.

\textit{Tóm tắt một câu:} Cấu trúc $H(S)$ thất bại vì đầu ra sau phụ thuộc có thể kiểm chứng vào đầu ra trước ($C\|D = G(B)$), khiến bộ phân biệt dễ dàng nhận biết chuỗi thật so với chuỗi ngẫu nhiên.

\subsection{Câu 2. (10 điểm) Hàm Giả Ngẫu Nhiên và Hoán Vị (Pseudorandom Functions and Permutations)}

\subsubsection{(a) (5 điểm) Phân tích cấu trúc PRF $F(K,X) = G(K) \oplus X$}

\textbf{Yêu cầu:} Xét cấu trúc PRF $F(K,X) = G(K) \oplus X$, trong đó $G$ là một PRG an toàn. Xác định $F$ có phải là một PRF an toàn không. Nếu không, mô tả một bộ phân biệt có thể phân biệt hiệu quả $F$ với một hàm ngẫu nhiên.

\begin{center}
    \textbf{Trả lời:}
\end{center}

$F(K,X) = G(K) \oplus X$ \textbf{không} phải là một PRF an toàn vì tồn tại quan hệ tuyến tính loại bỏ $G(K)$ khi so sánh hai truy vấn trên cùng một khóa.

Xét bộ phân biệt $D$:
\begin{enumerate}
  \item Truy vấn \textit{oracle} với hai đầu vào tùy ý $X_1, X_2$ để nhận $Y_1 = F(K,X_1)$ và $Y_2 = F(K,X_2)$.
    \item Tính và kiểm tra $Y_1 \oplus Y_2 \stackrel{?}{=} X_1 \oplus X_2$.
    \item Nếu đúng thì đoán REAL; ngược lại đoán RAND.
\end{enumerate}

Phân tích lợi thế:
\[
Y_1 \oplus Y_2 = (G(K) \oplus X_1) \oplus (G(K) \oplus X_2) = X_1 \oplus X_2
\]
Trong thư viện $L^{F}_{prf\text{-}real}$ đẳng thức luôn đúng (xác suất 1). Còn trong $L^{F}_{prf\text{-}rand}$, $Y_1, Y_2$ độc lập nên $\Pr[\,Y_1 \oplus Y_2 = X_1 \oplus X_2\,] = 2^{-m}$ với $m$ là độ dài đầu ra. Do đó $\mathrm{Adv}(D) \approx 1$.

\subsubsection{(b) (5 điểm) So sánh PRFs và PRPs}

\textbf{Yêu cầu:}
\begin{enumerate}
    \item Nêu khác biệt chính trong định nghĩa và thuộc tính.
    \item Mô tả cách PRPs có thể được "hạ cấp" thành PRFs, nhưng không nhất thiết ngược lại.
    \item Giải thích vì sao va chạm là không thể tránh khỏi với PRFs nhưng không phải với PRPs.
\end{enumerate}

\begin{center}
    \textbf{Trả lời:}
\end{center}

\begin{center}
\begin{tabular}{|l|p{5.3cm}|p{5.3cm}|}
        \hline
        \textbf{Thuộc tính} & \textbf{PRF (Hàm giả ngẫu nhiên)} & \textbf{PRP (Hoán vị giả ngẫu nhiên)} \\
        \hline
        Định nghĩa & Hàm có khóa, đầu ra không phân biệt được với hàm ngẫu nhiên lý tưởng & Hoán vị có khóa (song ánh), không phân biệt được với hoán vị ngẫu nhiên lý tưởng \\
        \hline
        Khả nghịch & Không yêu cầu nghịch đảo hiệu quả & Bắt buộc tồn tại nghịch đảo $F^{-1}$ hiệu quả \\
        \hline
        Va chạm đầu ra & Có thể xảy ra (do nguyên lý sinh nhật) & Không xảy ra (song ánh) \\
        \hline
Truy vấn & Truy vấn chọn lọc tới \textit{oracle} & Truy vấn chọn lọc tới \textit{oracle} (cả $F$ và thường cả $F^{-1}$) \\
        \hline
    \end{tabular}
\end{center}

Suy biến (PRP $\Rightarrow$ PRF): Mọi PRP cũng là PRF khi chỉ xét tính giả ngẫu nhiên (bỏ qua khả nghịch), nhưng không phải mọi PRF đều là PRP vì PRF không đảm bảo tính song ánh/khả nghịch.

Va chạm: PRF mô phỏng hàm ngẫu nhiên nên va chạm là không tránh khỏi khi ánh xạ từ không gian đầu vào lớn sang không gian đầu ra cố định (nguyên lý lỗ bồ câu); PRP là song ánh nên không có va chạm.
