% \subsection{Ràng Buộc và Giả Định}

% \indentpar \indentpar Phần này xác định các ràng buộc (constraints) ảnh hưởng đến phạm vi và quyết định thiết kế của hệ thống ITS, cùng với các giả định (assumptions) được đặt ra trong giai đoạn phân tích và thiết kế kiến trúc.
% Các yếu tố này giúp định hình phạm vi thực thi của dự án, làm cơ sở để đánh giá tính khả thi và xác định giới hạn mở rộng của hệ thống.

% \subsubsection{Technical Constraints (Ràng buộc kỹ thuật)}

% \begin{table}[ht]
% \centering
% \small
% \renewcommand{\tabularxcolumn}[1]{m{#1}}
% \renewcommand{\arraystretch}{1.5}
% \begin{tabularx}{\textwidth}{|>{\centering\arraybackslash}m{3.5cm}|>{\noindent\justifying\arraybackslash}X|}
% \hline
% \textbf{Hạng mục} & \textbf{Mô tả} \\
% \hline
% Quy mô đội ngũ phát triển & 4--5 thành viên (bao gồm Backend Developer, DevOps Engineer, AI Engineer, QA). Giới hạn nguồn lực khiến việc mở rộng chức năng cần được ưu tiên theo mức độ quan trọng (MoSCoW). \\
% \hline
% Công nghệ được phép sử dụng & Backend: Golang (GIN), Java (Spring Boot)\newline Frontend: Next.js 14 + TypeScript\newline Database: PostgreSQL, MongoDB, Redis\newline AI/ML: Python (FastAPI/Whisper/ONNX), Golang ML binding\newline Các công nghệ khác ngoài danh sách này không được đưa vào giai đoạn MVP. \\
% \hline
% Giới hạn hạ tầng triển khai & Sử dụng cluster Kubernetes 3 node (vCPU: 8, RAM: 32GB/node). Không được vượt quá quota tài nguyên đã cấp. \\
% \hline
% Ngân sách hạ tầng & Tối đa $\$300$/tháng, bao gồm chi phí cloud (GKE hoặc self-hosted k8s), MinIO object storage, và Redis caching. \\
% \hline
% Quy định bảo mật dữ liệu & Phải tuân thủ tiêu chuẩn FERPA và tương thích với yêu cầu GDPR khi xử lý dữ liệu người học (profile, điểm số). \\
% \hline
% Giới hạn hiệu năng mô hình AI & Các mô hình AI (Adaptive Engine, Feedback Generator) phải chạy được trên CPU, không yêu cầu GPU để giảm chi phí triển khai. \\
% \hline
% Giới hạn phần mềm phụ thuộc & Không sử dụng framework hoặc SDK thương mại có phí bản quyền. Ưu tiên mã nguồn mở. \\
% \hline
% Hệ thống tích hợp bên ngoài & ITS chỉ tích hợp với Auth Provider (JWT/OAuth2) và LMS API chuẩn LTI 1.3 trong giai đoạn đầu. \\
% \hline
% \end{tabularx}
% \renewcommand{\arraystretch}{1.0}
% \caption{Technical Constraints}
% \label{tab:technical_constraints}
% \end{table}
% \FloatBarrier

% \subsubsection{Business Constraints (Ràng buộc kinh doanh)}

% \begin{table}[ht]
% \centering
% \small
% \renewcommand{\tabularxcolumn}[1]{m{#1}}
% \renewcommand{\arraystretch}{1.5}
% \begin{tabularx}{\textwidth}{|>{\centering\arraybackslash}m{3.5cm}|>{\noindent\justifying\arraybackslash}X|}
% \hline
% \textbf{Hạng mục} & \textbf{Mô tả} \\
% \hline
% Thời gian thực hiện (Timeline) & Tổng thời gian dự án: 12 tuần (3 tháng), trong đó:\newline -- Tuần 1--3: Phân tích và thiết kế kiến trúc\newline -- Tuần 4--8: Phát triển và tích hợp các microservice chính (Auth, Learner, Adaptive Engine)\newline -- Tuần 9--12: Kiểm thử, tối ưu và triển khai MVP. \\
% \hline
% Ngân sách tổng thể & $\$1.000$ cho toàn bộ giai đoạn MVP (bao gồm hạ tầng, CI/CD, tài nguyên lưu trữ, domain và chứng chỉ bảo mật SSL). \\
% \hline
% Phạm vi phiên bản MVP & Chỉ bao gồm các chức năng cốt lõi: Học tập thích ứng (Adaptive Learning), Chấm điểm và Phản hồi tức thì, Báo cáo tiến độ. Các tính năng phụ như Gamification hoặc Chat 1-1 sẽ được lên kế hoạch cho phiên bản mở rộng. \\
% \hline
% Quy định và tuân thủ (Compliance) & Tuân thủ quy định bảo mật FERPA (Mỹ) và GDPR (Châu Âu).\newline Các thành phần xử lý dữ liệu cá nhân phải có cơ chế audit log và role-based access control. \\
% \hline
% Môi trường vận hành & ITS phải hoạt động ổn định trên môi trường cloud (GCP/GKE hoặc AWS EKS) và có khả năng triển khai on-premise cho cơ sở giáo dục nhỏ. \\
% \hline
% Chiến lược mở rộng & Mọi thành phần (service, database) phải được thiết kế theo nguyên tắc cloud-native, đảm bảo khả năng mở rộng ngang (scale-out) mà không cần viết lại logic nghiệp vụ. \\
% \hline
% Phụ thuộc nhân sự và lịch học kỳ & Mốc triển khai phải hoàn thành trước khi bắt đầu học kỳ mới, để có thể demo trong báo cáo môn học Kiến trúc Phần mềm. \\
% \hline
% \end{tabularx}
% \renewcommand{\arraystretch}{1.0}
% \caption{Business Constraints}
% \label{tab:business_constraints}
% \end{table}
% \FloatBarrier

% \subsubsection{Assumptions (Giả định)}

% \begin{table}[ht]
% \centering
% \small
% \renewcommand{\tabularxcolumn}[1]{m{#1}}
% \renewcommand{\arraystretch}{1.5}
% \begin{tabularx}{\textwidth}{|>{\centering\arraybackslash}m{4.5cm}|>{\noindent\justifying\arraybackslash}X|}
% \hline
% \textbf{Giả định} & \textbf{Giải thích} \\
% \hline
% Người dùng có kết nối Internet ổn định $\geq 10$ Mbps & Đảm bảo khả năng stream nội dung học tập (video, quiz, bài tập coding) mà không gián đoạn. \\
% \hline
% Trình duyệt hiện đại (Chrome 90+, Firefox 88+, Safari 14+) & Hệ thống frontend sử dụng WebSocket và API hiện đại, không đảm bảo tương thích với trình duyệt cũ. \\
% \hline
% Người dùng có tài khoản xác thực hợp lệ & Tất cả người học và giảng viên phải đăng ký tài khoản trước khi truy cập hệ thống. \\
% \hline
% Server vận hành 24/7 & Đảm bảo tính sẵn sàng cao (SLA $\geq 99.5\%$) và hỗ trợ học tập không giới hạn thời gian. \\
% \hline
% Cơ sở dữ liệu được sao lưu hàng ngày & Phục hồi dữ liệu nhanh trong trường hợp lỗi hoặc mất kết nối. \\
% \hline
% Các service hoạt động trong cùng một VPC nội bộ & Đảm bảo latency thấp và không bị giới hạn bởi firewall của nhà cung cấp cloud. \\
% \hline
% Người dùng chấp nhận chính sách quyền riêng tư (Privacy Policy) & Việc lưu trữ và xử lý dữ liệu học tập tuân theo quy định về bảo mật. \\
% \hline
% Không có downtime trong giờ học cao điểm (08:00--22:00) & Mọi đợt cập nhật hoặc bảo trì hệ thống được thực hiện sau 22:00. \\
% \hline
% Hệ thống phát triển và kiểm thử trong môi trường staging tương tự production & Đảm bảo tính nhất quán giữa môi trường phát triển và triển khai. \\
% \hline
% \end{tabularx}
% \renewcommand{\arraystretch}{1.0}
% \caption{Assumptions}
% \label{tab:assumptions}
% \end{table}
% \FloatBarrier

% \noindent\textbf{Tóm tắt}

% Các ràng buộc và giả định trên đảm bảo thiết kế kiến trúc ITS:
% \begin{itemize}
%     \item Hiện thực trong giới hạn nguồn lực và ngân sách của nhóm phát triển.
%     \item Tuân thủ tiêu chuẩn bảo mật và quyền riêng tư cho dữ liệu học viên.
%     \item Duy trì tính khả thi của MVP, với nền tảng vững chắc cho các giai đoạn mở rộng sau.
% \end{itemize}
