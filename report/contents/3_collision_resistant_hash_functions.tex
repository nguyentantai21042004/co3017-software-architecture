
\section{Hàm Băm Chống Va Chạm (Collision-Resistant Hash Functions)}

\subsection{Câu 1. (15 điểm) Thuộc tính Hàm băm}

\subsubsection{(a) (5 điểm) Khả năng chống va chạm (Collision Resistance)}

\textbf{Yêu cầu:}
\begin{enumerate}
    \item Giải thích tại sao va chạm phải tồn tại trong bất kỳ hàm băm nào ánh xạ đầu vào có độ dài tùy ý sang đầu ra có độ dài cố định.
    \item Sử dụng nghịch lý sinh nhật (birthday paradox), tính toán xấp xỉ số lượng băm phải được tính để tìm va chạm với xác suất 50\% trong một hàm băm an toàn 256-bit.
    \item Mô tả một kịch bản tấn công thực tế trong đó việc tìm va chạm băm sẽ làm tổn hại đến một hệ thống bảo mật.
\end{enumerate}

\begin{center}
\textbf{Trả lời:}
\end{center}

\textbf{1) Sự tồn tại của va chạm.} Va chạm chắc chắn tồn tại do nguyên lý lỗ bồ câu: ánh xạ từ tập vô hạn (đầu vào có độ dài tùy ý) sang tập hữu hạn (đầu ra có độ dài cố định). An ninh hàm băm là \textit{va chạm không thể tìm thấy trong thực tế tính toán}, chứ không phải va chạm không tồn tại.

\textbf{2) Tính toán birthday paradox.} Với hàm băm 256-bit, số lượng băm cần thiết để tìm va chạm với xác suất 50\%:
\[
Q \approx \sqrt{2^{256}} = 2^{128}
\]
Cần khoảng $2^{128}$ phép toán để tìm một va chạm.

\textbf{3) Kịch bản tấn công thực tế (giả mạo tài liệu).} Kẻ tấn công tạo hai tài liệu khác nhau $M_{\text{hợp pháp}}$ và $M_{\text{gian lận}}$ sao cho $H(M_{\text{hợp pháp}}) = H(M_{\text{gian lận}})$. Yêu cầu nạn nhân ký điện tử lên $H(M_{\text{hợp pháp}})$, sau đó gắn chữ ký vào $M_{\text{gian lận}}$, khiến nạn nhân bị ràng buộc với tài liệu gian lận.

\subsubsection{(b) (5 điểm) Cấu trúc hàm băm}

\textbf{Yêu cầu:}
\begin{enumerate}
    \item So sánh và đối chiếu cấu trúc Merkle-Damgård (SHA-2) và cấu trúc Sponge (SHA-3).
    \item Giải thích cách tấn công mở rộng độ dài hoạt động chống Merkle-Damgård và tại sao Sponge kháng được.
    \item Mô tả cấu trúc HMAC và cách nó bảo vệ chống tấn công mở rộng độ dài.
\end{enumerate}

\begin{center}
\textbf{Trả lời:}
\end{center}

\textbf{1) So sánh Merkle-Damgård và Sponge.}
\begin{itemize}
    \item \textbf{Merkle-Damgård (SHA-2):} Thông điệp được đệm và chia khối, xử lý tuần tự qua hàm nén. Dễ bị tấn công mở rộng độ dài.
    \item \textbf{Sponge (SHA-3):} Hoạt động qua hai pha: hấp thụ và vắt. Phần \textit{capacity} của trạng thái nội bộ bị ẩn, chỉ phần \textit{rate} được tiết lộ.
\end{itemize}

\textbf{2) Tấn công mở rộng độ dài.}
\begin{itemize}
    \item \textbf{Merkle-Damgård:} Nếu biết $H(M)$ và độ dài $M$, có thể tính $H(M \| \text{padding} \| P)$ mà không cần biết nội dung $M$.
    \item \textbf{Sponge:} Kháng được vì phần \textit{capacity} không được tiết lộ, không thể tiếp tục quá trình băm từ giá trị đã biết.
\end{itemize}

\textbf{3) Cấu trúc HMAC.}
\[
\text{HMAC}(K, m) = H((K \oplus \text{opad}) \| H((K \oplus \text{ipad}) \| m))
\]
HMAC chống được tấn công mở rộng độ dài vì khóa $K$ được dùng ở cả hai lớp băm, ngăn kẻ tấn công mở rộng thông điệp mà không biết khóa.

\subsubsection{(c) (5 điểm) Sự phát triển của hàm băm}

\textbf{Yêu cầu:}
\begin{enumerate}
    \item Mô tả các tấn công thành công chống MD5 và SHA-1 dẫn đến việc chúng bị loại bỏ.
    \item Giải thích khái niệm va chạm tiền tố đã chọn và tại sao chúng đặc biệt nguy hiểm với cơ quan cấp chứng chỉ.
    \item So sánh bảo mật của SHA-2 và SHA-3 chống các kỹ thuật phân tích mật mã đã biết.
\end{enumerate}

\begin{center}
\textbf{Trả lời:}
\end{center}

\textbf{1) Tấn công chống MD5 và SHA-1.}
\begin{itemize}
    \item \textbf{MD5 (128-bit):} Bị phá vỡ hoàn toàn từ năm 2004 (Xiaoyun Wang).
    \item \textbf{SHA-1 (160-bit):} Bị phá vỡ năm 2017 trong tấn công \textbf{SHAttered} của Google và CWI Amsterdam, yêu cầu sức mạnh tính toán tương đương 6,500 CPU-năm và 110 GPU-năm.
\end{itemize}

\textbf{2) Va chạm tiền tố đã chọn.} Cho hai tiền tố bất kỳ $P_1, P_2$, tồn tại các hậu tố $S_1, S_2$ sao cho $H(P_1 \| S_1) = H(P_2 \| S_2)$. Đặc biệt nguy hiểm với cơ quan cấp chứng chỉ vì có thể giả mạo chứng chỉ hợp pháp.

\textbf{3) So sánh SHA-2 và SHA-3.}
\begin{itemize}
    \item \textbf{SHA-2:} Sử dụng Merkle-Damgård, vẫn an toàn (256/512-bit), chưa có tấn công thực tế thành công.
    \item \textbf{SHA-3 (Keccak):} Sử dụng cấu trúc Sponge hoàn toàn mới, kháng các kỹ thuật đã phá MD5/SHA-1, cung cấp \textbf{Protocol Agility}.
\end{itemize}

\subsection{Câu 2. (15 điểm) Băm mật khẩu (Password Hashing)}

\subsubsection{(a) (5 điểm) Phân tích lưu trữ mật khẩu}

\textbf{Yêu cầu:} Phân tích tác động bảo mật nếu DB bị xâm phạm cho các cách lưu trữ: (1) plaintext, (2) mã hóa cùng khóa trên server, (3) SHA-256 không muối, (4) SHA-256 có muối, (5) Scrypt.

\begin{center}
\textbf{Trả lời:}
\end{center}

\begin{table}[H]
\centering
\begin{tabular}{|c|l|l|p{7cm}|}
\hline
\textbf{\#} & \textbf{Phương pháp lưu trữ} & \textbf{Tác động bảo mật} & \textbf{Phân tích} \\
\hline
1 & Plaintext & Thảm họa & Kẻ tấn công lấy ngay lập tức tất cả mật khẩu. \\
\hline
2 & Mã hóa (khóa trên cùng server) & Thất bại & Kẻ tấn công lấy được cả khóa giải mã, làm thất bại mã hóa. \\
\hline
3 & SHA-256 không muối & Kém & Dễ bị bảng rainbow/\,từ điển vì cùng mật khẩu $\Rightarrow$ cùng hash. \\
\hline
4 & SHA-256 có muối & Tốt & Kháng rainbow, nhưng yếu nếu băm quá nhanh (brute-force dễ). \\
\hline
5 & Scrypt/Argon2 & Xuất sắc & Muối + memory-hard giảm lợi thế GPU/\,ASIC. \\
\hline
\end{tabular}
\caption{So sánh các phương pháp lưu trữ mật khẩu}
\end{table}

\subsubsection{(b) (5 điểm) Muối (Salting)}

\textbf{Yêu cầu:}
\begin{enumerate}
    \item Muối chống bảng rainbow như thế nào.
    \item Tính dung lượng lưu trữ với 10{,}000 người dùng, muối 16B, hash 32B.
    \item Thực hành tốt nhất tạo/\,lưu trữ muối.
\end{enumerate}

\begin{center}
\textbf{Trả lời:}
\end{center}

\begin{table}[H]
\centering
\begin{tabular}{|c|l|l|p{7cm}|}
\hline
\textbf{\#} & \textbf{Phương pháp lưu trữ} & \textbf{Tác động bảo mật} & \textbf{Phân tích} \\
\hline
1 & Plaintext & Thảm họa & Kẻ tấn công lấy ngay lập tức tất cả mật khẩu. \\
\hline
2 & Mã hóa (khóa trên cùng server) & Thất bại & Kẻ tấn công lấy được cả khóa giải mã, làm thất bại mã hóa. \\
\hline
3 & SHA-256 không muối & Kém & Dễ bị bảng rainbow/\,từ điển vì cùng mật khẩu $\Rightarrow$ cùng hash. \\
\hline
4 & SHA-256 có muối & Tốt & Kháng rainbow, nhưng yếu nếu băm quá nhanh (brute-force dễ). \\
\hline
5 & Scrypt/Argon2 & Xuất sắc & Muối + memory-hard giảm lợi thế GPU/\,ASIC. \\
\hline
\end{tabular}
\caption{So sánh các phương pháp lưu trữ mật khẩu}
\end{table}

\subsubsection{(c) (5 điểm) Hàm băm mật khẩu chuyên dụng}

\textbf{Yêu cầu:} So sánh memory-hard (Scrypt/Argon2) với PBKDF2; giải thích tham số Scrypt $(N,r,p)$; so sánh tốc độ SHA-256, PBKDF2, Scrypt và ý nghĩa bảo mật.

\begin{center}
\textbf{Trả lời:}
\end{center}

\textbf{Memory-hard vs PBKDF2.} PBKDF2 tăng CPU qua lặp PRF; Scrypt/Argon2 buộc dùng bộ nhớ lớn $\Rightarrow$ giảm lợi thế GPU/\,ASIC.

\textbf{Tham số Scrypt.}
\begin{itemize}
    \item $N$ (CPU cost): Tăng số vòng lặp, tăng thời gian tính toán.
    \item $r$ (block size): Tăng băng thông/\,bộ nhớ.
    \item $p$ (parallelization): Tăng song song hóa và chi phí tổng.
\end{itemize}

\textbf{Tốc độ và tác động.}
\[
\text{SHA-256} \gg \text{PBKDF2} \gg \text{Scrypt/Argon2}
\]
GPU hiện đại $>2\times10^9$ SHA-256/s, nhưng chỉ \textit{vài nghìn} Scrypt/Argon2/s $\Rightarrow$ chi phí vét cạn tăng mạnh, làm tấn công ngoại tuyến kém khả thi.
