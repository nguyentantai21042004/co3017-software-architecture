\section{Góc Nhìn Kiến Trúc}

\indentpar \indentpar Chương này trình bày chi tiết kiến trúc của hệ thống ITS thông qua các góc nhìn (views) khác nhau, tuân theo mô hình 4+1 View Model. Mỗi góc nhìn làm rõ một khía cạnh của hệ thống---từ cấu trúc tổ chức mã nguồn đến hành vi và triển khai---giúp các bên liên quan (stakeholders) có một cái nhìn thống nhất và toàn diện.

Nếu Chương 3 tập trung vào việc ``chọn kiến trúc nào'' và ``quyết định ra sao'', thì Chương 4 trả lời câu hỏi ``kiến trúc đó trông như thế nào trong thực tế?''

\subsection{Module View}

\indentpar \indentpar Bắt đầu với Module View, góc nhìn mô tả cách hệ thống được phân rã và tổ chức code nội bộ theo Clean Architecture và Domain-Driven Design.
Module View (Góc nhìn Module) mô tả sự phân rã của hệ thống thành các thành phần (components) và các module con, cùng mối quan hệ giữa chúng. Góc nhìn này trả lời cho hai câu hỏi cốt lõi: \textit{``Hệ thống được tổ chức như thế nào?''} và \textit{``Các phần code bên trong được cấu trúc ra sao?''}

Module View đặc biệt quan trọng vì nó thể hiện cách các quyết định trong ADR-1 \textrightarrow{} ADR-4 (Polyglot, Clean Architecture, Repository Pattern) được hiện thực hóa ở cấp mã nguồn, giúp đạt được các đặc tính kiến trúc quan trọng nhất của ITS:
\textbf{AC1 -- Modularity}, \textbf{AC4 -- Testability}, và \textbf{AC7 -- Maintainability}.

\subsubsection{Phân rã Hệ thống}

\indentpar \indentpar Hệ thống ITS được thiết kế theo phong cách \textbf{Hybrid Microservices + Event-Driven Architecture}. Việc phân rã hệ thống \textbf{(System Decomposition)} dựa trên Bounded Contexts của Domain-Driven Design (DDD).

\textbf{Lưu ý về phạm vi MVP:} Trong giai đoạn hiện tại (MVP), nhóm tập trung hiện thực các service cốt lõi liên quan trực tiếp đến luồng học tập thích ứng (Content, Scoring, Learner Model, Adaptive Engine). Các service hỗ trợ như \textbf{User Management} và \textbf{Auth Service} được mô tả trong kiến trúc đích (Target Architecture) nhưng chưa được tách thành microservice riêng biệt trong phiên bản code hiện tại; thay vào đó, các chức năng này được giả lập hoặc xử lý đơn giản để ưu tiên kiểm chứng logic nghiệp vụ chính.

\vspace{1em}

\begin{figure}[H]
    \centering
    {\includegraphics[width=0.95\textwidth]{images/system_decomposition.png}}
    \caption{Sơ đồ phân rã module cấp cao của hệ thống ITS}
    \label{fig:system-decomposition}
\end{figure}

Phân rã này cũng thể hiện rõ \textbf{``Bounded Contexts''} của DDD, cho phép mỗi service có thể triển khai, kiểm thử và mở rộng độc lập mà vẫn phối hợp thông qua event bus (RabbitMQ theo ADR-8).

\subsubsection{Cấu trúc các Tầng}

\indentpar \indentpar Để hiện thực hóa ADR-3 (Clean/Hexagonal Architecture), tất cả service (Java và Go) đều tuân thủ kiến trúc phân tầng đồng tâm. Cách tiếp cận này tách biệt rõ Business Logic và Infrastructure, qua đó bảo đảm khả năng kiểm thử và bảo trì lâu dài.

\noindent\textbf{Quy tắc Phụ thuộc (Dependency Rule)}
\begin{itemize}[leftmargin=1.5em]
    \item Luồng phụ thuộc của mã nguồn luôn đi từ vòng ngoài vào vòng trong
    \item Tầng trung tâm (Domain, Application) không phụ thuộc vào Database, Framework hay UI
    \item Giao tiếp từ trong ra ngoài phải thông qua các interface theo nguyên tắc Dependency Inversion
\end{itemize}

\vspace{1em}

\begin{figure}[H]
    \centering
    \includegraphics[width=0.55\textwidth]{images/clean-architecture-layers.png}
    \caption{Sơ đồ tầng Clean Architecture với Dependency Rule}
    \label{fig:clean-architecture-layers}
\end{figure}

\renewcommand{\arraystretch}{1.3}
\begin{table}[H]
    \centering
    \small
    \begin{tabularx}{\textwidth}{|>{\centering\arraybackslash}p{2.5cm}|>{\raggedright\arraybackslash}X|>{\raggedright\arraybackslash}p{3.2cm}|>{\centering\arraybackslash}X|}
        \hline
        \textbf{Tầng}      & \textbf{Trách nhiệm}                                                                                                      & \textbf{Quy tắc phụ thuộc}                                                                     & \textbf{Ví dụ cụ thể trong ITS}                                                                                                                    \\
        \hline
        Domain             & Định nghĩa các khái niệm cốt lõi của nghiệp vụ (Entities, Value Objects, Domain Events nội bộ) cùng các quy tắc bất biến. & Hoàn toàn độc lập; không tham chiếu ra ngoài mà chỉ phơi bày interface để tầng khác hiện thực. & \begin{tabular}{@{}c@{}}\texttt{Learner.java}\\\texttt{Course.java}\\\texttt{SkillMasteryScore.java}\\\texttt{LearnerRepository.java}\end{tabular} \\
        \hline
        Application        & Đóng gói các Use Case, điều phối tương tác giữa Domain qua các Input/Output Port và thực thi kịch bản nghiệp vụ.          & Chỉ được phép phụ thuộc vào Domain; mọi phụ thuộc khác phải đi qua port.                       & \begin{tabular}{@{}c@{}}\texttt{CreateNewCourseUseCase.java}\\\texttt{UpdateSkillMasteryUseCase.java}\\\texttt{SkillMasteryDTO.java}\end{tabular}  \\
        \hline
        Interface Adapters & Chuyển đổi dữ liệu, hiện thực các port, gom/chiếu dữ liệu giữa Application với bên ngoài (HTTP, DB, MQ).                  & Phụ thuộc ngược vào Application và Domain để triển khai các adapter cụ thể.                    & \begin{tabular}{@{}c@{}}\texttt{LearnerController.java}\\\texttt{PostgresLearnerRepository.java}\end{tabular}                                      \\
        \hline
        Infrastructure     & Cung cấp chi tiết kỹ thuật: framework, driver, cấu hình triển khai, message broker, scheduler.                            & Phụ thuộc vào Interface Adapters để cung ứng tài nguyên và wiring hạ tầng.                     & \begin{tabular}{@{}c@{}}Spring Boot\\Gin\\PostgreSQL driver\\RabbitMQ client\end{tabular}                                                          \\
        \hline
    \end{tabularx}
    \caption{Vai trò và quy tắc phụ thuộc của các tầng Clean Architecture}
    \label{tab:clean-architecture-layers}
\end{table}
\renewcommand{\arraystretch}{1.0}

\subsubsection{Cấu trúc Gói}

\indentpar \indentpar Việc chuẩn hóa, minh họa cách tổ chức mã nguồn thực tế (Java \& Go) sao cho phù hợp với Clean Architecture và Polyglot Strategy (ADR-1) nhằm đảm bảo tính nhất quán giữa các ngôn ngữ, đồng thời giúp đội ngũ phát triển dễ dàng định hướng khi làm việc với các service khác nhau.

\noindent\textbf{a. Service Java (Content Service)}

\noindent\textbf{Cấu trúc thư mục}
\begin{center}
    \begin{minipage}{0.95\textwidth}
        \begin{verbatim}
src/main/java/co3017/microservices/content_service/
|-- adapter/
|   `-- http/                # REST controllers (QuestionController)
|-- config/                  # Configuration (CorsConfig)
|-- models/                  # Domain entities (Question)
|-- repository/              # JPA Repositories (QuestionRepository)
|-- usecase/                 # Business Logic
|   |-- QuestionUseCase.java # Interface
|   `-- QuestionService.java # Implementation
`-- ContentServiceApplication.java # Entry Point
\end{verbatim}
    \end{minipage}
\end{center}

\noindent\textbf{Giải thích cấu trúc}
\begin{itemize}[leftmargin=1.2em]
    \item \textbf{Domain Layer} (\texttt{models/}): Chứa các Entity JPA như \texttt{Question}. Đây là trung tâm của nghiệp vụ.
    \item \textbf{Application Layer} (\texttt{usecase/}): Chứa logic nghiệp vụ cốt lõi. \texttt{QuestionService} thực thi các quy tắc nghiệp vụ và gọi xuống Repository.
    \item \textbf{Interface Adapters} (\texttt{adapter/http}): Chứa Controller để xử lý HTTP request/response, đóng vai trò là cổng giao tiếp với bên ngoài.
    \item \textbf{Infrastructure} (\texttt{repository/}, \texttt{config/}): Triển khai các interface lưu trữ (Spring Data JPA) và cấu hình hệ thống.
\end{itemize}

\noindent\textbf{b. Service Go (Scoring, Learner Model, Adaptive Engine)}

Các service viết bằng Go tuân thủ \textbf{Standard Go Project Layout}, tối ưu cho microservices.

\noindent\textbf{Cấu trúc thư mục (Ví dụ: Scoring Service)}
\begin{center}
    \begin{minipage}{0.95\textwidth}
        \begin{verbatim}
sources/scoring/
|-- cmd/
|   `-- api/                 # Main entry point (main.go)
|-- internal/
|   `-- scoring/             # Module chính
|       |-- delivery/        # HTTP Handlers (Gin)
|       |-- usecase/         # Business Logic
|       |-- repository/      # Database Access (GORM)
|       `-- model/           # Structs & Entities
|-- pkg/                     # Shared libraries (nếu có)
`-- go.mod                   # Dependency definitions
\end{verbatim}
    \end{minipage}
\end{center}

\noindent\textbf{Giải thích cấu trúc}
\begin{itemize}[leftmargin=1.2em]
    \item \textbf{cmd/api}: Chứa hàm \texttt{main}, nơi khởi tạo ứng dụng, kết nối DB và đăng ký các route.
    \item \textbf{internal/}: Chứa mã nguồn riêng tư của service, không thể import từ bên ngoài.
          \begin{itemize}
              \item \textbf{model}: Định nghĩa các struct dữ liệu (Entity).
              \item \textbf{repository}: Tương tác trực tiếp với cơ sở dữ liệu.
              \item \textbf{usecase}: Chứa logic nghiệp vụ, gọi repository để lấy dữ liệu.
              \item \textbf{delivery}: Xử lý HTTP request, parse JSON và gọi usecase.
          \end{itemize}
\end{itemize}

Cách tổ chức này giúp code Go gọn gàng, dễ đọc và tuân thủ nguyên tắc "Separation of Concerns" tương tự như Clean Architecture nhưng được điều chỉnh cho phù hợp với đặc thù ngôn ngữ Go.

\subsubsection{Thiết kế Dữ liệu (Data Persistence Design)}

\indentpar \indentpar Tuân thủ nguyên tắc \textbf{Database-per-Service} của kiến trúc Microservices, hệ thống không sử dụng một cơ sở dữ liệu khổng lồ dùng chung. Thay vào đó, mỗi service sở hữu một lược đồ dữ liệu (schema) riêng biệt, tối ưu cho nghiệp vụ của nó.

\noindent\textbf{Chú thích trạng thái ERD:}
\begin{itemize}[leftmargin=0.7cm]
    \item \textbf{[MVP]}: Đã triển khai trong phiên bản hiện tại
    \item \textbf{[Target]}: Kế hoạch cho kiến trúc đích, chưa triển khai
    \item \textbf{[MVP + Target]}: Một phần đã triển khai, phần còn lại là kế hoạch
\end{itemize}

\noindent\textbf{Tổng quan MVP Database Schema:}
\indentpar MVP hiện tại triển khai 3 bảng chính phục vụ luồng học tập thích ứng:
\begin{itemize}[leftmargin=0.7cm]
    \item \texttt{questions} (Content Service): Lưu trữ câu hỏi với JSONB options
    \item \texttt{submissions} (Scoring Service): Ghi nhận bài nộp và điểm số
    \item \texttt{skill\_mastery} (Learner Model): Theo dõi mức độ thông thạo kỹ năng
\end{itemize}

\noindent\textbf{1. User Management Service (PostgreSQL) -- [Target Architecture]}
\indentpar Quản lý người dùng, vai trò (Roles) và quyền hạn (Permissions). Dữ liệu PII (Thông tin cá nhân) được tách riêng vào bảng \texttt{learner\_profiles} và được mã hóa để đảm bảo tuân thủ bảo mật (ADR-7).

\textit{Lưu ý: Service này thuộc kiến trúc đích (Target Architecture), chưa được triển khai trong phiên bản MVP hiện tại. Trong MVP, user ID được xử lý đơn giản hóa để tập trung vào luồng học tập thích ứng.}

\begin{figure}[H]
    \centering
    \includegraphics[width=0.8\textwidth]{images/erd_user_service.png}
    \caption{ERD của User Management Service [Target Architecture -- Planned]}
    \label{fig:erd-user-service}
\end{figure}

\noindent\textbf{2. Content Service (PostgreSQL + JSONB) -- [MVP + Target]}
\indentpar Quản lý cấu trúc khóa học phong phú. Sử dụng tính năng \texttt{JSONB} của PostgreSQL cho bảng \texttt{content\_units} và \texttt{questions} để lưu trữ dữ liệu bán cấu trúc (như danh sách đáp án, nội dung bài học đa phương tiện) một cách linh hoạt mà không cần quá nhiều bảng phụ (ADR-2).

\textit{Trạng thái triển khai: MVP hiện tại chỉ triển khai bảng \texttt{questions} với cột JSONB cho options. Các bảng courses, chapters, content\_units, metadata\_tags thuộc kiến trúc đích.}

\begin{figure}[H]
    \centering
    \includegraphics[width=0.9\textwidth]{images/erd_content_service.png}
    \caption{ERD của Content Service [MVP: questions table, Target: full hierarchy]}
    \label{fig:erd-content-service}
\end{figure}

\noindent\textbf{3. Learner Model Service (PostgreSQL) -- [MVP + Target]}
\indentpar Tập trung vào dữ liệu phân tích (Analytics). Bảng \texttt{skill\_mastery} lưu trữ điểm số thông thạo hiện tại, trong khi \texttt{learning\_history} là một nhật ký (log) ghi lại toàn bộ quá trình làm bài của người học, phục vụ cho việc tính toán và truy vết sau này.

\textit{Trạng thái triển khai: MVP hiện tại triển khai bảng \texttt{skill\_mastery} với composite primary key (user\_id, skill\_tag). Các bảng learning\_history và diagnostic\_results thuộc kiến trúc đích.}

\begin{figure}[H]
    \centering
    \includegraphics[width=0.8\textwidth]{images/erd_learner_model_service.png}
    \caption{ERD của Learner Model Service [MVP: skill\_mastery table, Target: full history]}
    \label{fig:erd-learner-model-service}
\end{figure}