\subsection{Lựa chọn phong cách kiến trúc}

\indentpar \indentpar Sau khi xác định rõ các đặc tính kiến trúc ưu tiên (ACs), bước tiếp theo là lựa chọn phong cách kiến trúc (Architecture Style) phù hợp để hiện thực hóa các ACs đó trong thực tế. Việc chọn kiến trúc phù hợp giúp đảm bảo hệ thống ITS (Intelligent Tutoring System) không chỉ đáp ứng đúng yêu cầu nghiệp vụ mà còn duy trì được tính mô-đun, khả năng mở rộng và độ tin cậy dài hạn.

\subsubsection{Tiêu chí đánh giá}

\indentpar \indentpar Để đảm bảo quá trình lựa chọn có căn cứ, các kiến trúc được đánh giá theo bộ tiêu chí chuẩn hóa, dựa trên mối liên hệ giữa Architecture Characteristics (ACs) và bối cảnh kỹ thuật của dự án. Những tiêu chí này đóng vai trò làm ``thước đo'' khách quan để so sánh các phong cách kiến trúc khác nhau.

\begin{enumerate}[leftmargin=0.7cm]
    \item \textbf{Mức độ đáp ứng các ACs chính}: đánh giá khả năng hiện thực hóa bốn AC quan trọng nhất:
    \begin{itemize}[nosep]
        \item AC1: Modularity -- hỗ trợ hoán đổi mô hình AI (Live AI Model Swapping -- FR9, FR12).
        \item AC2: Scalability -- mở rộng tới hơn $5{,}000$ người dùng đồng thời.
        \item AC3: Performance -- duy trì khả năng phản hồi thời gian thực (FR6).
        \item AC4: Testability -- đảm bảo tính chính xác của các thuật toán AI.
    \end{itemize}
    \item \textbf{Độ phức tạp kỹ thuật}: chi phí vận hành, khả năng quan sát (observability) và độ phức tạp phát triển -- yếu tố trade-off trực tiếp với Scalability và Modularity.
    \item \textbf{Năng lực đội ngũ}: khả năng DevOps, kinh nghiệm với Kubernetes, CI/CD, monitoring. Kiến trúc mạnh nhưng vượt quá năng lực đội ngũ sẽ tiềm ẩn rủi ro.
    \item \textbf{Tác động chi phí}: tổng chi phí sở hữu (TCO) gồm chi phí hạ tầng ban đầu và chi phí mở rộng dài hạn.
\end{enumerate}

Mục tiêu không phải là tìm kiến trúc ``hoàn hảo'', mà là kiến trúc ``ít tệ nhất'' -- đạt được cân bằng giữa hiệu năng, khả năng mở rộng và chi phí.

\renewcommand{\arraystretch}{1.3}
\begin{table}[H]
\centering
\small
\begin{tabularx}{\textwidth}{|>{\centering\arraybackslash}p{2.8cm}|>{\centering\arraybackslash}p{2cm}|>{\centering\arraybackslash}p{2cm}|>{\centering\arraybackslash}p{2cm}|>{\centering\arraybackslash}p{2cm}|>{\centering\arraybackslash}p{2cm}|>{\centering\arraybackslash}X|}
\hline
\textbf{Style} & \textbf{AC1} & \textbf{AC2} & \textbf{AC3} & \textbf{Complexity} & \textbf{Cost} & \textbf{Score (Avg)} \\
\hline
Layered (Monolith) & 1 sao & 1 sao & 4 sao & 1 sao & 1\$ & 1.75 / 5.0 \\
\hline
Modular Monolith & 3 sao & 2 sao & 4 sao & 2 sao & 1\$ & 2.75 / 5.0 \\
\hline
Microkernel & 4 sao & 2 sao & 2 sao & 2 sao & 1\$ & 2.5 / 5.0 \\
\hline
\textbf{Microservices} & 5 sao & 5 sao & 4 sao & 4 sao & 5\$ & \textbf{4.5 / 5.0} \\
\hline
Service-based & 4 sao & 4 sao & 2 sao & 3 sao & 2\$ & 3.25 / 5.0 \\
\hline
Service-oriented & 1 sao & 3 sao & 2 sao & 5 sao & 5\$ & 2.75 / 5.0 \\
\hline
\textbf{Event-driven} & 3 sao & 5 sao & 5 sao & 4 sao & 2\$ & \textbf{4.25 / 5.0} \\
\hline
Space-based & 4 sao & 5 sao & 4 sao & 5 sao & 5\$ & 4.5 / 5.0 \\
\hline
\end{tabularx}
\caption{So sánh các phong cách kiến trúc ứng với các tiêu chí chính}
\label{tab:architecture-style-comparison}
\end{table}
\renewcommand{\arraystretch}{1.0}

Phân tích kết quả và định hướng lựa chọn:
\begin{itemize}[leftmargin=0.7cm]
    \item Layered (Monolith): loại bỏ vì không đáp ứng Modularity và Scalability.
    \item Modular Monolith: phù hợp làm bước đệm/MVP nhờ chi phí thấp, dễ bảo trì.
    \item Microservices: đạt điểm cao nhất nhưng chi phí và độ phức tạp lớn -- phù hợp cho ITS.
    \item Event-driven: bổ trợ lý tưởng cho Microservices, mạnh về phản hồi (performance/responsiveness).
    \item Space-based: điểm cao nhưng quá phức tạp và dư thừa cho bối cảnh ITS.
\end{itemize}

\textbf{Sau phân tích, Microservices và Event-driven là hai phong cách vừa đạt hiệu năng cao vừa hỗ trợ xử lý real-time -- nền tảng cho kiến trúc lai (hybrid) của ITS.}

\subsubsection{Quyết định kiến trúc cuối cùng}

\indentpar \indentpar Dựa trên toàn bộ quá trình đánh giá, kiến trúc cuối cùng cho ITS là \textbf{Hybrid Microservices + Event-Driven}. Kiến trúc này cân bằng tính mô-đun, khả năng mở rộng, hiệu năng thời gian thực và khả thi với đội ngũ hiện tại.

\begin{itemize}[leftmargin=0.7cm]
    \item Microservices Architecture: nền tảng chính, phân chia theo miền nghiệp vụ (Domain-Driven Design).
    \item Event-Driven Architecture (EDA): cơ chế giao tiếp bất đồng bộ, xử lý phản hồi real-time và học thích ứng.
\end{itemize}

\subsubsection{Lý do lựa chọn}
\begin{enumerate}[leftmargin=0.7cm]
    \item Tối ưu các ACs cốt lõi:
    \begin{itemize}[nosep]
        \item Modularity (AC1) \\ Deployability (AC5): hỗ trợ Live AI Model Swapping (FR9, FR12).
        \item Scalability (AC2): mở rộng từng service độc lập, tránh bottleneck.
        \item Testability (AC4): kiểm thử độc lập nhờ Clean Architecture.
    \end{itemize}
    \item Đáp ứng yêu cầu thời gian thực:
    \begin{itemize}[nosep]
        \item Event-driven xử lý đồng thời SubmissionCompleted và cập nhật \texttt{LearnerModel} không gây nghẽn.
        \item Cải thiện Performance/Responsiveness (AC3) và fault tolerance.
    \end{itemize}
    \item Quản lý rủi ro: nhận diện độ phức tạp cao và có chiến lược giảm thiểu thông qua lộ trình phát triển.
\end{enumerate}

\subsubsection{Chiến lược triển khai}

\indentpar \indentpar ITS không chuyển sang Microservices ngay lập tức mà từng bước để giảm rủi ro:

\begin{enumerate}[leftmargin=0.7cm]
    \item \textbf{Phase 1 -- Modular Monolith (MVP):} xây dựng monolith có cấu trúc module, áp dụng Clean Architecture \& DIP để thuận lợi tách rời.
    \item \textbf{Phase 2 -- Extract Critical Services (Strangler Fig Pattern):} khi nhu cầu mở rộng xuất hiện (ví dụ ScoringService quá tải), tách module thành microservice riêng và định tuyến qua API Gateway.
    \item \textbf{Phase 3 -- Full Microservices Ecosystem:} tiếp tục tách các module còn lại khi hệ thống trưởng thành, hướng tới hệ sinh thái microservices tối ưu AC1, AC2, AC5.
\end{enumerate}
