\subsection{Lựa chọn phong cách kiến trúc}

\indentpar \indentpar Sau khi xác định rõ các đặc tính kiến trúc ưu tiên (ACs), bước tiếp theo là lựa chọn phong cách kiến trúc (Architecture Style) phù hợp để hiện thực hóa các ACs đó trong thực tế. Việc chọn kiến trúc phù hợp giúp đảm bảo hệ thống ITS (Intelligent Tutoring System) không chỉ đáp ứng đúng yêu cầu nghiệp vụ mà còn duy trì được tính mô-đun, khả năng mở rộng và độ tin cậy dài hạn.

\subsubsection{Tiêu chí đánh giá}

\indentpar \indentpar Để đảm bảo quá trình lựa chọn có căn cứ, các kiến trúc được đánh giá theo bộ tiêu chí chuẩn hóa, dựa trên mối liên hệ giữa Architecture Characteristics (ACs) và bối cảnh kỹ thuật của dự án. Những tiêu chí này đóng vai trò làm ``thước đo'' khách quan để so sánh các phong cách kiến trúc khác nhau.

\begin{enumerate}[leftmargin=0.7cm]
    \item Mức độ đáp ứng các ACs chính: đánh giá khả năng hiện thực hóa bốn AC quan trọng nhất:
          \begin{itemize}[nosep]
              \item AC1: Modularity -- hỗ trợ hoán đổi mô hình AI (Live AI Model Swapping -- FR9, FR12).
              \item AC2: Scalability -- mở rộng tới hơn $5{,}000$ người dùng đồng thời.
              \item AC3: Performance -- duy trì khả năng phản hồi thời gian thực (FR6).
              \item AC4: Testability -- đảm bảo tính chính xác của các thuật toán AI.
          \end{itemize}
    \item Độ phức tạp kỹ thuật: chi phí vận hành, khả năng quan sát (observability) và độ phức tạp phát triển -- yếu tố trade-off trực tiếp với Scalability và Modularity.
    \item Năng lực đội ngũ: khả năng DevOps, kinh nghiệm với Kubernetes, CI/CD, monitoring. Kiến trúc mạnh nhưng vượt quá năng lực đội ngũ sẽ tiềm ẩn rủi ro.
    \item Tác động chi phí: tổng chi phí sở hữu (TCO) gồm chi phí hạ tầng ban đầu và chi phí mở rộng dài hạn.
\end{enumerate}

Mục tiêu không phải là tìm kiến trúc ``hoàn hảo'', mà là kiến trúc ``ít tệ nhất'' -- đạt được cân bằng giữa hiệu năng, khả năng mở rộng và chi phí.

\renewcommand{\arraystretch}{1.3}
\begin{table}[H]
    \centering
    \small
    \begin{tabularx}{\textwidth}{|>{\centering\arraybackslash}p{2.8cm}|>{\centering\arraybackslash}p{2cm}|>{\centering\arraybackslash}p{2cm}|>{\centering\arraybackslash}p{2cm}|>{\centering\arraybackslash}p{2cm}|>{\centering\arraybackslash}p{2cm}|>{\centering\arraybackslash}X|}
        \hline
        \textbf{Style}         & \textbf{AC1} & \textbf{AC2} & \textbf{AC3} & \textbf{Complexity} & \textbf{Cost} & \textbf{Score (Avg)} \\
        \hline
        Layered (Monolith)     & 1 sao        & 1 sao        & 4 sao        & 1 sao               & 1\$           & 1.75 / 5.0           \\
        \hline
        Modular Monolith       & 3 sao        & 2 sao        & 4 sao        & 2 sao               & 1\$           & 2.75 / 5.0           \\
        \hline
        Microkernel            & 4 sao        & 2 sao        & 2 sao        & 2 sao               & 1\$           & 2.5 / 5.0            \\
        \hline
        Microservices          & 5 sao        & 5 sao        & 4 sao        & 4 sao               & 5\$           & 4.5 / 5.0            \\
        \hline
        Service-based          & 4 sao        & 4 sao        & 2 sao        & 3 sao               & 2\$           & 3.25 / 5.0           \\
        \hline
        Service-oriented       & 1 sao        & 3 sao        & 2 sao        & 5 sao               & 5\$           & 2.75 / 5.0           \\
        \hline
        Event-driven           & 3 sao        & 5 sao        & 5 sao        & 4 sao               & 2\$           & 4.25 / 5.0           \\
        \hline
        Space-based            & 4 sao        & 5 sao        & 4 sao        & 5 sao               & 5\$           & 4.5 / 5.0            \\
        \hline
    \end{tabularx}
    \caption{So sánh các phong cách kiến trúc ứng với các tiêu chí chính}
    \label{tab:architecture-style-comparison}
\end{table}
\renewcommand{\arraystretch}{1.0}

Phân tích kết quả và định hướng lựa chọn:
\begin{itemize}[leftmargin=0.7cm]
    \item Layered (Monolith): loại bỏ vì không đáp ứng Modularity và Scalability.
    \item Modular Monolith: phù hợp làm bước đệm/MVP nhờ chi phí thấp, dễ bảo trì.
    \item Microservices: đạt điểm cao nhất nhưng chi phí và độ phức tạp lớn -- phù hợp cho ITS.
    \item Event-driven: bổ trợ lý tưởng cho Microservices, mạnh về phản hồi (performance/responsiveness).
    \item Space-based: điểm cao nhưng quá phức tạp và dư thừa cho bối cảnh ITS.
\end{itemize}

Sau phân tích, Microservices và Event-driven là hai phong cách vừa đạt hiệu năng cao vừa hỗ trợ xử lý real-time -- nền tảng cho kiến trúc lai (hybrid) của ITS.

\subsubsection{Quyết định kiến trúc cuối cùng}

\indentpar \indentpar Dựa trên toàn bộ quá trình đánh giá, kiến trúc cuối cùng cho ITS là Hybrid Microservices + Event-Driven. Kiến trúc này cân bằng tính mô-đun, khả năng mở rộng, hiệu năng thời gian thực và khả thi với đội ngũ hiện tại.

\begin{itemize}[leftmargin=0.7cm]
    \item Microservices Architecture: nền tảng chính, phân chia theo miền nghiệp vụ (Domain-Driven Design).
    \item Event-Driven Architecture (EDA): cơ chế giao tiếp bất đồng bộ, xử lý phản hồi real-time và học thích ứng.
\end{itemize}

\subsubsection{Lý do lựa chọn}
\begin{enumerate}[leftmargin=0.7cm]
    \item Tối ưu các ACs cốt lõi:
          \begin{itemize}[nosep]
              \item Modularity (AC1) và Deployability (AC5): hỗ trợ Live AI Model Swapping (FR9, FR12).
              \item Scalability (AC2): mở rộng từng service độc lập, tránh bottleneck.
              \item Testability (AC4): kiểm thử độc lập nhờ Clean Architecture.
          \end{itemize}
    \item Đáp ứng yêu cầu thời gian thực:
          \begin{itemize}[nosep]
              \item Event-driven xử lý đồng thời SubmissionCompleted và cập nhật \texttt{LearnerModel} không gây nghẽn.
              \item Cải thiện Performance/Responsiveness (AC3) và fault tolerance.
          \end{itemize}
    \item Quản lý rủi ro: nhận diện độ phức tạp cao và có chiến lược giảm thiểu thông qua lộ trình phát triển.
\end{enumerate}

\subsubsection{Chiến lược triển khai}

\indentpar \indentpar ITS không chuyển sang Microservices ngay lập tức mà từng bước để giảm rủi ro:

\begin{enumerate}[leftmargin=0.7cm]
    \item Phase 1 -- MVP (Hiện tại): Xây dựng hệ thống với các service cốt lõi (Content, Scoring, Learner, Adaptive) chạy trên nền tảng Docker Compose. Sử dụng giao tiếp REST để đơn giản hóa việc tích hợp và kiểm thử nhanh logic nghiệp vụ.
    \item Phase 2 -- Modular Monolith \& Optimization: Tối ưu hóa code nội bộ theo Clean Architecture, chuẩn bị cho việc mở rộng.
    \item Phase 2 -- Extract Critical Services (Strangler Fig Pattern): khi nhu cầu mở rộng xuất hiện (ví dụ ScoringService quá tải), tách module thành microservice riêng và định tuyến qua API Gateway. Strangler Fig Pattern là mô hình di chuyển dần từ hệ thống cũ sang mới, trong đó các chức năng được tách ra từng phần một, đồng thời hệ thống cũ vẫn hoạt động song song cho đến khi hoàn toàn thay thế.
    \item Phase 3 -- Full Microservices Ecosystem: tiếp tục tách các module còn lại khi hệ thống trưởng thành, hướng tới hệ sinh thái microservices tối ưu AC1, AC2, AC5.
\end{enumerate}

\subsubsection{Phân tích Chi phí -- Lợi ích (TCO)}

\indentpar \indentpar Để đánh giá tính khả thi về mặt kinh tế của quyết định kiến trúc, bảng dưới đây so sánh Tổng Chi phí Sở hữu (Total Cost of Ownership -- TCO) giữa kiến trúc Monolith truyền thống và Microservices được lựa chọn cho ITS trong chu kỳ $3$ năm.

\renewcommand{\arraystretch}{1.4}
\begin{table}[H]
    \centering
    \small
    \begin{tabularx}{\textwidth}{|>{\noindent\justifying\arraybackslash}p{3.5cm}|>{\centering\arraybackslash}p{2.8cm}|>{\centering\arraybackslash}p{2.8cm}|>{\noindent\justifying\arraybackslash}X|}
        \hline
        \textbf{Hạng mục Chi phí} & \textbf{Monolith} & \textbf{Microservices} & \textbf{Ghi chú}                                                 \\
        \hline
        \multicolumn{4}{|c|}{\textbf{Chi phí Phát triển Ban đầu (Year 0)}}                                                                        \\
        \hline
        Thiết kế kiến trúc        & \$5,000           & \$15,000               & MS cần thiết kế domain boundaries, API contracts, event schemas. \\
        \hline
        Phát triển core features  & \$40,000          & \$60,000               & MS tăng $50\%$ do boilerplate, inter-service communication.      \\
        \hline
        CI/CD setup               & \$2,000           & \$8,000                & MS cần pipeline riêng cho mỗi service, container registry.       \\
        \hline
        Testing infrastructure    & \$3,000           & \$10,000               & MS cần integration tests, contract tests, Testcontainers.        \\
        \hline
        \textbf{Subtotal Year 0}  & \textbf{\$50,000} & \textbf{\$93,000}      & MS cao hơn $86\%$ ban đầu.                                       \\
        \hline
        \multicolumn{4}{|c|}{\textbf{Chi phí Hạ tầng Hàng năm}}                                                                                   \\
        \hline
        Compute (VMs/Containers)  & \$6,000/năm       & \$12,000/năm           & MS cần nhiều container instances hơn.                            \\
        \hline
        Database (PostgreSQL)     & \$3,600/năm       & \$7,200/năm            & MS: $3$ DB instances vs $1$ cho Monolith.                        \\
        \hline
        Message Broker (RabbitMQ) & \$0               & \$2,400/năm            & Chỉ MS cần message broker.                                       \\
        \hline
        Monitoring/Logging        & \$1,200/năm       & \$3,600/năm            & MS cần distributed tracing, centralized logging.                 \\
        \hline
        \textbf{Subtotal/năm}     & \textbf{\$10,800} & \textbf{\$25,200}      & MS cao hơn $133\%$ hạ tầng.                                      \\
        \hline
        \multicolumn{4}{|c|}{\textbf{Chi phí Bảo trì Hàng năm}}                                                                                   \\
        \hline
        Bug fixes \& patches      & \$8,000/năm       & \$6,000/năm            & MS dễ isolate và fix bugs hơn.                                   \\
        \hline
        Feature development       & \$20,000/năm      & \$15,000/năm           & MS cho phép parallel development.                                \\
        \hline
        DevOps/Operations         & \$5,000/năm       & \$12,000/năm           & MS cần kỹ năng K8s, monitoring cao hơn.                          \\
        \hline
        \textbf{Subtotal/năm}     & \textbf{\$33,000} & \textbf{\$33,000}      & Tương đương về bảo trì.                                          \\
        \hline
        \multicolumn{4}{|c|}{\textbf{Chi phí Mở rộng (Scaling) -- Khi đạt 5,000 users}}                                                           \\
        \hline
        Horizontal scaling        & \$15,000          & \$5,000                & MS scale từng service, Monolith scale toàn bộ.                   \\
        \hline
        Database sharding         & \$20,000          & \$8,000                & MS đã có DB riêng, dễ shard hơn.                                 \\
        \hline
        Refactoring effort        & \$30,000          & \$0                    & Monolith cần refactor lớn để scale.                              \\
        \hline
        \textbf{Subtotal Scaling} & \textbf{\$65,000} & \textbf{\$13,000}      & MS tiết kiệm $80\%$ khi scale.                                   \\
        \hline
    \end{tabularx}
    \caption{So sánh TCO: Monolith vs Microservices (chu kỳ 3 năm)}
    \label{tab:tco_comparison}
\end{table}
\renewcommand{\arraystretch}{1.0}

\noindent Tổng hợp TCO 3 năm:
\begin{itemize}[leftmargin=0.7cm]
    \item Monolith: \$50,000 + (\$10,800 + \$33,000) $\times$ 3 + \$65,000 = \$246,400
    \item Microservices: \$93,000 + (\$25,200 + \$33,000) $\times$ 3 + \$13,000 = \$280,600
\end{itemize}

\noindent Phân tích Lợi ích -- Chi phí:
\begin{enumerate}[leftmargin=0.7cm]
    \item Chi phí cao hơn $14\%$: Microservices tốn thêm $\sim$\$34,200 trong 3 năm so với Monolith.
    \item Lợi ích vô hình: Microservices mang lại giá trị khó định lượng:
          \begin{itemize}[nosep]
              \item Time-to-market: Parallel development giảm $30\%$ thời gian release features mới.
              \item Fault isolation: Lỗi một service không ảnh hưởng toàn hệ thống.
              \item Technology flexibility: Có thể dùng ngôn ngữ/framework tối ưu cho từng service.
              \item Team autonomy: Các team có thể làm việc độc lập, tăng productivity.
          \end{itemize}
    \item Break-even point: Khi hệ thống cần scale lên $>10,000$ users, Microservices trở nên rẻ hơn do chi phí scaling thấp hơn $80\%$.
    \item Kết luận: Chi phí tăng thêm $14\%$ là chấp nhận được để đạt được Modularity (AC1), Scalability (AC2), và khả năng Live AI Model Swapping (FR12) -- các yêu cầu cốt lõi của ITS.
\end{enumerate}

\FloatBarrier

\subsubsection{Ma trận Rủi ro Kiến trúc}

\indentpar \indentpar Việc lựa chọn kiến trúc Hybrid Microservices + Event-Driven mang lại nhiều lợi ích nhưng cũng đi kèm các rủi ro cần được nhận diện và quản lý. Bảng dưới đây trình bày ma trận rủi ro với đánh giá xác suất, tác động và chiến lược giảm thiểu cho từng rủi ro.

\renewcommand{\arraystretch}{1.4}
\begin{longtable}{|>{\centering\arraybackslash}p{0.6cm}|>{\noindent\justifying\arraybackslash}p{3.8cm}|>{\centering\arraybackslash}p{1.4cm}|>{\centering\arraybackslash}p{1.4cm}|>{\noindent\justifying\arraybackslash}p{4.5cm}|>{\centering\arraybackslash}p{1.8cm}|}
    \caption{Ma trận Rủi ro Kiến trúc}
    \label{tab:risk_matrix}
    \\
    \hline
    \textbf{ID} & \textbf{Mô tả Rủi ro}                                                                                                                     & \textbf{Xác suất} & \textbf{Tác động} & \textbf{Chiến lược Giảm thiểu}                                                                                                    & \textbf{Trạng thái} \\
    \hline
    \endfirsthead
    \caption[]{Ma trận Rủi ro Kiến trúc (tiếp theo)}
    \\
    \hline
    \textbf{ID} & \textbf{Mô tả Rủi ro}                                                                                                                     & \textbf{Xác suất} & \textbf{Tác động} & \textbf{Chiến lược Giảm thiểu}                                                                                                    & \textbf{Trạng thái} \\
    \hline
    \endhead
    \hline
    \endfoot
    \hline
    \endlastfoot
    R1          & Độ phức tạp vận hành cao: Microservices yêu cầu kỹ năng DevOps, Kubernetes, monitoring vượt quá năng lực đội ngũ hiện tại.       & Cao               & Cao               & Triển khai theo giai đoạn (MVP $\rightarrow$ Full MS). Đào tạo đội ngũ. Sử dụng Docker Compose cho MVP trước khi chuyển sang K8s. & Đang giảm thiểu     \\
    \hline
    R2          & Không nhất quán dữ liệu phân tán: Eventual consistency giữa các service có thể gây ra trạng thái không đồng bộ tạm thời.         & Trung bình        & Cao               & Áp dụng Saga Pattern (ADR-9). Thiết kế idempotent consumers. Sử dụng Outbox Pattern cho event publishing.                         & Đang giảm thiểu     \\
    \hline
    R3          & Single Point of Failure: Auth Service và API Gateway là điểm lỗi đơn, sự cố gây downtime toàn hệ thống.                          & Trung bình        & Cao               & Triển khai HA với nhiều replica trên Kubernetes. Health checks và auto-restart. Circuit breaker pattern.                          & Kế hoạch            \\
    \hline
    R4          & Chi phí hạ tầng cao: Microservices yêu cầu nhiều tài nguyên hơn monolith (nhiều container, DB instances, message broker).        & Cao               & Trung bình        & Bắt đầu với MVP tối thiểu. Sử dụng shared database cho giai đoạn đầu. Tối ưu resource limits. Auto-scaling theo nhu cầu.          & Chấp nhận           \\
    \hline
    R5          & Polyglot complexity: Vận hành hai hệ sinh thái (Java + Go) tăng độ phức tạp CI/CD, debugging và onboarding.                      & Trung bình        & Trung bình        & Chuẩn hóa cấu trúc dự án. Chia sẻ template CI/CD. Thống nhất chuẩn logging/monitoring. Đào tạo chéo đội ngũ.                      & Đang giảm thiểu     \\
    \hline
    R6          & Network latency: Giao tiếp giữa các service qua mạng tăng độ trễ so với monolith, ảnh hưởng AC3 (Performance).                   & Trung bình        & Trung bình        & Sử dụng async messaging (RabbitMQ) cho non-critical paths. Caching với Redis. Service mesh cho internal traffic.                  & Đang giảm thiểu     \\
    \hline
    R7          & Message broker failure: RabbitMQ sự cố gây mất event, ảnh hưởng luồng Scoring $\rightarrow$ LearnerModel $\rightarrow$ Adaptive. & Thấp              & Cao               & Triển khai RabbitMQ cluster với mirrored queues. Persistent messages. Dead letter queues cho retry. Fallback sync API.            & Kế hoạch            \\
    \hline
    R8          & Security breach: Tấn công nội bộ bypass Gateway, giả mạo header X-User-ID/X-User-Roles.                                          & Thấp              & Cao               & Sử dụng VPC riêng và Network Policies. Chỉ Gateway gọi được service nội bộ. JWT với TTL ngắn. Audit logging.                      & Kế hoạch            \\
    \hline
    R9          & AI Model deployment failure: Triển khai phiên bản AI mới gây lỗi, ảnh hưởng trải nghiệm học sinh.                                & Trung bình        & Cao               & Blue/Green deployment. Canary releases với $10\%$ traffic. Rollback tự động trong $5$ phút. A/B testing trước production.         & Đang giảm thiểu     \\
    \hline
    R10         & Database performance degradation: PostgreSQL chậm khi tải nặng hoặc truy vấn chưa tối ưu.                                        & Trung bình        & Trung bình        & Read replicas. Connection pooling (PgBouncer). Index optimization. Slow query monitoring. Vertical scaling khi cần.               & Đang giảm thiểu     \\
    \hline
\end{longtable}
\renewcommand{\arraystretch}{1.0}

\noindent Ghi chú về mức độ đánh giá:
\begin{itemize}[leftmargin=0.7cm]
    \item Xác suất: Thấp ($< 20\%$), Trung bình ($20\%$--$50\%$), Cao ($> 50\%$)
    \item Tác động: Thấp (ảnh hưởng nhỏ, dễ khắc phục), Trung bình (ảnh hưởng một phần hệ thống), Cao (ảnh hưởng nghiêm trọng, downtime hoặc mất dữ liệu)
    \item Trạng thái: Đang giảm thiểu (đã có biện pháp), Kế hoạch (sẽ triển khai), Chấp nhận (rủi ro được chấp nhận)
\end{itemize}

\FloatBarrier

\vspace{0.5em}
\noindent Kết luận: Quyết định kiến trúc lai này đóng vai trò là nền tảng cho toàn bộ các quyết định kỹ thuật chi tiết được trình bày trong Mục~3.3 (Architecture Decision Records) và Mục~3.4 (Design Principles).
