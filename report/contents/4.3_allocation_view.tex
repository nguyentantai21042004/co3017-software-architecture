\subsection{Góc nhìn Phân bổ}

\indentpar \indentpar \textbf{Lưu ý quan trọng:} Góc nhìn này mô tả \textbf{Kiến trúc Đích (Target Architecture - Phase 3)} của hệ thống ITS khi triển khai trên môi trường Production quy mô lớn (On-Premise Kubernetes).
\indentpar Đối với giai đoạn \textbf{MVP (Phase 1)} hiện tại, hệ thống được triển khai đơn giản hóa bằng \textbf{Docker Compose} trên một máy chủ vật lý duy nhất để phục vụ phát triển và kiểm thử nhanh (xem chi tiết tại Mục~6).

\indentpar \indentpar Trong kiến trúc đích On-Premise, hệ thống ITS được triển khai hoàn toàn trên hạ tầng nội bộ (private infrastructure), bao gồm:
\begin{itemize}[leftmargin=0.7cm]
    \item Các Server vật lý hoặc Virtual Machines trong trung tâm dữ liệu.
    \item Một Kubernetes Cluster tự quản lý (Self-managed K8s) do đội IT vận hành.
    \item Các thành phần lưu trữ, mạng và giám sát được triển khai cục bộ.
\end{itemize}

\indentpar \indentpar Góc nhìn này ánh xạ các thành phần kiến trúc (Service Components) từ Mục~4.2 vào các tài nguyên vật lý và ảo tại trung tâm dữ liệu On-Premise.

\indentpar \indentpar Liên kết với phần trước:
\begin{itemize}[leftmargin=0.7cm]
    \item Mục~4.1 Module View → cho biết cấu trúc code.
    \item Mục~4.2 Component \& Connector View → cho biết các thành phần và cách chúng giao tiếp.
    \item Mục~4.3 Allocation View → mô tả chúng chạy trên máy chủ/vùng mạng nào, với cơ chế HA và DR ra sao trong môi trường On-Premise.
\end{itemize}

\indentpar \indentpar Mục tiêu của Deployment View trong On-Premise:
\begin{itemize}[leftmargin=0.7cm]
    \item \textbf{AC2 -- Scalability:} Scale-out bằng Node Pools và tự động phân phối workload.
    \item \textbf{AC5 -- Deployability:} Tự động hóa CI/CD nội bộ, sử dụng Container Registry riêng.
    \item \textbf{AC9 -- Observability:} Quản lý logs/metrics phân tán trong hạ tầng nội bộ.
    \item \textbf{High Availability \& Fault Tolerance:} Đảm bảo hệ thống chạy ổn định ngay cả khi một node hoặc tủ rack gặp sự cố.
\end{itemize}

\subsubsection{Kiến trúc Triển khai (On-Premise Architecture)}

\indentpar \indentpar Kiến trúc triển khai On-Premise vẫn tuân thủ nguyên tắc Cloud-Native, nhưng tất cả thành phần được host trên:
\begin{itemize}[leftmargin=0.7cm]
    \item Máy chủ vật lý (Bare-metal servers) với CPU/RAM cao.
    \item Hệ thống lưu trữ SAN/NAS cho database và persistent volumes.
    \item Hệ thống ảo hóa nội bộ (VMware vSphere / Proxmox / OpenStack) cho Kubernetes nodes.
    \item Load Balancer cứng (hardware LB) hoặc HAProxy/NGINX nội bộ.
    \item Private Docker Registry để lưu images.
\end{itemize}

\noindent \textbf{Hạ tầng phần cứng (Physical Infrastructure Layer)}

\renewcommand{\arraystretch}{1.3}
\begin{table}[H]
    \centering
    \small
    \begin{tabularx}{\textwidth}{|>{\raggedright\arraybackslash}p{3.5cm}|>{\raggedright\arraybackslash}X|}
        \hline
        \textbf{Thành phần}\rule{0pt}{1.6em} & \textbf{Mô tả}                                               \\
        \hline
        Rack Servers                         & 6--10 server vật lý, CPU Intel Xeon/AMD EPYC, RAM 128--256GB \\
        \hline
        Network                              & Switch Layer 2/3, VLAN tách biệt (DMZ / Internal / Storage)  \\
        \hline
        Storage                              & SAN/NAS hỗ trợ RAID-10, Fiber Channel/iSCSI                  \\
        \hline
        Management                           & vCenter/Proxmox Dashboard cho quản lý VM                     \\
        \hline
    \end{tabularx}
    \caption{Hạ tầng phần cứng On-Premise}
    \label{tab:physical-infrastructure}
\end{table}
\renewcommand{\arraystretch}{1.0}

\noindent \textbf{Kubernetes Cluster (Self-Managed)}

\indentpar \indentpar Hệ thống chạy trên một cluster Kubernetes tự triển khai:

\renewcommand{\arraystretch}{1.3}
\begin{table}[H]
    \centering
    \small
    \begin{tabularx}{\textwidth}{|>{\raggedright\arraybackslash}p{3.5cm}|>{\raggedright\arraybackslash}X|}
        \hline
        \textbf{Thành phần}\rule{0pt}{1.6em} & \textbf{Mô tả}                        \\
        \hline
        Master Nodes                         & 3 master để đảm bảo HA (etcd cluster) \\
        \hline
        Worker Nodes                         & 6--12 worker (tuỳ tải)                \\
        \hline
        CNI                                  & Calico hoặc Cilium                    \\
        \hline
        Ingress                              & NGINX Ingress hoặc Istio Gateway      \\
        \hline
        Persistent Volumes                   & Rook-Ceph / Longhorn / NFS            \\
        \hline
    \end{tabularx}
    \caption{Cấu hình Kubernetes Cluster}
    \label{tab:k8s-cluster}
\end{table}
\renewcommand{\arraystretch}{1.0}

\noindent \textbf{Phân vùng Mạng (Network Segmentation)}

\indentpar \indentpar Vì đây là môi trường On-Premise, việc phân vùng mạng (segmentation) là cực kỳ quan trọng:
\begin{itemize}[leftmargin=0.7cm]
    \item \textbf{DMZ Layer} → Load Balancer + API Gateway.
    \item \textbf{Application Layer} → các pod Java/Go.
    \item \textbf{Data Layer} → PostgreSQL, MongoDB, RabbitMQ, Redis.
    \item \textbf{Monitoring Layer} → Prometheus, Grafana, Loki.
\end{itemize}

\begin{figure}[H]
    \centering
    \includegraphics[width=\textwidth]{images/deployment_architecture_onprem.png}
    \caption{Kiến trúc Triển khai On-Premise}
    \label{fig:deployment-architecture-onprem}
\end{figure}

\subsubsection{Thông số Container (On-Premise Resource Planning)}

\indentpar \indentpar Do hệ thống chạy On-Premise, tài nguyên cần được quy hoạch tĩnh (static capacity planning), kết hợp với autoscaling dựa trên metrics nhưng bị ràng buộc bởi tài nguyên phần cứng.

\renewcommand{\arraystretch}{1.3}
\begin{table}[H]
    \centering
    \small
    \begin{tabularx}{\textwidth}{|>{\raggedright\arraybackslash}p{2.8cm}|>{\centering\arraybackslash}p{1.5cm}|>{\centering\arraybackslash}p{1.8cm}|>{\centering\arraybackslash}p{1.8cm}|>{\centering\arraybackslash}p{1.5cm}|>{\centering\arraybackslash}p{1.8cm}|}
        \hline
        \textbf{Service}\rule{0pt}{1.6em} & \textbf{Image Size} & \textbf{Memory Req/Lmt} & \textbf{CPU Req/Lmt} & \textbf{Replicas} & \textbf{Node Pool} \\
        \hline
        API Gateway (Go)                  & 30MB                & 256MB/512MB             & 0.25/1               & 3--10             & Node Pool A        \\
        \hline
        Auth Service (Java)               & 250MB               & 512MB/1GB               & 0.5/1.5              & 2--5              & Node Pool B        \\
        \hline
        User Service (Java)               & 250MB               & 512MB/1GB               & 0.5/1.5              & 2--5              & Node Pool B        \\
        \hline
        Content Service (Java)            & 250MB               & 512MB/1GB               & 0.5/1.5              & 2--5              & Node Pool B        \\
        \hline
        Scoring Svc (Go)                  & 40MB                & 256MB/1GB               & 0.5/2                & 3--15             & Node Pool A        \\
        \hline
        Learner Model (Go)                & 40MB                & 256MB/1GB               & 0.5/2                & 3--10             & Node Pool A        \\
        \hline
        Adaptive Engine (Go)              & 40MB                & 512MB/2GB               & 1/3                  & 2--8              & Node Pool A        \\
        \hline
    \end{tabularx}
    \caption{Thông số Container và Resource Planning}
    \label{tab:container-specs}
\end{table}
\renewcommand{\arraystretch}{1.0}

\subsubsection{Thành phần Hạ tầng (On-Premise Stack)}

\indentpar \indentpar Dưới đây là toàn bộ thành phần hạ tầng bắt buộc khi triển khai ITS On-Premise.

\noindent \textbf{1. Load Balancer (Lớp DMZ)}

\begin{itemize}[leftmargin=0.7cm]
    \item Có thể dùng:
          \begin{itemize}[nosep,leftmargin=0.9cm]
              \item HAProxy (active-passive)
              \item NGINX Plus
              \item F5 BIG-IP (nếu doanh nghiệp có sẵn)
          \end{itemize}
    \item Chức năng:
          \begin{itemize}[nosep,leftmargin=0.9cm]
              \item SSL termination
              \item Reverse proxy
              \item Rate limit
              \item WAF
          \end{itemize}
\end{itemize}

\noindent \textbf{2. Database \& Storage Zone}

\begin{itemize}[leftmargin=0.7cm]
    \item \textbf{PostgreSQL HA Cluster:}
          \begin{itemize}[nosep,leftmargin=0.9cm]
              \item Patroni + etcd hoặc
              \item pgPool-II + Streaming Replication
          \end{itemize}
    \item \textbf{MongoDB Replica Set:}
          \begin{itemize}[nosep,leftmargin=0.9cm]
              \item Chạy 3 node trong các rack khác nhau.
          \end{itemize}
    \item \textbf{Redis Sentinel Cluster:}
          \begin{itemize}[nosep,leftmargin=0.9cm]
              \item Dùng local SSD của server để đạt IOPS cao.
          \end{itemize}
    \item \textbf{RabbitMQ Cluster:}
          \begin{itemize}[nosep,leftmargin=0.9cm]
              \item Mirrored queues (Highly Available).
              \item Phù hợp với Adaptive Learning \& Scoring Event-driven flow.
          \end{itemize}
\end{itemize}

\noindent \textbf{3. Observability Stack (AC9 -- On-Premise)}

\indentpar \indentpar Toàn bộ stack giám sát được cài trong namespace \texttt{its-monitoring}:
\begin{itemize}[leftmargin=0.7cm]
    \item \textbf{Prometheus} -- metrics (scrape từ pods + node exporter)
    \item \textbf{Loki} -- collector logs
    \item \textbf{Grafana} -- dashboard hợp nhất
    \item \textbf{Alertmanager} -- cảnh báo nội bộ (email/SMS/Zalo OA)
    \item \textbf{OpenTelemetry Collector} -- gửi trace từ Java/Go đến Tempo/Jaeger
    \item \textbf{Node Exporter} -- metrics phần cứng của mỗi server
\end{itemize}

\subsubsection{So sánh Cloud vs On-Premise}

\renewcommand{\arraystretch}{1.3}
\begin{table}[H]
    \centering
    \small
    \begin{tabularx}{\textwidth}{|>{\raggedright\arraybackslash}p{4.5cm}|>{\raggedright\arraybackslash}X|}
        \hline
        \textbf{Cloud}\rule{0pt}{1.6em}            & \textbf{On-Premise}                          \\
        \hline
        Autoscaling theo tài nguyên không giới hạn & Scale bị giới hạn bởi số server vật lý       \\
        \hline
        Managed Databases                          & Tự quản lý DB HA (Patroni/Mongo Replica Set) \\
        \hline
        Load Balancer do Cloud cung cấp            & HAProxy/NGINX hoặc thiết bị phần cứng        \\
        \hline
        Observability Cloud Services               & Tự triển khai Prometheus + Loki + Grafana    \\
        \hline
        Storage quản lý bởi cloud                  & SAN/NAS + CSI Driver                         \\
        \hline
    \end{tabularx}
    \caption{So sánh Cloud và On-Premise}
    \label{tab:cloud-vs-onprem}
\end{table}
\renewcommand{\arraystretch}{1.0}
