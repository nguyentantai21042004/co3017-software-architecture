% 3.3 Allocation View

% Trong kiến trúc On-Premise, hệ thống ITS được triển khai hoàn toàn trên hạ tầng nội bộ (private infrastructure), bao gồm:
% 	•	Các Server vật lý hoặc Virtual Machines trong trung tâm dữ liệu.
% 	•	Một Kubernetes Cluster tự quản lý (Self-managed K8s) do đội IT vận hành.
% 	•	Các thành phần lưu trữ, mạng và giám sát được triển khai cục bộ.

% Góc nhìn này ánh xạ các thành phần kiến trúc (Service Components) từ Mục 3.2 vào các tài nguyên vật lý và ảo tại trung tâm dữ liệu On-Premise.

% 🔗 Liên kết với phần trước:
% 	•	3.1 Module View → cho biết cấu trúc code.
% 	•	3.2 Component & Connector View → cho biết các thành phần và cách chúng giao tiếp.
% 	•	3.3 Allocation View → mô tả chúng chạy trên máy chủ/vùng mạng nào, với cơ chế HA và DR ra sao trong môi trường On-Premise.

% Mục tiêu của Deployment View trong On-Premise:
% 	•	AC2 – Scalability: Scale-out bằng Node Pools và tự động phân phối workload.
% 	•	AC5 – Deployability: Tự động hóa CI/CD nội bộ, sử dụng Container Registry riêng.
% 	•	AC9 – Observability: Quản lý logs/metrics phân tán trong hạ tầng nội bộ.
% 	•	High Availability & Fault Tolerance: Đảm bảo hệ thống chạy ổn định ngay cả khi một node hoặc tủ rack gặp sự cố.

% ⸻

% 3.3.1 Deployment Architecture (On-Premise Architecture)

% Kiến trúc triển khai On-Premise vẫn tuân thủ nguyên tắc Cloud-Native, nhưng tất cả thành phần được host trên:
% 	•	Máy chủ vật lý (Bare-metal servers) với CPU/RAM cao.
% 	•	Hệ thống lưu trữ SAN/NAS cho database và persistent volumes.
% 	•	Hệ thống ảo hóa nội bộ (VMware vSphere / Proxmox / OpenStack) cho Kubernetes nodes.
% 	•	Load Balancer cứng (hardware LB) hoặc HAProxy/NGINX nội bộ.
% 	•	Private Docker Registry để lưu images.

% 🖥️ Hạ tầng phần cứng (Physical Infrastructure Layer)

% Thành phần	Mô tả
% Rack Servers	6–10 server vật lý, CPU Intel Xeon/AMD EPYC, RAM 128–256GB
% Network	Switch Layer 2/3, VLAN tách biệt (DMZ / Internal / Storage)
% Storage	SAN/NAS hỗ trợ RAID-10, Fiber Channel/iSCSI
% Management	vCenter/Proxmox Dashboard cho quản lý VM


% ⸻

% 🧩 Kubernetes Cluster (Self-Managed)

% Hệ thống chạy trên một cluster Kubernetes tự triển khai:

% Thành phần	Mô tả
% Master Nodes	3 master để đảm bảo HA (etcd cluster)
% Worker Nodes	6–12 worker (tuỳ tải)
% CNI	Calico hoặc Cilium
% Ingress	NGINX Ingress hoặc Istio Gateway
% Persistent Volumes	Rook-Ceph / Longhorn / NFS


% ⸻

% 🌐 Network Segmentation

% Vì đây là môi trường On-Premise, việc phân vùng mạng (segmentation) là cực kỳ quan trọng:
% 	•	DMZ Layer → Load Balancer + API Gateway.
% 	•	Application Layer → các pod Java/Go.
% 	•	Data Layer → PostgreSQL, MongoDB, RabbitMQ, Redis.
% 	•	Monitoring Layer → Prometheus, Grafana, Loki.

% ⸻

% 🖼️ HÌNH 3.10 – Deployment Architecture (On-Premise)

% Gợi ý khi vẽ bản đẹp (Figma / Draw.io):
% 	•	Tách thành 3 tầng: DMZ → App Nodes → Data Nodes.
% 	•	Dùng icon server để thể hiện Bare-metal.
% 	•	Dùng màu pastel để đảm bảo tính academic.

% Dưới đây là bản Mermaid để bạn xem logic:

% graph TD
%     subgraph Users
%         User(User<br/>Learner / Instructor / Admin)
%     end

%     subgraph DMZ["DMZ Network (On-Prem Load Balancer)"]
%         LB(Hardware Load Balancer<br/>HAProxy/NGINX)
%     end

%     subgraph OnPremCluster["On-Premise Kubernetes Cluster"]
%         direction TB

%         Ingress(Ingress Controller<br/>TLS Termination)

%         subgraph "Namespace: its-prod"
%             APIPod(API Gateway Pods<br/>Go)
%             JavaPods(Java Services<br/>Auth / User / Content)
%             GoPods(Go Services<br/>Scoring / Adaptive / Learner Model)
%         end

%         subgraph "Namespace: its-monitoring"
%             Mon(Prometheus • Loki • Grafana)
%         end
%     end

%     subgraph DataZone["Data Zone (Bare-metal / VMs)"]
%         PG[(🐘 PostgreSQL<br/>Primary + Standby)]
%         MG[(🍃 MongoDB<br/>Replica Set)]
%         RD[(🔥 Redis HA Cluster)]
%         MQ[(🐰 RabbitMQ<br/>Clustered)]
%     end

%     User --> LB --> Ingress
%     Ingress --> APIPod
%     Ingress --> JavaPods
%     Ingress --> GoPods

%     JavaPods -- JDBC --> PG
%     JavaPods -- Redis --> RD

%     GoPods -- Mongo --> MG
%     GoPods -- JDBC --> PG
%     GoPods -- AMQP --> MQ

%     APIPod -- AMQP --> MQ
%     Mon --- APIPod
%     Mon --- JavaPods
%     Mon --- GoPods


% ⸻

% 3.3.2 Container Specifications (On-Premise Resource Planning)

% Do hệ thống chạy On-Premise, tài nguyên cần được quy hoạch tĩnh (static capacity planning), kết hợp với autoscaling dựa trên metrics nhưng bị ràng buộc bởi tài nguyên phần cứng.

% Service	Image Size	Memory Req/Lmt	CPU Req/Lmt	Replicas	Node Pool
% API Gateway (Go)	30MB	256MB/512MB	0.25/1	3–10	Node Pool A
% Auth Service (Java)	250MB	512MB/1GB	0.5/1.5	2–5	Node Pool B
% User Service (Java)	250MB	512MB/1GB	0.5/1.5	2–5	Node Pool B
% Content Service (Java)	250MB	512MB/1GB	0.5/1.5	2–5	Node Pool B
% Scoring Svc (Go)	40MB	256MB/1GB	0.5/2	3–15	Node Pool A
% Learner Model (Go)	40MB	256MB/1GB	0.5/2	3–10	Node Pool A
% Adaptive Engine (Go)	40MB	512MB/2GB	1/3	2–8	Node Pool A


% ⸻

% 3.3.3 Infrastructure Components (On-Premise Stack)

% Dưới đây là toàn bộ thành phần hạ tầng bắt buộc khi triển khai ITS On-Premise.

% ⸻

% 1️⃣ Load Balancer (Lớp DMZ)
% 	•	Có thể dùng:
% 	•	HAProxy (active-passive)
% 	•	NGINX Plus
% 	•	F5 BIG-IP (nếu doanh nghiệp có sẵn)
% 	•	Chức năng:
% 	•	SSL termination
% 	•	Reverse proxy
% 	•	Rate limit
% 	•	WAF

% ⸻

% 2️⃣ Database & Storage Zone

% 🐘 PostgreSQL HA Cluster
% 	•	Patroni + etcd hoặc
% 	•	pgPool-II + Streaming Replication

% 🍃 MongoDB Replica Set
% 	•	Chạy 3 node trong các rack khác nhau.

% 🔥 Redis Sentinel Cluster
% 	•	Dùng local SSD của server để đạt IOPS cao.

% 🐰 RabbitMQ Cluster
% 	•	Mirrored queues (Highly Available).
% 	•	Phù hợp với Adaptive Learning & Scoring Event-driven flow.

% ⸻

% 3️⃣ Observability Stack (AC9 – On-Premise)

% Toàn bộ stack giám sát được cài trong namespace its-monitoring.
% 	•	Prometheus – metrics (scrape từ pods + node exporter)
% 	•	Loki – collector logs
% 	•	Grafana – dashboard hợp nhất
% 	•	Alertmanager – cảnh báo nội bộ (email/SMS/Zalo OA)
% 	•	OpenTelemetry Collector – gửi trace từ Java/Go đến Tempo/Jaeger
% 	•	Node Exporter – metrics phần cứng của mỗi server

% ⸻

% TÓM TẮT (TL;DR)

% Cloud	On-Premise
% Autoscaling theo tài nguyên không giới hạn	Scale bị giới hạn bởi số server vật lý
% Managed Databases	Tự quản lý DB HA (Patroni/Mongo Replica Set)
% Load Balancer do Cloud cung cấp	HAProxy/NGINX hoặc thiết bị phần cứng
% Observability Cloud Services	Tự triển khai Prometheus + Loki + Grafana
% Storage quản lý bởi cloud	SAN/NAS + CSI Driver


% ⸻

% Nếu bạn muốn, mình có thể viết thêm 3.3.4 CI/CD Pipeline (On-Premise) hoặc 3.4 Runtime View, hoặc vẽ lại Hình 3.10 theo style Draw.io đẹp chuẩn doanh nghiệp.

% Bạn muốn tiếp tục phần nào?