\section{Nghiên Cứu Điển Hình Mật Mã Ứng Dụng}

\subsection{Câu 1. (10 điểm) Phân Tích Các Chế Độ Mã Khối}

\textbf{Yêu cầu:} Với tham chiếu đến các chế độ mã khối được đề cập trong bài giảng, phân tích các kịch bản sau:

\subsubsection{(a) (4 điểm) Ứng dụng lưu trữ tệp an toàn (Tệp lớn)}

\textbf{Yêu cầu:} Một ứng dụng lưu trữ tệp an toàn cần mã hóa các tệp người dùng. So sánh các chế độ CBC, CTR và AES-GCM cho ứng dụng này, thảo luận về:
\begin{enumerate}
    \item Tác động hiệu suất đối với các tệp lớn.
    \item Lan truyền lỗi nếu các phần của văn bản mã hóa bị hỏng.
    \item Các tác động bảo mật của việc tái sử dụng IV/nonce.
    \item Đảm bảo tính toàn vẹn dữ liệu và lợi thế của mã hóa xác thực với AES-GCM.
\end{enumerate}

\begin{center}
\textbf{Trả lời:}
\end{center}

\begin{table}[H]
\centering
\begin{tabular}{|l|p{4cm}|p{4cm}|p{4cm}|}
\hline
\textbf{Tiêu chí} & \textbf{CBC} & \textbf{CTR} & \textbf{AES-GCM} \\
\hline
\textbf{Hiệu suất (Tệp lớn)} & Kém: mã hóa tuần tự, không song song hóa. & Tốt nhất: mã hóa/giải mã hoàn toàn song song. & Tốt nhất: song parallel nhờ CTR + GHASH. \\
\hline
\textbf{Lan truyền lỗi} & Lỗi khối $C_i$ $\Rightarrow$ ảnh hưởng $M_i$ và $M_{i+1}$. & Lỗi chỉ ảnh hưởng khối bị hỏng ($M_i$). & Lỗi bất kỳ $\Rightarrow$ Tag mismatch, từ chối toàn bộ. \\
\hline
\textbf{Tái sử dụng IV/Nonce} & IV tái sử dụng $\Rightarrow$ rò rỉ lặp plaintext đầu tiên. & Nonce tái sử dụng $\Rightarrow$ keystream lặp, mất bí mật. & Nonce tái sử dụng $\Rightarrow$ mất bí mật \& xác thực. \\
\hline
\textbf{Tính toàn vẹn} & Không (cần MAC riêng). & Không (cần MAC riêng). & Có (GHASH xác thực tích hợp). \\
\hline
\textbf{Kết luận} & Tránh dùng & Tốt nếu không cần xác thực & \textbf{Tốt nhất}: Hiệu suất + Xác thực \\
\hline
\end{tabular}
\caption{So sánh CBC, CTR, AES-GCM cho lưu trữ tệp lớn}
\end{table}

\textbf{Lợi thế của AES-GCM:} AES-GCM là lựa chọn tối ưu vì kết hợp:
\begin{itemize}
    \item \textbf{Hiệu suất cao}: Tận dụng song parallel hóa của CTR + GHASH.
    \item \textbf{Mã hóa Xác thực (AEAD)}: Đảm bảo tệp không bị thao túng trong lưu trữ.
    \item \textbf{Phát hiện lỗi}: Bất kỳ sai sót nào trong ciphertext đều bị phát hiện qua xác thực Tag.
\end{itemize}

\subsubsection{(b) (3 điểm) Ứng dụng nhắn tin thời gian thực (Tin nhắn ngắn)}

\textbf{Yêu cầu:} Một ứng dụng nhắn tin thời gian thực cần mã hóa các tin nhắn ngắn với độ trễ tối thiểu. So sánh các chế độ CBC, CTR và AES-GCM cho ứng dụng này, thảo luận về:
\begin{enumerate}
    \item Khả năng song parallel hóa cho mã hóa/giải mã.
    \item Tính phù hợp cho dữ liệu phát trực tuyến (streaming).
    \item Bảo vệ chống lại các cuộc tấn công văn bản mã hóa đã chọn.
    \item Cách AES-GCM giải quyết nhu cầu xác thực so với các chế độ không xác thực.
\end{enumerate}

\begin{center}
\textbf{Trả lời:}
\end{center}

\textbf{1) Khả năng song parallel hóa.} CBC mã hóa tuần tự (không song parallel); CTR và AES-GCM mã hóa hoàn toàn song parallel, phù hợp với tin nhắn ngắn/độ trễ thấp.

\textbf{2) Tính phù hợp cho streaming.} CTR hoạt động như mật mã dòng (keystream độc lập); AES-GCM cũng song parallel nhưng cần toàn bộ dữ liệu để tính Tag cuối. Cả hai đều phù hợp hơn CBC.

\textbf{3) Bảo vệ chống CCA.} CBC và CTR dễ bị Padding/Null Oracle; AES-GCM an toàn CCA nhờ xác thực Tag trước khi giải mã.

\textbf{4) Giải quyết nhu cầu xác thực.} Chỉ AES-GCM cung cấp xác thực tích hợp (GHASH). CBC, CTR cần HMAC riêng, tăng chi phí.

\textbf{Kết luận:} \textbf{AES-GCM tốt nhất} cho ứng dụng nhắn tin, vừa có hiệu suất cao vừa đảm bảo xác thực thông điệp.

\subsubsection{(c) (3 điểm) Cụ thể đối với AES-GCM}

\textbf{Yêu cầu:} Giải thích tác động bảo mật của việc tái sử dụng nonce trong AES-GCM so với CTR. Thảo luận về sự đánh đổi hiệu suất AES-GCM vs CTR+HMAC. Giải thích cách AEAD bảo vệ chống tấn công CCA.

\begin{center}
\textbf{Trả lời:}
\end{center}

\textbf{1) Tái sử dụng nonce: AES-GCM vs CTR.}
\begin{itemize}
    \item \textbf{CTR}: Nonce tái sử dụng $\Rightarrow$ keystream $G(K,N)$ lặp lại. Kẻ tấn công XOR hai ciphertext: $C_1 \oplus C_2 = M_1 \oplus M_2$. \textbf{Mất bí mật}.
    \item \textbf{AES-GCM}: Nonce tái sử dụng $\Rightarrow$ keystream lặp (mất bí mật) + GHASH không phục hồi được, cho phép giả mạo Tag. \textbf{Mất bí mật + xác thực}.
\end{itemize}

\textbf{2) Hiệu suất: AES-GCM vs CTR+HMAC.}
\begin{itemize}
    \item \textbf{AES-GCM}: GHASH song parallel, tận dụng AES-NI/PCLMULQDQ, kết hợp mã hóa + xác thực trong một lược đồ.
    \item \textbf{CTR+HMAC}: Gọi hai primitive riêng, overhead lớn hơn.
    \item \textbf{Kết luận}: AES-GCM \textbf{hiệu suất tốt hơn} CTR+HMAC.
\end{itemize}

\textbf{3) Bảo vệ chống tấn công CCA.}
\begin{itemize}
    \item \textbf{CBC/CTR}: Dẻo dai (malleable), kẻ tấn công sửa ciphertext, khai thác Oracle giải mã để học plaintext.
    \item \textbf{AEAD (AES-GCM)}: Áp dụng "Verify before Decrypt". Bất kỳ thao túng ciphertext nào $\Rightarrow$ Tag mismatch, từ chối toàn bộ trước khi giải mã. Ngăn chặn truy vấn Oracle thành công.
    \item \textbf{Kết luận}: AEAD \textbf{an toàn CCA}, CBC/CTR không.
\end{itemize}

\subsection{Câu 2. (10 điểm) Phân Tích Bảo Mật Hàm Băm}

\textbf{Mô tả hệ thống:} Một hệ thống cập nhật phần mềm sử dụng hàm băm để xác minh tính toàn vẹn:
\begin{enumerate}
    \item Nhà cung cấp đăng các băm SHA-1 của các tệp cập nhật trên trang web HTTPS.
    \item Người dùng tải xuống tệp qua HTTP (tiết kiệm băng thông).
    \item Ứng dụng xác minh tệp bằng cách tính băm SHA-1 và so sánh với băm từ HTTPS.
    \item Nếu khớp, bản cập nhật được cài đặt tự động.
\end{enumerate}

\subsubsection{(a) Xác định ít nhất ba lỗ hổng bảo mật}

\begin{center}
\textbf{Trả lời:}
\end{center}

\begin{enumerate}
    \item \textbf{Sử dụng SHA-1}: SHA-1 đã bị phá vỡ (tấn công SHAttered). Lỗ hổng: Va chạm (Collision).
    \item \textbf{Tải xuống tệp qua HTTP}: Kênh không xác thực, không bảo vệ tính toàn vẹn. Lỗ hổng: Tấn công Man-in-the-Middle (MITM).
    \item \textbf{Thiếu Xác thực Người gửi}: Chỉ xác minh tính toàn vẹn, không có MAC hoặc Chữ ký số. Lỗ hổng: Giả mạo (Impersonation).
\end{enumerate}

\subsubsection{(b) Kịch bản tấn công cụ thể}

\begin{center}
\textbf{Trả lời:}
\end{center}

\textbf{1) Tấn công Va chạm SHA-1.}
Kẻ tấn công tạo hai tệp $M_{\text{hợp pháp}}$ và $M_{\text{độc hại}}$ sao cho $H_{\text{SHA-1}}(M_{\text{hợp pháp}}) = H_{\text{SHA-1}}(M_{\text{độc hại}})$. Chặn HTTP, thay thế $M_{\text{hợp pháp}}$ bằng $M_{\text{độc hại}}$. Ứng dụng cài đặt tệp độc hại vì băm khớp.

\textbf{2) Tấn công MITM / Downgrade.}
Kẻ tấn công chặn HTTP, thay tệp mới bằng phiên bản cũ (có lỗ hổng). Nếu băm SHA-1 của phiên bản cũ vẫn trong danh sách được công bố, ứng dụng chấp nhận và cài đặt phiên bản dễ bị tấn công.

\textbf{3) Giả mạo Tệp và Băm.}
Nếu kẻ tấn công làm hỏng HTTPS/DNS, họ có thể đăng băm độc hại trên trang web giả mạo. Vì thiếu chữ ký số, người dùng không thể phân biệt.

\subsubsection{(c) Đề xuất cải tiến}

\begin{center}
\textbf{Trả lời:}
\end{center}

\begin{table}[H]
\centering
\begin{tabular}{|l|l|l|}
\hline
\textbf{Lỗ hổng} & \textbf{Đề xuất Cải tiến} & \textbf{Duy trì Hiệu suất} \\
\hline
Sử dụng SHA-1 & Nâng cấp lên SHA-256 hoặc SHA-3 & SHA-256 rất nhanh \\
\hline
Thiếu Xác thực & Ký điện tử vào băm (ECDSA/RSA) & Ký 1 lần trên hash nhỏ \\
\hline
Tải qua HTTP & Giữ tệp qua HTTP, tắt cài tự động & Buộc xác minh chữ ký trước \\
\hline
\end{tabular}
\caption{Cải tiến hệ thống cập nhật}
\end{table}

\subsubsection{(d) Thiết kế hệ thống thay thế an toàn}

\begin{center}
\textbf{Trả lời:}
\end{center}

\textbf{Kỹ thuật:} Mã hóa Xác thực (AEAD) + Mật mã Khóa Công khai (PKC).

\textbf{Quy trình Bên cung cấp:}
\begin{enumerate}
    \item Tệp cập nhật ($M$).
    \item Tính băm $H_{\text{SHA-256}}(M)$.
    \item Ký số: $\text{Sig} \leftarrow \text{PKC.Sign}(K_{\text{PR}}, H(M))$.
    \item Công bố: Tệp $M$, băm, chữ ký trên HTTPS.
\end{enumerate}

\textbf{Quy trình Bên người dùng:}
\begin{enumerate}
    \item Tải tệp $M$ qua HTTP.
    \item Tải băm $H(M)$ và $\text{Sig}$ qua HTTPS.
    \item \textbf{Bước 1 (Xác thực Người gửi):} Xác minh $\text{Sig}$ qua khóa công khai nhà cung cấp.
    \item \textbf{Bước 2 (Toàn vẹn):} Tính $H_{\text{SHA-256}}(M)$ đã tải, so sánh với $H(M)$ xác minh.
    \item \textbf{Bước 3:} Nếu cả hai thành công, cài đặt.
\end{enumerate}

\textbf{Lợi thế:}
\begin{itemize}
    \item \textbf{Bảo mật cao}: Chữ ký số $\Rightarrow$ xác thực người gửi.
    \item \textbf{Hiệu suất}: Ký/xác minh 1 lần trên hash nhỏ.
    \item \textbf{Linh hoạt}: Tệp lớn qua HTTP, hash/chữ ký qua HTTPS.
\end{itemize}

\subsection{Câu 3. (10 điểm) Thiết Kế Hệ Thống Quản Lý Mật Khẩu}

\textbf{Yêu cầu hệ thống:}
\begin{itemize}
    \item Khôi phục tài khoản an toàn khi quên mật khẩu.
    \item Kháng tấn công từ điển ngoại tuyến nếu DB bị xâm phạm.
    \item Hỗ trợ xác thực hiệu suất cao cho cơ sở người dùng lớn.
    \item Phát hiện và ngăn chặn credential stuffing.
\end{itemize}

\subsubsection{(a) Nguyên thủy mật mã để lưu trữ mật khẩu}

\begin{center}
\textbf{Trả lời:}
\end{center}

\textbf{Nguyên thủy chính:} Hàm băm mật khẩu chuyên dụng Memory-Hard như \textbf{Argon2} (hoặc Scrypt).

\textbf{Lưu trữ:}
\begin{itemize}
    \item Hàm băm của mật khẩu.
    \item Salt ngẫu nhiên duy nhất ($\ge 16$ byte).
    \item Tham số chi phí ($N, r, p$ cho Argon2).
\end{itemize}

\textbf{Tại sao:}
\begin{enumerate}
    \item \textbf{Kháng tấn công từ điển ngoại tuyến}: Salt vô hiệu hóa Bảng Rainbow.
    \item \textbf{Kháng tấn công vét cạn}: Tính Memory-Hard tăng chi phí GPU/ASIC lên đáng kể.
\end{enumerate}

\subsubsection{(b) Cơ chế khôi phục mật khẩu}

\begin{center}
\textbf{Trả lời:}
\end{center}

\textbf{Cơ chế:} Password Reset Token qua kênh email/SMS.

\textbf{Quy trình:}
\begin{enumerate}
    \item Người dùng yêu cầu đặt lại mật khẩu.
    \item Hệ thống tạo Token ngẫu nhiên, bí mật, thời gian giới hạn (ví dụ: 128 bit).
    \item Lưu trữ: Băm SHA-256 của Token + Thời gian hết hạn trong DB (không phải Token plaintext).
    \item Gửi Token (plaintext) qua email.
    \item Người dùng nhấp liên kết, hệ thống băm Token nhận được, so sánh với băm DB.
    \item Nếu khớp + chưa hết hạn, cho phép đặt mật khẩu mới.
\end{enumerate}

\textbf{Phân tích Bảo mật:}
\begin{itemize}
    \item Chỉ lưu băm Token $\Rightarrow$ ngay cả khi DB xâm phạm, kẻ tấn công không sử dụng được Token.
    \item An ninh phụ thuộc vào: tính bí mật + thời gian giới hạn Token.
    \item Chuyển gánh nặng bảo mật sang kênh email người dùng.
\end{itemize}

\subsubsection{(c) Cân bằng Bảo mật vs Hiệu suất}

\begin{center}
\textbf{Trả lời:}
\end{center}

\textbf{1) Bảo mật ưu tiên:}
Tham số chi phí cao cho Argon2 (ví dụ: thời gian băm $\approx 500$ms).

\textbf{2) Cân bằng Hiệu suất:}
\begin{itemize}
    \item \textbf{Xác thực}: Quá trình băm chậm chỉ xảy ra 1 lần/đăng nhập. Độ trễ 500ms chấp nhận được.
    \item \textbf{Tải lớn}: Argon2 hỗ trợ song parallel (tham số $p$) để phân tán tải băm trên nhiều lõi CPU $\Rightarrow$ hỗ trợ cơ sở người dùng lớn.
\end{itemize}

\subsubsection{(d) Lỗ hổng tiềm ẩn và giảm thiểu}

\begin{center}
\textbf{Trả lời:}
\end{center}

\begin{table}[H]
\centering
\begin{tabular}{|l|l|}
\hline
\textbf{Lỗ hổng} & \textbf{Giảm thiểu (Mitigation)} \\
\hline
\textbf{Credential Stuffing} & Theo dõi tần suất đăng nhập thất bại. Triển khai MFA bắt buộc. \\
\hline
\textbf{Timing Side-Channel} & So sánh hàm băm bằng constant-time comparison. \\
\hline
\textbf{Token bị đánh cắp} & TTL ngắn (15 phút), one-time use, đăng xuất toàn bộ thiết bị. \\
\hline
\end{tabular}
\caption{Lỗ hổng tiềm ẩn và cách giảm thiểu}
\end{table}
