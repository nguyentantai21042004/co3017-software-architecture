% Chuẩn bị trước mỗi tuần

\subsubsection{Tuần 1-4: Nền tảng SOLID}
Tập trung nắm vững các nguyên tắc SOLID - là nền tảng cho toàn bộ môn học

\subsubsection{Tuần 5-7: Tư duy kiến trúc}
Rèn luyện tư duy phân tích và thiết kế hệ thống

\subsubsection{Tuần 8-10: Mô hình kiến trúc}
Nghiên cứu kỹ các mô hình kiến trúc, so sánh ưu nhược điểm

\subsubsection{Tuần 11: Seminar}
Chuẩn bị tốt cho seminar, chọn chủ đề phù hợp

\subsubsection{Tuần 12-13: Tài liệu hóa}
Thực hành viết tài liệu và ADRs

\subsubsection{Lưu ý quan trọng}

\begin{itemize}
\item \textbf{Tuần 3 và 10}: Có video học tập - cần chuẩn bị và xem trước
\item \textbf{Tuần 9}: Có phần tự học - cần chủ động nghiên cứu Event-driven và Space-based Architecture
\item \textbf{Deadline cứng}: Assignment phải nộp trước 23h59 ngày 07/12/2025
\item \textbf{Thuyết trình}: Chuẩn bị kỹ lưỡng cho presentation tuần cuối
\end{itemize}

\subsubsection{Chiến lược học tập}

\begin{enumerate}
\item \textbf{Nắm vững lý thuyết}: Đặc biệt là SOLID principles và các architectural patterns
\item \textbf{Thực hành tích cực}: Tham gia đầy đủ 2 bài tập trong lớp
\item \textbf{Nghiên cứu sâu}: Chuẩn bị tốt cho phần advanced topics
\item \textbf{Quản lý thời gian}: Bắt đầu làm assignment sớm để tránh áp lực deadline
\end{enumerate}

\subsubsection{Kết quả đạt được sau khóa học}

Sau khi hoàn thành môn học, sinh viên sẽ có khả năng:
\begin{itemize}
\item Thiết kế kiến trúc phần mềm cho các hệ thống phức tạp
\item Áp dụng các nguyên tắc SOLID trong thực tế
\item Lựa chọn mô hình kiến trúc phù hợp cho từng dự án
\item Tài liệu hóa và truyền đạt ý tưởng kiến trúc hiệu quả
\item Đưa ra quyết định kiến trúc có căn cứ và quản lý chúng
\end{itemize}